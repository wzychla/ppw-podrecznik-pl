\subsection{Konwersje między typami}
\label{KonwersjeMiedzyTypami}

Możliwość przypisania wprost wartości jednego 
typu do wartości innego typu zależy tylko od tego, czy zdefiniowano bezpośredni operator
konwersji między tymi typami. Najczęściej taka próba nie powiedzie się, bowiem operatory
konwersji zdefiniowano tylko dla wybranych par typów.

Zobaczmy jak kompilator reaguje na próbę wymuszenia
konwersji między wartościami różnych typów gdy operator konwersji bezpośredniej istnieje tylko
dla konwersji w jedną stronę.

\begin{scriptsize}
\begin{verbatim}
/* Wiktor Zychla, 2003 */
using System;

namespace Example
{
  public class CMain
  {
    public static void Main()
    {
      int  i=1;
      long j=1;

      j=i;
      i=j;
    }
  }
}

Microsoft (R) Visual C# .NET Compiler version 7.00.9466
for Microsoft (R) .NET Framework version 1.0.3705
Copyright (C) Microsoft Corporation 2001. All rights reserved.

example.cs(14,6): error CS0029: Cannot implicitly convert type 'long' to 'int'
\end{verbatim}
\end{scriptsize}

Wszystko się zgadza - konwersja z wartości 32-bitowej do wartości 64-bitowej jest legalna (tak została
określona) bowiem nie powoduje utraty danych. Konwersja odwrotna może prowadzić do utraty danych i
choć kompilator taką konwersję dopuści, programista musi swój zamiar potwierdzić:

\begin{scriptsize}
\begin{verbatim}
j=i;
i=(int)j;
\end{verbatim}
\end{scriptsize}

\subsubsection{Typy wbudowane}

Wartość każdego typu można przekształcić do wartości typu {\em string} za pomocą 
metody {\bf ToString()}. Wbudowane klasy arytmetyczne mają statyczne metody {\em Parse}, 
służące do konwersji napisów.

\begin{scriptsize}
\begin{verbatim}
/* Wiktor Zychla, 2003 */
using System;

namespace Example
{
  public class CMain
  {    
    public static void Main()
    {
       string sI = "45";
       string sF = "123,5";

       int    i = int.Parse( sI );
       float  f = float.Parse( sF );

       Console.WriteLine( "{0}; {1}", i, f );
    }
  }
}
\end{verbatim}
\end{scriptsize}

Bardziej ogólne podejście możliwe jest dzięki konwersjom między typami wbudowanymi,
dostępnymi jako statyczne metody klasy {\em System.Convert}. 

\begin{scriptsize}
\begin{verbatim}
/* Wiktor Zychla, 2003 */
using System;

namespace Example
{
  public class CMain
  {    
    public static void Main()
    {
       string sI = "45";
       string sF = "123,5";

       int    i = Convert.ToInt32( sI );
       double f = Convert.ToDouble( sF );

       Console.WriteLine( "{0}; {1}", i, f );
    }
  }
}
\end{verbatim}
\end{scriptsize}

\subsubsection{Typy własne}

Możliwe są dwa rodzaje konwersji wprost (przez przypisanie):
\begin{description}
\item [konwersja jawna] Z konwersją jawną mamy do czynienia wtedy, kiedy wymiana wartości
między dwoma typami wymaga jawnego rzutowania, na przykład:
\begin{scriptsize}
\begin{verbatim}
double d = 1.5;
int    i = (int)d;
\end{verbatim}
\end{scriptsize}
\item [konwersja niejawna] Z konwersją niejawną mamy do czynienia wtedy, kiedy wymiana wartości
między dwoma typami nie wymaga jawnego rzutowania, na przykład:
\begin{scriptsize}
\begin{verbatim}
int    i = 1;
double d = i;
\end{verbatim}
\end{scriptsize}
\end{description}

W przypadku typów wbudowanych rodzaje konwersji są już zdeterminowane, jednak w przypadku własnych klas 
programista staje przed wyborem rodzajów konwersji między swoim typem a innymi typami.
Wybór odpowiedniego typu konwersji zależy od tego czy podczas konwersji może dojść do utraty danych, czy nie.
Wyobraźmy sobie sytuację, w której typ A jest bardziej pojemny informacyjnie niż typ B. Wobec tego 
konwersja danej typu B do danej typu A może być niejawna, ponieważ nie istnieje ryzyko utraty informacji.
Konwersja w odwrotną stronę powinna być jawna, aby klient klasy był świadomy możliwej utraty informacji.

O tym, czy konwersja jest dokonywana jawnie, czy niejawnie, decyduje
definicja operatora konwersji. Operatory konwersji definiuje się jako jawne przez użycie kwalifikatora
{\em explicit} lub niejawne {\em implicit}. 

Jako przykład rozważmy strukturę opisującą liczby rzymskie. Konwersja liczby rzymskiej do
typu {\em int} może być niejawna, bowiem {\em int} jest bardziej pojemny informacyjnie niż zbiór
wszystkich liczb rzymskich. Konwersja odwrotna powinna być jawna.

\begin{scriptsize}
\begin{verbatim}
/* Wiktor Zychla, 2003 */
using System;
using System.Text;

namespace Example
{
  public class CMain
  {    
    struct RomanNumeral
    {
      readonly string[] literals;
      int value;

      public RomanNumeral( int value )
      {
        if ( value > 3999 ) throw new ArgumentOutOfRangeException();

        literals = new string[] { "I", "V", "X", "L", 
                                  "C", "D", "M" };        
        this.value = value;
      }

      public static explicit operator RomanNumeral( int value )
      {
        return new RomanNumeral( value );
      }

      public static implicit operator int(RomanNumeral roman)
      {
         return roman.value;
      }

      string BuildRomanString( int index, int v )
      {  
         if ( v <= 0 ) return String.Empty;       

         string  s = String.Empty;
         int digit = v%10;
         int j;

         if ( digit == 4 ) 
           s = literals[index]+literals[index+1]+s;
         else if ( digit == 9 ) 
           s = literals[index]+literals[index+2]+s;
         else if ( digit >= 5 && digit <= 8 )         
         {
            s = literals[ index+1 ];
            digit -= 5;
         }
         if ( digit >= 1 && digit <=3 )
         {
            for ( j=0; j<digit; j++ )
              s = s+literals[ index ];
         } 

         return BuildRomanString( index+2, v/10 )+s;
      }
    
      public override string ToString()
      {
        string s = String.Empty;
        int tVal = value;

        return BuildRomanString( 0, value );
      }
	}

    public static void Main()
    {
      int          i = 2003;
      RomanNumeral r = (RomanNumeral)i;
      int          j = r;

      Console.WriteLine( "{0}, {1}, {2}", i, r, j );
    }
  }
}
\end{verbatim}
\end{scriptsize}

\subsubsection{Dziedziczenie a konwersje}

Jak w każdym obiektowym języku programowania, obiekt klasy potomnej można zawsze
niejawnie zrzutować na obiekt klasy macierzystej. W drugą stronę możliwa jest tylko
konwersja jawna, jednak uda się ona tylko wtedy, kiedy dany obiekt jest rzeczywiście
odpowiedniego typu.

Oprócz rzutowania statycznego, między typami referencyjnymi możliwe jest również
rzutowanie dynamiczne za pomocą operatora {\em as}. Jeśli wartość rzutowania dynamicznego
równa jest {\em null}, to znaczy że rzutowanie nie powiodło się. Rzutowanie dynamiczne
pozwala uniknąć wyjątku, który byłby wyrzucony przez rzutowanie statyczne przy błędzie rzutowania,
na przykład:

\begin{scriptsize}
\begin{verbatim}
/* Wiktor Zychla, 2003 */
using System;
using System.Collections;

namespace Example
{ 
  public class CMain
  {    
    static void f1( object o ) 
    {
      // możliwy wyjątek!
      Hashtable h = (Hashtable)o; 
    }
    static void f2( object o ) 
    {
      // rzutowanie może nie udać się
      // ale nie będzie wyjątku
      Hashtable h = o as Hashtable; 
      if ( h == null )
      {
      }
    }
    public static void Main()
    {
       ArrayList a = new ArrayList();

       f1( a );
       f2( a );
    }
  }
}
\end{verbatim}
\end{scriptsize}

\subsection{Wyjątki}

Wyjątki są mechanizmem jaki nowoczesne języki programowania wykorzystują w celu zwiększenia
produktywności i czytelności kodu. Projektanci platformy \NET{} przyjęli, że biblioteki systemowe
informują kod użytkownika o błędach za pomocą wyjątków i zachęcają użytkowników do zerwania z przyzwyczajeniami,
znanymi na przykład z C, polegającymi na przekazywaniu informacji o błędach przez wartości funkcji.

Wyjątki przechwytuje się za pomocą składni

\begin{scriptsize}
\begin{verbatim}
try
{
  // blok w którym przechwytywane będą wyjątki
}
catch ( TypWyjątku wyjątek )
{
  // obsługa przechwyconego wyjątku
}
finally 
{
  // kod wykonywany zawsze na zakończenie bloku
}
\end{verbatim}
\end{scriptsize}

Wyjątki {\em wyrzuca} się za pomocą składni

\begin{scriptsize}
\begin{verbatim}
throw <obiekt opisujący wyjątek>
\end{verbatim}
\end{scriptsize}

Istnieją trzy podstawowe strategie dotyczące wyjątków. Wybór odpowiedniej strategii należy do programisty:

\begin{description}
\item [brak obsługi wyjątków] Kod funkcji nie zawiera obsługi wyjątków
\item [informacja dla funkcji wołającej] Kod klauzuli catch wyrzuca przechwycony wyjątek 
\item [pełne obsłużenie wyjątku] Kod klauzuli catch zawiera pełny kod obsługi wyjątku i nie informuje funkcji
wołającej o problemach
\end{description}

Wyjątki są obiektami klas, które dziedziczą z klasy {\em Exception}. Klasa ta może być przeciążona, choć
już w podstawowej wersji zawiera sposo użytecznych informacji, takich jak informacja diagnostyczna czy
ślad stosu, na przykład:

\begin{scriptsize}
\begin{verbatim}
/* Wiktor Zychla, 2003 */
using System;
using System.Collections;

namespace Example
{ 
  public class CMain
  {    
    public static void f()
    {
      try
      {
        int i = 1-1;
        int j = 1/i; 
      }
      catch ( Exception ex )
      {
        Console.WriteLine( "Informacja o błędzie:\r\n{0}\r\nSlad stosu:\r\n{1}", 
                           ex.Message, ex.StackTrace );
      }
    }
    public static void Main()
    {
      f();
    }
  }
}

C:\Example>csc.exe example.cs /debug
Microsoft (R) Visual C# .NET Compiler version 7.00.9466
for Microsoft (R) .NET Framework version 1.0.3705
Copyright (C) Microsoft Corporation 2001. All rights reserved.


C:\Example>example.exe
Informacja o błędzie:
Attempted to divide by zero.
Slad stosu:
   at Example.CMain.f() in C:\000\example.cs:line 14
\end{verbatim}
\end{scriptsize}
