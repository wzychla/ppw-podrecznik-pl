\subsection{Klasa {\em string}}

\subsubsection{Podstawowe możliwości klasy {\em string}}

Obsługa napisów możliwa jest w C\# dzięki klasie {\em string}. Każdy obiekt tej klasy ma dostęp do
wielu przydatnych propercji i metod. 

\begin{scriptsize}
\begin{verbatim}
/* Wiktor Zychla, 2003 */
using System;
using System.Collections;

namespace Example
{ 
  public class CMain
  {    
    public static void Main()
    {
      string s = "Ala ma kota";
      Console.WriteLine( "Length('{0}'): {1}", s, s.Length );
    	
      Console.WriteLine( "ToLower: '{0}', ToUpper: '{1}'", 
                        s.ToLower(), s.ToUpper() );
    	
      string sS = "Ala";
      if ( s.StartsWith( sS ) )
        Console.WriteLine( "'{0}' StartsWith '{1}'", s, sS );
    	
      string sE = "ota";
      if ( s.EndsWith( sE ) )
        Console.WriteLine( "'{0}' EndsWith '{1}'", s, sE );
    	
      Console.WriteLine( "Remove(2,3): '{0}'", s.Remove( 2, 3 ) );
    	
      string sI = "!++!";
      Console.WriteLine( "Insert(3, {1}): '{0}'", s.Insert( 3, sI ), sI );
    	
      Console.WriteLine( "Substring(4, 2): '{0}'", s.Substring( 4, 2 ) );
    	
      Console.WriteLine( "IndexOf( 'a' ): '{0}'", s.IndexOf( 'a' ) );
      Console.WriteLine( "LastIndexOf( 'a' ): '{0}'", s.LastIndexOf( 'a' ) );    	      
    	
      Console.WriteLine( "PadLeft( 20, '_' ): '{0}'", s.PadLeft( 20, '_' ) );
    	
      string sT = "   qwerty    ";
      Console.WriteLine( "Trim('{0}'): '{1}'", sT, sT.Trim() );
    }
  }
}

C:\Example>example.exe
Length('Ala ma kota'): 11
ToLower: 'ala ma kota', ToUpper: 'ALA MA KOTA'
'Ala ma kota' StartsWith 'Ala'
'Ala ma kota' EndsWith 'ota'
Remove(2,3): 'Ala kota'
Insert(3, !++!): 'Ala!++! ma kota'
Substring(4, 2): 'ma'
IndexOf( 'a' ): '2'
LastIndexOf( 'a' ): '10'
PadLeft( 20, '_' ): '_________Ala ma kota'
Trim('   qwerty    '): 'qwerty'
\end{verbatim}
\end{scriptsize}

\subsubsection{Formatowanie}

C\# udostępnia bardzo elegancki sposób formatowania napisów, przypominającą trochę sposób formatowania
znany z C, jednak znacznie udoskonalony. Programista przygotowuje formatowany napis, przetykany wyrażeniami
formatującymi, zawierającymi numery kolejnych parametrów dla formatowanego napisu wraz ze wskazaniem
sposobu formatowania.

Zobaczmy najpierw prosty przykład:

\begin{scriptsize}
\begin{verbatim}
/* Wiktor Zychla, 2003 */
using System;
using System.Collections;

namespace Example
{ 
  public class CMain
  {    
    public static void Main()
    {
      string s = "{0}+{1}={2}\r\n{0}+{2}={3}\r\n{2}+{3}={4}";
      int i0 = 17, i1 = 23;	
      Console.WriteLine( s, i0, i1, i0+i1, 2*i0+i1, 3*i0+2*i1 );
    }
  }
}
\end{verbatim}
\end{scriptsize}

Wyższość takiego sposobu przekazywania parametrów dla wyrażeń formatujących nad tym dostępnym w C polega na tym,
że wyrażenie formatujące zawiera w sobie numer parametru, w związku z czym ten sam parametr może w napisie
występować wiele razy bez konieczności powtarzania go na liście parametrów, na przykład:

\begin{scriptsize}
\begin{verbatim}
Console.WriteLine( "{0}, {0}, {0}, {0}", 1 );
\end{verbatim}
\end{scriptsize}

Wyniki formatowania mogą być przekazane nie tylko do konsoli, ale do zmiennej typu {\em string}, dzięki
statycznej funkcji {\em Format} w klasie {\em string}, na przykład:

\begin{scriptsize}
\begin{verbatim}
string s = "{0}+{1}={2}\r\n{0}+{2}={3}\r\n{2}+{3}={4}";
int i0 = 17, i1 = 23;	

string sResult = String.Format( s, i0, i1, i0+i1, 2*i0+i1, 3*i0+2*i1 );
\end{verbatim}
\end{scriptsize}

Standardowy sposób formatowania wartości numerycznych pozwala, oprócz numeru parametru, dodać do
wyrażenia formatującego także długość pola oraz sposób formatowania, na przykład:

\begin{scriptsize}
\begin{verbatim}
Console.WriteLine(
    "{0,5} {1,5}", 123, 456);      // wyrównaj do prawej
Console.WriteLine(
    "{0,-5} {1,-5}", 123, 456);    // wyrównaj do lewej

  123   456
123   456
\end{verbatim}
\end{scriptsize}

Dostępne sposoby formatowania wyrażeń numerycznych:

\begin{center}
$$\begin{array}{ll}
\mbox{Znak formatujący} & \mbox{Interpretacja}  \\
\hline \\
\mbox{C lub c} & \mbox{Finansowa}   \\
\mbox{D lub d} & \mbox{Dziesiętna}  \\
\mbox{E lub e} & \mbox{Wykładnicza} \\
\mbox{F lub f} & \mbox{Ustalona ilość pozycji dziesiętnych} \\
\mbox{G lub g} & \mbox{Ogólna}      \\
\mbox{N lub n} & \mbox{Numeryczna}  \\ 
\mbox{P lub p} & \mbox{Procentowa}  \\
\mbox{R lub r} & \mbox{Możliwa do ponownego sparsowania} \\
\mbox{X lub x} & \mbox{Heksadecymalna}
\end{array}$$
\end{center}

Przykład:

\begin{scriptsize}
\begin{verbatim}
/* Wiktor Zychla, 2003 */
using System;
using System.Collections;

namespace Example
{ 
  public class CMain
  {    
    public static void Main()
    {
      int i = 123456;
      Console.WriteLine("{0:C}", i); 
      Console.WriteLine("{0:D}", i); 
      Console.WriteLine("{0:E}", i); 
      Console.WriteLine("{0:F}", i); 
      Console.WriteLine("{0:G}", i); 
      Console.WriteLine("{0:N}", i); 
      Console.WriteLine("{0:P}", i); 
      Console.WriteLine("{0:X}", i); 
      
      Console.WriteLine();
      
      double d = 123.456;
      Console.WriteLine("{0:E}", d); 
      Console.WriteLine("{0:F}", d); 
      Console.WriteLine("{0:G}", d); 
      Console.WriteLine("{0:N}", d); 
      Console.WriteLine("{0:P}", d); 
      Console.WriteLine("{0:R}", d);     	
    }
  }
}

C:\Example>example.exe
123_456,00 zł
123456
1,234560E+005
123456,00
123456
123_456,00
12_345_600,00%
1E240

1,234560E+002
123,46
123,456
123,46
12_345,60%
123,456
\end{verbatim}
\end{scriptsize}

Programista może wyposażyć własne obiekty w możliwość dowolnego zadawania parametrów formatowania. 
Umożliwia to interfejs {\em IFormattable}.

\begin{scriptsize}
\begin{verbatim}
/* Wiktor Zychla, 2003 */
using System;

namespace Example
{
  class CFormatExample : IFormattable
  {
     public int    ie;
     public string se;

     private CFormatExample() {}
     public  CFormatExample( int ie, string se )
     {
       this.ie = ie; this.se = se;
     }
     public string ToString( string format, IFormatProvider fp )
     {
       switch ( format )
       {
          case "A" : return ie.ToString();
          case "B" : return se;
          default  : return String.Format( "{0}:{1}", ie, se );

       } 
     } 
  }
  public class CExample 
  {
    public static void Main(string[] args)
    {
      CFormatExample fe = new CFormatExample( 17, "Ala ma kota" );
       
      Console.WriteLine( String.Format( "{0}", fe ) );
      Console.WriteLine( String.Format( "{0:A}", fe ) );
      Console.WriteLine( String.Format( "{0:B}", fe ) );
    }
  }
}

C:\Example>example.exe
17:Ala ma kota
17
Ala ma kota
\end{verbatim}
\end{scriptsize}


\subsubsection{Encoding}

Klasyczny problem związany z obsługą napisów to konwersja napisu do tablicy znaków i tablicy znaków do
napisu. Klasycznie do tego problemu podchodzi się traktując napis jako tablicę znaków (tak jest na przykład
w C).

Zauważmy jednak, że istnieje wiele możliwych standardów kodowania znaków. Niekoniecznie kod znaku w standardzie
ASCII musi być taki sam jak w standardzie UNICODE, pewne znaki mogą wręcz być niedostępne w jednych standardach
a dostępne w innych.

W C\# udostępniono klasę {\em System.Text.Encoding}, za pomocą której można konwertować napisy i tablice
znaków w następujących standardach:

\begin{center}
$$\begin{array}{ll}
\mbox{Standard kodowania} & \mbox{Uwagi}  \\
\hline \\
\mbox{ASCII} & \mbox{}   \\
\mbox{BigEndianUnicode} & \mbox{}  \\
\mbox{Unicode} & \mbox{} \\
\mbox{UTF7} & \mbox{Unicode, strona kodowa 65000} \\
\mbox{UTF8} & \mbox{Unicode, strona kodowa 65001}
\end{array}$$
\end{center}

Przykład:

\begin{scriptsize}
\begin{verbatim}
/* Wiktor Zychla, 2003 */
using System;
using System.Text;

namespace Example
{ 
  public class CMain
  {    
    public static void Main()
    {
        byte[] ba = new byte[]
            {72, 101, 108, 108, 111};
        string s = Encoding.ASCII.GetString(ba);
        Console.WriteLine(s);
    	
    	string sP = "Chrząszcz chrzęści w Żyrardówku";
    	byte[] unicodeB = Encoding.Unicode.GetBytes( sP );
    	char[] unicodeC = Encoding.Unicode.GetChars( unicodeB );

        Console.WriteLine();
    	foreach ( byte b in unicodeB )
    	  Console.Write( "{0:D3} ", b );
        Console.WriteLine();
    	foreach ( char c in unicodeC )
    	  Console.Write( "{0} ", c );
    }
  }
}

C:\Example>example
Hello

067 000 104 000 114 000 122 000 005 001 115 000 122 000 099 000 122 000 032 000
099 000 104 000 114 000 122 000 025 001 091 001 099 000 105 000 032 000 119 000
032 000 123 001 121 000 114 000 097 000 114 000 100 000 243 000 119 000 107 000
117 000
C h r z ą s z c z   c h r z ę ś c i   w   Ż y r a r d ó w k u
\end{verbatim}
\end{scriptsize}

