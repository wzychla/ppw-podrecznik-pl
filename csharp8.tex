\subsection{Porównanie C\# z innymi językami}

C\#, Java oraz C++ mają wspólne korzenie, stąd C\# ma z Javą i C++ zdecydowanie więcej elementów wspólnych
niż z innymi językami. 

C\# wśród współczesnych języków obiektowych zajmuje miejsce szczególne. Łączy bowiem w sobie 
bardzo wysoką wydajność (co stawia go obok C++), z niezwykle eleganckim modelem obiektowym (co stawia go obok
SmallTalka). 

\subsubsection{C\# a C++}

C++ było próbą zbudowania języka obiektowego na bazie składni języka C. C\# jest językiem
od początku do końca zaprojektowanym jako język obiektowy. Podobieństwo składni jest
zabiegiem celowym, wprowadzonym po to aby ułatwić programistom przejście ze świata 
C i C++ do C\#\footnote{Jest to naprawdę duża zaleta. C\# jako zupełnie nowy język mógł mieć
przecież zupełnie nową składnię i sposób pisania kodu (na przykład z góry na dół i z prawa na lewo).}.

Kilka ważniejszych różnic między C\# a C++:

\begin{itemize}
\item W C\# z punktu widzenia programisty wszystko jest obiektem. 
\item System typów C\# jest o wiele silniejszy niż w C++. Typy śledzone są dynamicznie, to znaczy
że nawet w czasie wykonania programu nie ma możliwości konwersji pomiędzy wartościami niezgodnych typów.
\item W C\# nie ma plików nagłówkowych, ponadto kolejność klas w projekcie nie ma znaczenia.
\item W C\# testowanie warunków wymaga wyrażeń typu {\tt bool}. W C++ można zamiennie wykorzystywać
w takich przypadkach wyrażenia typu {\tt int}, na przykład {\tt if (1)}.
\item W C\# nie ma jawnej destrukcji obiektów. Niszczeniem nieużywanej pamięci zajmuje się odśmiecacz.
\item W C\# nie ma szablonów. Jednorodny model obiektowy pozwala pisać kod elegantszy niż przy pomocy
szablonów w C++.
\item Przekazywanie błędów w C\# odbywa się za pomocą wyjątków. Ta reguła stosowana jest konsekwentnie.
\item Model obiektowy C\# dopuszcza tylko pojedyńcze dziedziczenie z możliwością implementowania wielu
interfejsów.

\end{itemize}

\subsubsection{C\# a Java}

Na temat podobieństw i różnic między C\# a Javą toczy się wiele dyskusji. Z pewnością C\# nie zajmie
miejsca Javy, bowiem zakresy stosowalności C\# i Javy nieco się rozmijają: Java była, jest i będzie
nadal najlepszym wyborem jeśli chodzi o aplikacje przenośne. Z drugiej strony, C\# pozwala na rozwiązanie
tych samych problemów co Java, tyle że przy użyciu znacznie wydajniejszego środowiska i przy użyciu
prostszych technik. 

Kilka ważniejszych różnic między C\# a Javą:

\begin{itemize}
\item Typy proste w Javie tworzą osobny świat. Aby traktować typy proste i typy referencyjne w sposób jednorodny
należy umieszczać obiekty typów prostych w obiektach referencyjnych, za pomocą tzw. klas {\em wrapperów}.
Typy proste w C\# są z punktu widzenia modelu obiektowego takie same jak typy referencyjne. Obiekt typu
prostego jest obiektem - ma pola, metody i inne właściwości. W C\# wyrażenia {\tt 5.ToString()} czy
{\tt (5-x).ToString()} są jak najbardziej poprawne. W C\# programista może definiować własne typy proste, 
zastępując słowo {\em class} słowem {\em struct}.

\item W C\# istnieje możliwość przekazania parametrów przez referencje za pomocą {\em ref} i {\em out}. W Javie
parametry są zawsze przekazywane przez wartość.

\item Iterfejsy w Javie mogą zawierać składowe, w C\# nie. Jednak w C\# istnieje możliwość implementowania 
interfejsów, których funkcje składowe mają te same nazwy.

\item Java nie wykrywa przepełnień podczas obliczeń matematycznych. W C\# można wymusić wykrywanie przepełnień
za pomocą bloku osygnowanego jako {\em checked}.

\item W C\# wolno pisać kod korzystajacy ze wskaźników.

\item Mechanizm delegatów, dostępny w C\#, jest bardzo elegancki i pozwala nie tylko
zapanować nad wskaźnikami do funkcji, ale także rozwiązuje problem zdarzeń prościej niż w Javie.

\end{itemize}

Jak pisze w {\em Thinking in C\#} Bruce Eckel, jeden z długoletnich propagatorów Javy:

\begin{quote}
Java has succeeded in two key areas: as the dominant language for writing server-side applications
and as the top language for teaching computer science in colleges. The .NET Framework is better
for both these areas, although it is not inevitable that it will become dominant in either.
For writing client applications, there is no question that C\# clearly 
outstrips Java.\footnote{{\em Java sprawdziła się na dwóch kluczowych frontach: jako 
język do oprogramowywania aplikacji po stronie serwera oraz jako język do nauczania informatyki 
na studiach. .NET Framework jest lepszy i tu i tam, jednak nie koniecznie zdominuje Javę w którymkolwiek
zakresie. Z pewnością jednak jeśli chodzi o aplikacje klienckie, C\# zostawia Javę daleko w tyle.}}
\end{quote}
