\chapter*{Zamiast wstępu}

\section*{Dla kogo jest ten skrypt}

Skrypt skierowany jest do programistów, którzy chcą dowiedzieć się jakich narzędzi i języków używać
aby pisać programy pod Windows oraz jak wygląda sam system widziany oczami programisty. Powstał jako
materiał pomocniczny do wykładu "Programowanie pod Windows", układ materiału odpowiada więc 
przebiegowi wykładu.

Zakładam, że czytelnik potrafi programować w C, wie co to jest kompilator, kod źródłowy i wynikowy, zna
trochę C++ lub Javę. Dość dokładnie omawiam elementy języka C\#, można więc rozdział poświęcony omówieniu
tego języka potraktować jako mini-leksykon C\#.

Poznawanie nowych języków i metod programwania traktuję jako nie tylko pracę ale i bardzo uzależniające
hobby. Ucząc się nowych rzeczy, czytam to co autor ma do powiedzenia na ich temat, a potem
staram się dokładnie analizować listingi przykładowych programów. Niestety, bardzo często zdarza się, że
kody przykładowych programów w książkach są koszmarnie długie! Autorzy przykładów być może kierują się
przekonaniem, że przykładowy kod powinien wyczerpywać demonstrowane zagadnienie w sposób pełny, a
ponadto zapoznać czytelnika przy okazji z paroma dodatkowymi, czasami niezwiązanymi z tematem, elementami. Tylko
jak, chcąc nauczyć się czegoś szybko, znaleźć czas na analizę czasami kilkunastu stron kodu źródłowego,
aby między 430 a 435 wierszem znaleźć interesujący mnie fragment?

Nie potrafię odpowiedzieć na to pytanie. Dlatego kody przykładowych programów w tym skrypcie są bardzo
krótkie, czasami wręcz symboliczne. Zakładam bowiem, że programista który chce na przykład dowiedzieć się
jak działa {\em ArrayList} nie potrzebuje jako przykładu 10 stron kodu źródłowego prostej aplikacji bazodanowej,
tylko 10-15 linijek demonstrujących użycie tego a nie innego obiektu. Mimo to przeważająca większość 
przykładów to kompletne programy, gotowe do uruchomienia.

Zapraszam do lektury.

\section*{Uwagi do wersji 1.01}

Przygotowana w roku 2003 wersja 0.99 niniejszych notatek przez wiele lat z powodzeniem zaspokajała podstawowe
potrzeby informacyjne w temacie programowania w systemie Windows. Wraz z wprowadzaniem nowych elementów do języka C\#
z czasem niedostatek informacyjny stał się mocno odczuwalny. W 2017 rozpoczęto więc proces aktualizacji notatek, który 
ma za zadanie zasypanie luki jaką niewątpliwie był brak poruszenia tak ważnych tematów jak chociażby programowanie generyczne
czy LINQ. 

Po raz kolejny zapraszam do lektury jak również do dzielenia się uwagami, które w miarę możliwości będą uwzględniane
przy przygotowywaniu kolejnych wersji.