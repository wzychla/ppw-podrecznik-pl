\subsection{Kolekcje wbudowane i {\em System.Collections}}

Przygotowując swoją aplikację do określonych zadań, programista musi zmierzyć się z dwoma czynnikami
kształtującymi jej obraz: algorytmami i strukturami danych. O ile konkretne algorytmy są zwykle zależne
od postaci problemu, który aplikacja ma rozwiązywać, o tyle te same struktury danych spotyka się niemal co chwila.

Pewną bardzo specjalną grupę struktur danych stanowią byty, które bardzo ogólnie możnaby nazwać 
{\bf kontenerami}. Zadaniem kontenerów jest {\em grupowanie} w większe struktury obiektów, 
które z jakichś powodów powinny być trzymane razem. Różne języki programowania wspomagają programistów 
w tym zakresie w różny sposób: w C tablice są częścią języka, wszystkie inne struktury danych programista
musi przygotować sam; w C++ możliwości C rozszerzono przez dodanie biblioteki szablonów STL, w której
programista może znaleźć większość potrzebnych rodzajów kontenerów oraz algorytmy do operowania na nich.

Programista tworząc aplikację w C\# ma do dyspozycji solidną bibliotekę kontenerów (zwanych tu {\em kolekcjami}),
zgromadzone w przestrzeni nazw {\em System.Collections}. 

\subsubsection{Tablice}

Tablice są najprostszymi kontenerami. W C\#, podobnie jak w wielu innych językach, 
programista po określeniu rozmiaru tablicy nie ma wprost możliwości zmiany jej rozmiaru. Z tego powodu
tablice przydają się najczęściej tam, gdzie ilość elementów jest z góry znana. 

\begin{scriptsize}
\begin{verbatim}
/* Wiktor Zychla, 2003 */
using System;

namespace Example
{
  public class CExample 
  {
    public static void Main(string[] args)
    {
      Console.Write( "Podaj ilosc elementow tablicy: " );

      int     n = int.Parse( Console.ReadLine() );
      int[] tab = new int[n];

      for ( int i=0; i<n; i++ )
        tab[i] = 2*i+1;

      for ( int i=0; i<n; i++ )
        Console.WriteLine( "{0} element tablicy -> {1}", i, tab[i] );
    }
  }
}

C:\Example>example.exe
Podaj ilosc elementow tablicy: 7
0 element tablicy -> 1
1 element tablicy -> 3
2 element tablicy -> 5
3 element tablicy -> 7
4 element tablicy -> 9
5 element tablicy -> 11
6 element tablicy -> 13
\end{verbatim}
\end{scriptsize}

Tablice mogą być inicjowane w momencie deklaracji, na przykład:

\begin{scriptsize}
\begin{verbatim}
int[] tab = {1, 2, 3, 4, 5, 6};
\end{verbatim}
\end{scriptsize}

lub równoważnie

\begin{scriptsize}
\begin{verbatim}
int[] tab = new int[]{1, 2, 3, 4, 5, 6};
\end{verbatim}
\end{scriptsize}

Inaczej niż w przypadku prostych imperatywnych języków programowania, tablice w C\# są w każdej chwili
swojego istnienia świadome swoich atrybutów. Oznacza, to że programista może na przykład w każdej
chwili dowiedzieć się jaki jest rozmiar tablicy:

\begin{scriptsize}
\begin{verbatim}
/* Wiktor Zychla, 2003 */
using System;

namespace Example
{
  public class CExample 
  {
    public static void Main(string[] args)
    {
       int[] tab = new int[]{1, 2, 3, 4, 5, 6};

       Console.WriteLine( tab.Length.ToString() );
    }
  }
}

C:\Example>example.exe
6
\end{verbatim}
\end{scriptsize}

\subsubsection{Tablice referencji}

O ile w przypadku tablic, przechowujących obiekty o typach prostych, dostęp do elementów tablicy
możliwy jest natychmiast po przydzieleniu pamięci dla tablicy, o tyle w przypadku typów 
referencyjnych programista może być w pierwszej chwili zaskoczony:

\begin{scriptsize}
\begin{verbatim}
/* Wiktor Zychla, 2003 */
using System;

namespace Example
{
  public class CObiekt
  {
    public int dana;
    public CObiekt() {}
  }

  public class CExample 
  {
    public static void Main(string[] args)
    {
      const int IL  = 5;
      CObiekt[] tab = new CObiekt[IL];
   
      tab[0].dana   = 1;           
    }
  }
}

C:\Example>example

Unhandled Exception: System.NullReferenceException: Object reference not set to
an instance of an object.
   at Example.CExample.Main(String[] args)
\end{verbatim}
\end{scriptsize}

Dlaczego próba odwołania do elementu, zainicjowanej przecież tablicy, kończy się niepowodzeniem?
Otóz dzieje się tak dlatego, że w przypadku tablic przechowujących obiekty typów referencyjnych, 
zainicjowanie tablicy:

\begin{scriptsize}
\begin{verbatim}
...
const int IL  = 5;
CObiekt[] tab = new CObiekt[IL];
\end{verbatim}
\end{scriptsize}

spowoduje utworzenie kontenera zawierającego 5 niezainicjowanych referencji. Aby odwoływać się do
elementów tablicy, należy więc oprócz zainicjowania samej tablicy, zainicjować jej elementy:

\begin{scriptsize}
\begin{verbatim}
...
const int IL  = 5;
CObiekt[] tab = new CObiekt[IL];

for ( int i=0; i<IL; i++ )
  tab[i] = new CObiekt();
...
\end{verbatim}
\end{scriptsize}

\subsubsection{Tablice wielowymiarowe}

Tablice wielowymiarowe deklaruje się w C\# równie łatwo jak jednowymiarowe, zaś ich 
obsługa również nie nastręcza żadnych trudności. W każdej chwili programista może dowiedzieć się
jakie są wymiary takiej tablicy:

\begin{scriptsize}
\begin{verbatim}
/* Wiktor Zychla, 2003 */
using System;

namespace Example
{
  public class CExample 
  {
    public static void Main(string[] args)
    {
       int[,,,] tab = new int[3,5,2,6];

       tab[0, 2, 1, 4] = 3;

       Console.WriteLine( "Tablica {0}-wymiarowa.", tab.Rank );
       for ( int i=0; i<tab.Rank; i++ )
         Console.WriteLine( "Dlugosc w {0}-tym wymiarze : {1}", 
	                      i, tab.GetLength(i) );
    }
  }
}

C:\Example>example
Tablica 4-wymiarowa.
Dlugosc w 0-tym wymiarze : 3
Dlugosc w 1-tym wymiarze : 5
Dlugosc w 2-tym wymiarze : 2
Dlugosc w 3-tym wymiarze : 6
\end{verbatim}
\end{scriptsize}

Pewnym wariantem tablic wielowymiarowych są tzw. {\em tablice postrzępione} (ang. {\em jagged arrays}).

\begin{scriptsize}
\begin{verbatim}
/* Wiktor Zychla, 2003 */
using System;

namespace Example
{
  public class CExample 
  {
    public static void Main(string[] args)
    {
       int[][] tab = new int[4][];

       tab[0] = new int[6];
       tab[1] = new int[2];
       tab[2] = new int[3];
       tab[3] = new int[5];

       tab[2][2] = 5;       
    }
  }
}
\end{verbatim}
\end{scriptsize}

Aby zrozumieć różnicę między zwykłymi tablicami wielowymiarowymi, a tablicami postrzępionymi, wyobraźmy
sobie 2-wymiarową tablicę zadeklarowaną w następujący sposób:

\begin{scriptsize}
\begin{verbatim}
int[,] tab = new int[3,3];
tab[1,1]   = 5;
\end{verbatim}
\end{scriptsize}

oraz jej postrzępionego kuzyna

\begin{scriptsize}
\begin{verbatim}
int[][] tab = new int[10][];
tab[0]      = new int[3];
tab[1]      = new int[2];
tab[2]      = new int[1];
tab[1][1]   = 5;
\end{verbatim}
\end{scriptsize}

Tablica dwuwymiarowa ma dokładnie 9 elementów ułożonych w prostokątną macierz 3 na 3 elementy. W przeciwieństwie
do niej, tablica postrzępiona przechowuje referencje do trzech tablic, z których pierwsza ma 3 elementy,
druga 2, a trzecia tylko 1. 

Zarówno w jednym jak i w drugim przypadku z punktu widzenia użytkownika są to tablice dwuwymiarowe, jednak
tablica postrzępiona może optymalniej wykorzystywać zasoby pamięci, definując w razie potrzeby krótsze lub 
dłuższe "podtablice". Można więc powiedzieć, że $n$-wymiarowa tablica jest po prostu macierzą $n$-wymiarową,
zaś $n$-wymiarowa tablica postrzępiona to w istocie $n$ niezależnych tablic o wymiarze $n-1$, z których
każda może mieć inne rozmiary.

\subsubsection{{\bf ArrayList}}

Zwykłe tablice zdecydowanie nie rozwiązują problemu kontenerów, bowiem tablice mają bardzo poważną wadę.
Otóż ilość elementów tablicy musi być znana w momencie inicjowania tablicy. Gdyby w trakcie działania programu
ilość danych uległa zmianie, programista stanąłby przed zadaniem mozolnego przekopiowania tablicy do innej,
najprawdopodobniej większej tablicy.

Aby pokonać tę niedogodność należy skorzystać z kolekcji, z których najprostszą jest {\em ArrayList}.
W przeciwieństwie do na przykład kolekcji z STL w C++, ArrayList i pozostałe kolekcje .NET korzystają
z jednorodnego interfejsu, traktującego wszystko to co wrzucono do kontenera jako {\em object}.

\begin{scriptsize}
\begin{verbatim}
/* Wiktor Zychla, 2003 */
using System;
using System.Collections;

namespace Example
{
  public class CExample 
  {
    static void InfoOKolekcji( ArrayList a )
    {
      Console.WriteLine( "Kolekcja ma {0} elementow: ", a.Count );
      foreach ( object o in a )
        Console.WriteLine( "{0} : {1}", o.GetType(), o );
    }
    public static void Main(string[] args)
    {
      ArrayList a = new ArrayList();
      
      a.Add( 5 );
      a.Add( 7 );
      a.Add( 9 );
      a.Add( true );
      a.Add( "ala ma kota" );
      
      InfoOKolekcji( a );
    }
  }
}

C:\Example>example.exe
Kolekcja ma 5 elementow:
System.Int32 : 5
System.Int32 : 7
System.Int32 : 9
System.Boolean : True
System.String : ala ma kota
\end{verbatim}
\end{scriptsize}

Oczywiście sytuacja, w której w jednym kontenerze znajdują się obiekty różnych typów jest dość rzadka.
Najczęściej programista używa kontenera jak zwykłej tablicy, o której rozmiary nie chce się martwić. 
Wtedy iteracja przez kolejne elementy może wręcz wymuszać typ elementu

\begin{scriptsize}
\begin{verbatim}
...
foreach ( int i in a )
  ...
\end{verbatim}
\end{scriptsize}

co oczywiście spowoduje wyrzucenie wyjątku, jeśli przypadkiem któryś element kontenera nie jest 
takiego typu jakiego spodziewa się programista. W przypadku wątpliwości zawsze można rzutować
dynamicznie

\begin{scriptsize}
\begin{verbatim}
...
foreach ( object o in a )
  if ( o is int )
    ...
\end{verbatim}
\end{scriptsize}

\subsubsection{Kolekcje silnie otypowane}

Programiści, którzy przychodzący ze świata C++, gdzie korzystali z kontenerów z biblioteki STL, niejednokrotnie
zgłaszają pod adresem kontenerów C\#-owych jedno zastrzeżenie. "Otóż" - jak mawiają - "możliwość umieszczania
w kontenerze obiektów dowolnego typu, oznacza podatność takich kontenerów na przypadkowe błędy". 

Rzeczywiście, jeśli z jakichś powodów programista spodziewa się w kontenerze obiektów typu {\tt int}, z związku
z czym napisze gdzieś w kodzie

\begin{scriptsize}
\begin{verbatim}
...
foreach ( int i in a )
  ...
\end{verbatim}
\end{scriptsize}

to może skończyć się to wyrzuceniem wyjątku, w przypadku omyłkowego umieszczenia w kontenerze obiektu innego
typu. Być może nawet obiekt taki umieszczany jest w kontenerze statycznie:

\begin{scriptsize}
\begin{verbatim}
...
a.Add( "Ala ma kota" );
  ...
\end{verbatim}
\end{scriptsize}

a kompilator mimo to nie zgłasza żadych zastrzeżeń. 

Dzieje się tak dlatego, że jak już powiedziano, kolekcje przechowują referencje do obiektów promując je
wcześniej do typu {\em object} i dopiero na wyraźne życzenie programista może dowiedzieć się jaki
jest prawdziwy typ obiektu przechowywanego w kontenerze. Kiedy programista korzysta z C++ kontenera
{\tt vector<T>}, kompilator jest w stanie {\em statycznie} wychwycić tego rodzaju błąd.

Cóż, z perspektywy programisty kolekcje STL mają zdecydowanie poważniejsze wady (wynikające z tego, że 
zdefiniowane są w postaci szablonów), których nie można w żaden sposób obejść. Okazuje się za to, że
w C\# przez utworzenie własnej klasy dziedziczącej z {\bf CollectionBase} można
zdefiniować kontenery otypowane statycznie. 

\begin{scriptsize}
\begin{verbatim}
/* Wiktor Zychla, 2003 */
using System;
using System.Collections;

public class IntArrayList : System.Collections.CollectionBase
{
  public virtual void Add( int i )
  {
    InnerList.Add( i );
  }
  public int this[int index]
  {
    get { return (int)InnerList[index]; }
  }
}

namespace Example
{
  public class CExample 
  {
    public static void Main(string[] args)
    {
      IntArrayList a = new IntArrayList();
      a.Add( 4 ); 
      a.Add( 7 );
      a.Add( 11 );
      a.Add( "Ala ma kota" );

      Console.WriteLine( a[2] ); 
    }
  }
}

C:\Example>csc.exe example.cs
Microsoft (R) Visual C# .NET Compiler version 7.00.9466
for Microsoft (R) .NET Framework version 1.0.3705
Copyright (C) Microsoft Corporation 2001. All rights reserved.

example.cs(27,7): error CS1502: The best overloaded method match for
        'IntArrayList.Add(int)' has some invalid arguments
example.cs(27,14): error CS1503: Argument '1': cannot convert from 'string' to
        'int'
\end{verbatim}
\end{scriptsize}

\subsubsection{Stos, kolejka}

Wbudowane w {\em System.Collections} kontenery {\bf Stack} i {\bf Queue} zachowują się dokładnie tak,
jak należałoby tego oczekiwać. Oprócz "zwykłych" operacji wstawiania i zdejmowania elementów, 
zarówno kolejka jak i stos umożliwiają "podejrzenie" aktualnie dostępnej wartości.

\begin{scriptsize}
\begin{verbatim}
/* Wiktor Zychla, 2003 */
using System;
using System.Collections;

namespace Example
{
  public class CExample 
  {
    public static void Main(string[] args)
    {
      Stack s = new Stack();

      s.Push( 4 );
      s.Push( "Ala ma kota" );
      s.Push( 3 );
      s.Pop();

      Console.WriteLine( s.Peek() );

      Queue q = new Queue();
      q.Enqueue( 4 );
      q.Enqueue( 5 );
      Console.WriteLine( q.Peek() );

      q.Dequeue();      
      Console.WriteLine( q.Peek() );
    }
  }
}
\end{verbatim}
\end{scriptsize}

\subsubsection{{\bf Hashtable}}

{\bf Hashtable} jest {\em kolekcją asocjacyjną}, to znaczy że pamięta pary w postaci klucz $\rightarrow$ wartość.
Dzięki wewnętrznej strukturze, czas dostępu do wartości skojarzonej z kluczem jest bardzo szybki i nie
zależy od ilości elementów w kolekcji. 

Hashtable wykorzystuje się na przykład do pamiętania odwzorowań częściowych (par $x$ $\rightarrow$ $f(x)$) lub
fragmentów tabel bazodanowych (par $ID$ $\rightarrow$ {\em rekord z tabeli}).

W przeciwieństwie do innych kontenerów, element Hashtable'a jest więc parą typu {\tt DictionaryEntry}. 
Programista musi o tym pamiętać podczas przeglądania kolekcji.

\begin{scriptsize}
\begin{verbatim}
/* Wiktor Zychla, 2003 */
using System;
using System.Collections;

namespace Example
{
  public class CExample 
  {
    public static void Main(string[] args)
    {
      Hashtable h = new Hashtable();
      h.Add( 5,  "Ala ma kota" );
      h.Add( 3,  "Kot ma Ale" );
      h.Add( 18, "Ktos ma cos" );

      foreach ( DictionaryEntry de in h )
        Console.WriteLine( "Para {0} - {1}", de.Key, de.Value );
    }
  }
}

C:\Example>example.exe
Para 18 - Ktos ma cos
Para 5 - Ala ma kota
Para 3 - Kot ma Ale
\end{verbatim}
\end{scriptsize}

Innym sposobem przeglądania Hashtable'a jest przeglądanie kolekcji kluczy oraz kolekcji wartości niezależnie.

\begin{scriptsize}
\begin{verbatim}
/* Wiktor Zychla, 2003 */
using System;
using System.Collections;

namespace Example
{
  public class CExample 
  {
    public static void Main(string[] args)
    {
      Hashtable h = new Hashtable();
      h.Add( 5,  "Ala ma kota" );
      h.Add( 3,  "Kot ma Ale" );
      h.Add( 18, "Ktos ma cos" );

      // przegladaj wartosci
      foreach ( string s in h.Values )
        Console.WriteLine( "Wartosc {0}", s );

      // przegladaj klucze, skojrz wartosci
      foreach ( int key in h.Keys )
        Console.WriteLine( "Para {0} - {1}", key, h[key] );
    }
  }
}

C:\Example>example.exe
Wartosc Ktos ma cos
Wartosc Ala ma kota
Wartosc Kot ma Ale
Para 18 - Ktos ma cos
Para 5 - Ala ma kota
Para 3 - Kot ma Ale
\end{verbatim}
\end{scriptsize}

Pewnym zaskoczeniem może być fakt, że elementy Hashtable'a są ujawniane kolejności {\em innej}, niż 
były umieszczane w kolekcji. Wyjaśnieniem tego zjawiska i sposobami 
radzenia sobie z nim zajmiemy się na stronie \pageref{sect:skladanie_enum}.

\subsubsection{Własne kolekcje i interfejs {\bf IEnumerable}}

Istnienie wbudowanych kontenerów nie oznacza, że każdy kolejny tworzony kontener musi dziedziczyć
z któregoś już zdefiniowanego. Inwencja programistów jest w końcu nieograniczona i wewnętrzna
reprezentacja jakiegoś kontenera zdefiniowanego przez uzytkownika może być mocno odległa od
typowej. 

To czego potrzeba, aby kontener spełniał swoje zadanie, to umożliwienie klientowi korzystającemu z niego
jakiegoś ogólnego mechanizmu przeglądania elementów, tak aby klient nie musiał być świadomy
wewnętrznej reprezentacji danych w kontenerze.

Taką możliwość daje para interfejsów {\bf IEnumerable} oraz {\bf IEnumerator}.

{\bf IEnumerator} ma 3 elementy:

\begin{description}
\item [bool MoveNext()] 
Metoda {\bf MoveNext} służy klientowi do poinformowania interfejsu o tym, że klient chce
obejrzeć kolejny element kontenera. Metoda ta powinna zwrócić wartość {\em true} jeśli po obejrzeniu
kolejnego elementu klient może kontytuować przeglądanie oraz {\em false} w przeciwnym przypadku

\item [object Current]
Propercja {\bf Current} powinna ujawniać bieżący element kontenera. 

\item [void Reset()]
Metoda {\bf Reset} powinna umożliwić klientowi przywrócenie stanu wyjściowego przeglądania elementów,
czyli najczęściej ustawienie bieżącego elementu jako pierwszego elementu kontenera.
\end{description}

{\bf IEnumerable} ma tylko 1 element:

\begin{description}
\item [IEnumerator GetEnumerator()] 
Metoda {\bf GetEnumerator} służy do pobrania instancji obiektu pozwalającego przeglądać zawartość
kolekcji. 
\end{description}

\begin{scriptsize}
\begin{verbatim}
/* Wiktor Zychla, 2003 */
using System;
using System.Collections;

namespace Example
{
  public class MyCol : IEnumerable
  {
     public class MyColEnumerator : IEnumerator
     {
        int index;
        MyCol myCol = null;

        public bool MoveNext()
        {
          index++;
          if ( index >= IL ) 
            return false;
          else
            return true;
        }     

        public object Current
        {
          get { return myCol.t[index]; }
        }
 
        public void Reset()
        {
          index = -1;
        }

        public MyColEnumerator( MyCol myCol )
        {
           this.myCol = myCol;
           Reset();
        }        
     } 

     const int IL = 3;
     int[] t  = new int[IL];

     public IEnumerator GetEnumerator()
     {
       return new MyColEnumerator( this );
     }     

     public MyCol() 
     {
       for ( int i=0; i<IL; i++ ) t[i] = 2*i;
     }
  }


  public class CExample 
  {
    public static void Main(string[] args)
    {
       MyCol myCol = new MyCol();

       IEnumerator ie = myCol.GetEnumerator();
       while ( ie.MoveNext() )
         Console.WriteLine( ie.Current.ToString() );
    }
  }
}

C:\Example>example.exe
0
2
4
\end{verbatim}
\end{scriptsize}

Zaimplementowanie interfejsu {\bf IEnumerable} umożliwia także klientom kontenera na przeglądanie go
za pomocą {\bf foreach}. {\bf Foreach} jest {\bf cukierkiem syntaktycznym}\footnote{Curkierek
syntaktyczny to sympatyczne tłumaczenie angielskiego terminu {\em syntactic sugar}. Termin ten
oznacza taki element składni języka, który z jednej strony nie wnosi niczego nowego do możliwości samego języka, 
z drugiej zaś strony upraszcza kod, bądź czyni go przejrzystszym. Typowym przykładem cukierka syntaktycznego
jest pętla {\bf for} w rodzinie języków C-podobnych. Język nie straciłby nic, gdyby wyeliminować z niego
konstrukcję {\bf for (;;)}, bowiem to samo można zawsze wyrazić przy pomocy {\bf while}. Pętla {\bf for}
jest jednak czytelniejsza.}, który w trakcie kompilacji jest tłumaczony do postaci takiej, jak 
w powyższym przykładzie. 

\begin{scriptsize}
\begin{verbatim}
...
MyCol myCol = new MyCol();

foreach ( int i in myCol )
  Console.WriteLine( i.ToString() );
...
\end{verbatim}
\end{scriptsize}

\subsubsection{Sortowanie kolekcji}

Framework pozwala rozwiązać problem sortowania w dość elegancki sposób. W przypadku typów prostych
kryteria sortowania są już ustalone, zaś programista musi jedynie skorzystać z odpowiednich sposobów ich
użycia. W przypadku własnych typów programista może określić różne porządki sortowania przez użycie
któregoś z interfejsów: {\bf IComparer} lub {\bf IComparable}.

Zacznijmy od najprostrzego przykładu: sortowania tablic i kolekcji zawierających obiekty typów 
prostych. Aby osiągnąć zamierzony cel, wystarczy skorzystać ze statycznej funkcji {\tt Array.Sort} w celu
posortowania tablicy lub metody {\tt Sort} kolekcji typu {\bf ArrayList}.

\begin{scriptsize}
\begin{verbatim}
/* Wiktor Zychla, 2003 */
using System;
using System.Collections;

namespace Example
{
  public class CExample 
  {
    static void Wypisz( IEnumerable ie )
    {
       Console.Write( "{0}: [", ie.GetType() );
       foreach ( int i in ie )
         Console.Write( "{0},", i );
       Console.WriteLine( "]" );
    }

    public static void Main(string[] args)
    {
       const int IL = 10;
       Random r = new Random();

       int[]      tab = new int[IL];
       ArrayList atab = new ArrayList();		

       for ( int i=0; i<IL; i++ )
       {
         tab[i]   = r.Next()%100;
         atab.Add(r.Next()%100);
       }

       Wypisz( tab ); 
       Array.Sort( tab );
       Wypisz( tab );  

       Wypisz( atab );
       atab.Sort();
       Wypisz( atab );
    }
  }
}

C:\Example>example.exe
System.Int32[]: [94,43,42,78,52,50,88,47,73,48,]
System.Int32[]: [42,43,47,48,50,52,73,78,88,94,]
System.Collections.ArrayList: [29,92,23,60,77,15,99,7,46,15,]
System.Collections.ArrayList: [7,15,15,23,29,46,60,77,92,99,]
\end{verbatim}
\end{scriptsize}

Przy okazji tego przykładu zauważmy, że zarówno tablice jak i kolekcje implementują interfejs {\em IEnumerable},
zwracający domyślny enumerator do przeglądania elementów w kontenerze. Skorzystaliśmy z tego sprytnie
przekazując do funkcji {\tt Wypisz} obiekt typu {\em IEnumerable}, dzięki czemu jedna i ta sama
funkcja służy do przeglądania elementów tablicy i kolekcji. 

Powyższy przykład oczywiście nie rozwiązuje problemu, bowiem w przypadku typów użytkownika funkcje sortujące
nie miałyby żadnych podstaw do określenia porządku sortowania. 

W najprostrzym scenariuszu programista we własnej klasie implementuje interfejs {\em IComparable}, dzięki
któremu instancja obiektu wie jak porównać się z inną instancją obiektu. 

Załóżmy, że w klasie {\tt COsoba} mamy pola przechowujące imię i nazwisko i chcemy skonstruować porządek, który
w pierwszej kolejności porównywałby nazwisko, zaś w przypadku równych nazwisk porównywałby imiona.

\begin{scriptsize}
\begin{verbatim}
/* Wiktor Zychla, 2003 */
using System;
using System.Collections;

namespace Example
{
  public class COsoba : IComparable
  {
    public string imie;
    public string nazwisko;

    public int CompareTo( object o )
    {
      if ( o is COsoba )
      {
        COsoba o2 = o as COsoba;
        
        if ( this.nazwisko == o2.nazwisko )
          return this.imie.CompareTo( o2.imie );
        else
          return this.nazwisko.CompareTo( o2.nazwisko );
      }
      else
        return -1;
    }

    public override string ToString()
    {
      return String.Format( "{0} {1}", nazwisko, imie );
    } 

    public COsoba( string imie, string nazwisko )
    {
      this.imie = imie; this.nazwisko = nazwisko;
    } 
  }

  public class CExample 
  {
    static void Wypisz( IEnumerable ie )
    {
       foreach ( object o in ie )
         Console.WriteLine( "{0},", o );
    }

    public static void Main(string[] args)
    {
       ArrayList atab = new ArrayList();		

       atab.Add( new COsoba( "Jan", "Kowalski" ) );
       atab.Add( new COsoba( "Zdzisław", "Kowalski" ) );
       atab.Add( new COsoba( "Jan", "Malinowski" ) );
       atab.Add( new COsoba( "Tomasz", "Abacki" ) );

       Wypisz( atab );
       atab.Sort();
       Wypisz( atab );
    }
  }
}

C:\Example>example.exe
Kowalski Jan,
Kowalski Zdzisław,
Malinowski Jan,
Abacki Tomasz,

Abacki Tomasz,
Kowalski Jan,
Kowalski Zdzisław,
Malinowski Jan,
\end{verbatim}
\end{scriptsize}

Zwróćmy uwagę w jaki sposób odbywa się ustalenie sortowania według 2 pól obiektu: otóż najpierw odbywa się
porównanie nazwisk, a następnie, w przypadku równości nazwisk, porównanie imion. Porównanie odbywa się
za pomocą tego samego mechanizmu, który jest właśnie oprogramowywany, czyli za pomocą interfejsu {\em IComparable},
tyle że tym razem metoda pochodzi z klasy {\em string}.

\begin{scriptsize}
\begin{verbatim}
if ( this.nazwisko == o2.nazwisko )
  return this.imie.CompareTo( o2.imie );
else
  return this.nazwisko.CompareTo( o2.nazwisko );
\end{verbatim}
\end{scriptsize}

W taki sam sposób można ustalać dowolne kryteria sortowania według dowolnej ilości pól. Co jednak zrobić
w przypadku, kiedy dla jednego rodzaju obiektów chciałoby się kilka różnych porządków sortowania?
Załóżmy, że w klasie COsoba dołożymy nowe pole określające wiek osoby i chcielibyśmy aby istniał 
inny porządek sortowania niż alfabetyczny - na przykład porządek chronologiczny (a może jeszcze jakieś inne)? 
Jak rozwiązać taki problem, skoro zaimplementowanie interfejsu {\em IComparable} pozwala określić tylko jeden
porządek sortowania?

Otóż aby określić więcej niż jeden porządek sortowania należy utworzyć jakąś pomocniczą klasę, która
będzie implementować interfejs {\em IComparer}. Interfejs ten ma tylko jedną metodę {\tt Compare}, która
służy do porównywania dwóch obiektów. W celu użycia wybranego intefejsu do uporządkowania obiektów
w kontenerze, należy użyć przeciążonej wersji metody {\tt Sort}, która oczekuje jako parametru właśnie
obiektu typu {\em IComparer}.

\begin{scriptsize}
\begin{verbatim}
/* Wiktor Zychla, 2003 */
using System;
using System.Collections;

namespace Example
{
  public class COsoba : IComparable
  {
    public class COsobaSortByDataUr : IComparer
    {
      public int Compare( object obj1, object obj2 )
      {
        COsoba o1 = obj1 as COsoba;
        COsoba o2 = obj2 as COsoba;
        
        return o1.dataUr.CompareTo( o2.dataUr );
      }
      public COsobaSortByDataUr() {}
    }

    public string   imie;
    public string   nazwisko;
    public DateTime dataUr;

    public int CompareTo( object o )
    {
      if ( o is COsoba )
      {
        COsoba o2 = o as COsoba;
        
        if ( this.nazwisko == o2.nazwisko )
          return this.imie.CompareTo( o2.imie );
        else
          return this.nazwisko.CompareTo( o2.nazwisko );
      }
      else
        return -1;
    }

    public override string ToString()
    {
      return String.Format( "{0} {1}, ur. {2:d}", nazwisko, imie, dataUr );
    } 

    public COsoba( string imie, string nazwisko, string dataUr )
    {
      this.imie   = imie; this.nazwisko = nazwisko; 
      this.dataUr = DateTime.Parse( dataUr );
    } 
  }

  public class CExample 
  {
    static void Wypisz( IEnumerable ie )
    {
       Console.WriteLine();
       foreach ( object o in ie )
         Console.WriteLine( "{0},", o );
    }

    public static void Main(string[] args)
    {
       ArrayList atab = new ArrayList();		

       atab.Add( new COsoba( "Jan",      "Kowalski", "1994-03-01" ) );
       atab.Add( new COsoba( "Zdzisław", "Kowalski", "1992-11-29" ) );
       atab.Add( new COsoba( "Jan",    "Malinowski", "1990-02-16" ) );
       atab.Add( new COsoba( "Tomasz",   "Abacki"  , "1991-01-12" ) );

       Wypisz( atab );
       atab.Sort();
       Wypisz( atab );
       atab.Sort( new COsoba.COsobaSortByDataUr() );
       Wypisz( atab );

    }
  }
}

C:\Example>example.exe

Kowalski Jan, ur. 1994-03-01,
Kowalski Zdzisław, ur. 1992-11-29,
Malinowski Jan, ur. 1990-02-16,
Abacki Tomasz, ur. 1991-01-12,

Abacki Tomasz, ur. 1991-01-12,
Kowalski Jan, ur. 1994-03-01,
Kowalski Zdzisław, ur. 1992-11-29,
Malinowski Jan, ur. 1990-02-16,

Malinowski Jan, ur. 1990-02-16,
Abacki Tomasz, ur. 1991-01-12,
Kowalski Zdzisław, ur. 1992-11-29,
Kowalski Jan, ur. 1994-03-01,
\end{verbatim}
\end{scriptsize}

Zauważmy, że tam gdzie jawnie nie podano odpowiedniego kryterium sortowania, zostanie użyte sortowanie
określone przez interfejs {\em IComparable} implementowany przez obiekt. Każde inne kryterium musi być
użyte jawnie.

W powyższym przykładzie klasa udostępniająca interfejs do sortowania została umieszczona wewnątrz
klasy głównej, aby programista korzystający z niej miał świadomość jej przeznaczenia. Mimo to sposób wywołania
sortowania nie jest zbyt elegancki:

\begin{scriptsize}
\begin{verbatim}
atab.Sort( new COsoba.COsobaSortByDataUr() );
\end{verbatim}
\end{scriptsize}

Można uczynić kod nieco przejrzystszym przez dołożenie do klasy {\tt COsoba} publicznej
statycznej propercji zwracającej odpowiedni obiekt:

\begin{scriptsize}
\begin{verbatim}
  ...
  public class COsoba : IComparable
  {
    class COsobaSortByDataUr : IComparer
    {
      public int Compare( object obj1, object obj2 )
      {
        COsoba o1 = obj1 as COsoba;
        COsoba o2 = obj2 as COsoba;
        
        return o1.dataUr.CompareTo( o2.dataUr );
      }
      public COsobaSortByDataUr() {}
    }

    public string   imie;
    public string   nazwisko;
    public DateTime dataUr;

    public static IComparer SortByDataUr
    {
       get 
       { 
         return (IComparer)(new COsobaSortByDataUr());
       }
    }
    ... 
\end{verbatim}
\end{scriptsize}

Klasa implementująca sortowanie nie musi już być publiczna, zaś sortowanie z użyciem odpowiedniego
kryterium jest już proste:

\begin{scriptsize}
\begin{verbatim}
atab.Sort( COsoba.SortByDataUr );
\end{verbatim}
\end{scriptsize}

\subsubsection{Opakowywanie enumeratorów}

Intensywne korzystanie z kolekcji znacząco wpływa na wydajność pracy programisty. Kod tworzony jest szybciej,
jest w nim mniej pomyłek i jest znacznie czytelniejszy.

Załóżmy, że aplikacja rozwija się pomyślnie i w pewnym momencie pojawiają się dodatkowe okolicznosci.
Elementy jakiegoś kontenera powinny być przeglądnięte, 
a część z nich, spełniająca jakieś kryteria, usunięta. Naiwnie
napisalibyśmy coś, co nieoczekowanie kończy się katastrofą!

\begin{scriptsize}
\begin{verbatim}
/* Wiktor Zychla, 2003 */
using System;
using System.Collections;

namespace Example
{
  public class CExample 
  {
    public static void Main(string[] args)
    {
       ArrayList atab = new ArrayList();		

       atab.Add( 5 );
       atab.Add( 10 );
       atab.Add( 3 );

       foreach ( int i in atab )
         if ( i < 4 )
           atab.Remove( i );
    }
  }
}

C:\Example>example.exe

Unhandled Exception: System.InvalidOperationException: Collection was modified;
enumeration operation may not execute.
   at System.Collections.ArrayListEnumeratorSimple.MoveNext()
   at Example.CExample.Main(String[] args)
\end{verbatim}
\end{scriptsize}

Natknęliśmy się na dość spory problem - w trakcie enumeracji nie wolno modyfikować zawartości
kontenera, bowiem enumeracja traci sens, jeśli - obrazowo mówiac - usuwa się jej grunt spod nóg.

Ten problem można rozwiązać na kilka sposobów, na przykład w czasie enumeracji tworząc 
pomocniczą listę referencji do obiektów, które należy usunąć, a potem usuwać je w kolejnej pętli, lub
korzystając z innych pętli niż {\em foreach}, gdzie istnieje możliwość wyspecyfikowania bardziej subtelnych
warunków zakończenia iteracji, uwzględniających możliwe zmiany w zawartości kontenera.

Okazuje się, że można postąpić jeszcze inaczej, definiując enumerator 
opakowujący\footnote{Autorem pomysłu jest współtwórca C\#, Eric Gunnerson, którego artykuł na ten temat
można znaleźć na stronach MSDN.}. 

Enumerator taki będzie inicjowany dowolnym obiektem, który implementuje interfejs {\em IEnumerable}, 
a następnie będzie tworzył kopię referencji do obiektów w źródłowym kontenerze. Jeżeli programista
zechce obejrzeć opakowane elementy, dostanie do ręki listę tych właśnie kopii referencji do obiektów
z oryginalnej kolekcji.

Aby skorzystać z iteratora opakowującego, zamiast

\begin{scriptsize}
\begin{verbatim}
foreach ( int i in atab )
  if ( i < 4 )
    atab.Remove( i );
\end{verbatim}
\end{scriptsize}

programista napisze

\begin{scriptsize}
\begin{verbatim}
foreach ( int i in new IterIsolate( atab ) )
  if ( i < 4 )
    atab.Remove( i );
\end{verbatim}
\end{scriptsize}

Wadą takiego rozwiązania jest konieczność tworzenia listy z duplikatami referencji do obiektów
z oryginalnej kolekcji. Zaletą - niezwykła elegancja kodu. 

\begin{scriptsize}
\begin{verbatim}
/* Wiktor Zychla, 2003. IterIsolate: Eric Gunnerson */
using System;
using System.Collections;

namespace Example
{
  public class IterIsolate: IEnumerable
  {
    internal class IterIsolateEnumerator: IEnumerator
    {
      protected ArrayList items;
      protected int currentItem;

      internal IterIsolateEnumerator(IEnumerator enumerator)
      {
        IterIsolateEnumerator chainedEnumerator = 
          enumerator as IterIsolateEnumerator;

        if (chainedEnumerator != null)
        {
          items = chainedEnumerator.items;					
        }
        else
        {
          items = new ArrayList();
          while (enumerator.MoveNext() != false)
          {
            items.Add(enumerator.Current);
          }
          IDisposable disposable = enumerator as IDisposable;
          if (disposable != null)
          {
            disposable.Dispose();
          }
        }
        currentItem = -1;
      }

      public void Reset()
      {
        currentItem = -1;
      }

      public bool MoveNext()
      {
        currentItem++;
        if (currentItem == items.Count)
          return false;

        return true;
      }

      public object Current
      {
        get
        {
          return items[currentItem];
        }
      }
    }

    public IterIsolate(IEnumerable enumerable)
    {
      this.enumerable = enumerable;
    }

    public IEnumerator GetEnumerator()
    {
      return new IterIsolateEnumerator(enumerable.GetEnumerator());
    }

    protected IEnumerable enumerable;
  }

  public class CExample 
  {
    public static void Main(string[] args)
    {
       ArrayList atab = new ArrayList();		

       atab.Add( 5 );
       atab.Add( 10 );
       atab.Add( 3 );

       foreach ( int i in new IterIsolate( atab ) )
         if ( i < 4 )
           atab.Remove( i );
    }
  }
}
\end{verbatim}
\end{scriptsize}

\label{sect:skladanie_enum}

Bardzo podobnego pomysłu można użyć do rozwiązania problemu przeglądania elementów kolekcji
{\em Hashtable} w ustalonej przez programistę kolejności. W tym celu utworzymy nowy enumerator
opakowujący, który utworzy kopie referencji i posortuje je w ustalonej kolejności.

\begin{scriptsize}
\begin{verbatim}
/* Wiktor Zychla, 2003 */
using System;
using System.Collections;

namespace Example
{
  public class IterIsolate: IEnumerable
  {
    ... jak wyżej ...
  }

  public class IterSort: IterIsolate, IEnumerable
  {
    internal class IterSortEnumerator: IterIsolateEnumerator, IEnumerator
    {
      internal IterSortEnumerator(IEnumerator enumerator, IComparer comparer): base(enumerator)
      {
        if (comparer != null)
        {
          items.Sort(comparer);
        }
        else
        {
          items.Sort();
        }
      }
    }

    public IterSort(IEnumerable enumerable): base(enumerable) {}

    public IterSort(IEnumerable enumerable, IComparer comparer): base(enumerable)
    {
      this.comparer = comparer;
    }

    public new IEnumerator GetEnumerator()
    {
      return new IterSortEnumerator(enumerable.GetEnumerator(), comparer);
    }
    IComparer comparer;
  }

  public class COsoba : IComparable
  {
    ... jak wyżej ...
  }

  public class CExample 
  {
    public static void Main(string[] args)
    {
      Hashtable hOsoby = new Hashtable();

      // w kolekcji pamiętamy pary : ID -> Osoba
      hOsoby.Add( 7,  new COsoba( "Jan",      "Kowalski", "1994-03-01" ) );
      hOsoby.Add( 10, new COsoba( "Zdzisław", "Kowalski", "1992-11-29" ) );
      hOsoby.Add( 3,  new COsoba( "Jan",    "Malinowski", "1990-02-16" ) );
      hOsoby.Add( 17, new COsoba( "Tomasz",   "Abacki"  , "1991-01-12" ) );

      // przeglądaj kolekcję
      foreach ( COsoba o in hOsoby.Values )
        Console.WriteLine( o.ToString() );

      Console.WriteLine();
       
      // przeglądaj posortowaną kolekcję
      foreach ( COsoba o in new IterSort( hOsoby.Values, COsoba.SortByDataUr ) )
        Console.WriteLine( o.ToString() );
    }
  }
}

C:\Example>example.exe
Kowalski Zdzisław, ur. 1992-11-29
Kowalski Jan, ur. 1994-03-01
Abacki Tomasz, ur. 1991-01-12
Malinowski Jan, ur. 1990-02-16

Malinowski Jan, ur. 1990-02-16
Abacki Tomasz, ur. 1991-01-12
Kowalski Zdzisław, ur. 1992-11-29
Kowalski Jan, ur. 1994-03-01
\end{verbatim}
\end{scriptsize}

W podoby sposób można utworzyć inne enumeratory opakowujące, na przykład {\tt IterSelect}, który jako
parametr przyjąłby predykat przyjmujący jako parametr obiekt z kolekcji i zwracający wartości 
{\em true} lub {\em false}. Podczas przeglądania kolekcji taki enumerator udostępniałby tylko
te obiekty z kolekcji, dla których wartość predykatu byłaby równa {\em true}. 

\begin{scriptsize}
\begin{verbatim}
public delegate bool IterSelectDelegate(object o);
	
public class IterSelect: IterIsolate, IEnumerable
{
  internal class IterSelectEnumerator: IterIsolateEnumerator, IEnumerator
  {
    internal IterSelectEnumerator(IEnumerator enumerator, 
                             IterSelectDelegate selector): base(enumerator)
    {
      for (int index = items.Count - 1; index >= 0; index--)
      {
        if (!selector(items[index]))
          items.RemoveAt(index);
      }

      currentItem = items.Count;
    }

    public new void Reset()
    {
      currentItem = items.Count;
    }

    public new bool MoveNext()
    {
      currentItem--;
      if (currentItem < 0)
        return false;

      return true;
    }
  }

  public IterSelect(IEnumerable enumerable, IterSelectDelegate selector): base(enumerable)
  {
    this.selector = selector;
  }

  public new IEnumerator GetEnumerator()
  {
    return new IterSelectEnumerator(enumerable.GetEnumerator(), selector);
  }

  IterSelectDelegate selector;
}
\end{verbatim}
\end{scriptsize}

Dzięki możliwości składania takich enumeratorów, programista mógłby więc napisać:

\begin{scriptsize}
\begin{verbatim}
foreach ( COsoba o in 
  new IterSelect( 
     new IterSort( hOsoby, COsoba.SortByDataUr ), 
        COsoba.RokUrodzenia( 1990 ) ) )
{
   // posortowane obiekty COsoba z kolekcji hOsoby
   // ale tylko te urodzone w 1990 roku
}
\end{verbatim}
\end{scriptsize}

