\subsection{Dynamiczne tworzenie kodu}

Jedną z najciekawszych możliwości biblioteki .NET jest dynamiczne tworzenie kodu w czasie działania
aplikacji. Programista może zażyczyć sobie utworzenia instancji obiektu kompilatora, skompilować
fragment kodu na dysk lub do pamięci a nawet dynamicznie dołączyć taki kod do swojej aplikacji.

Najpierw zobaczmy w jaki sposób utworzyć dynamicznie obiekt kompilatora, skompilować kod
do postaci wykonywalnej, a następnie uruchomić skompilowany kod jako nowy proces w systemie.
Chcielibyśmy ponadto, aby tak utworzony obiekt kompilatora przechwytywał i raportował błędy kompilacji.

\begin{scriptsize}
\begin{verbatim}
/* Wiktor Zychla, 2003 */
using System;
using System.Diagnostics;
using System.IO;
using System.CodeDom;
using System.CodeDom.Compiler;

using Microsoft.CSharp;

namespace Example
{
  public class CExample 
  {
    public static void Main(string[] args)
    {
      string sFileName;
      string sOutFileName;

      Console.Write( "Podaj nazwe pliku do skompilowania: " );
      sFileName    = Console.ReadLine();
      sOutFileName = 
        Path.GetFileNameWithoutExtension( sFileName ) + ".exe";

      if ( File.Exists( sFileName ) )
      {
        CSharpCodeProvider codeProvider = new CSharpCodeProvider();
        ICodeCompiler icc = codeProvider.CreateCompiler();

        CompilerParameters parameters = new CompilerParameters();
        parameters.GenerateExecutable = true;
        parameters.OutputAssembly     = sOutFileName;
     
        CompilerResults results = icc.CompileAssemblyFromFile( parameters, sFileName );
      
        if (results.Errors.Count > 0)
        {
          foreach(CompilerError CompErr in results.Errors)
          {
            Console.WriteLine( "Linia: " + CompErr.Line + 
                               ", Numer: " + CompErr.ErrorNumber );
            Console.WriteLine( CompErr.ErrorText );
          }
        }
        else
          Process.Start( sOutFileName );            
      }
    }
  }
}

C:\Example>example.exe
Podaj nazwe pliku do skompilowania: test.cs
Linia: 5, Numer: CS1514
{ expected
\end{verbatim}
\end{scriptsize}

Dynamiczne kompilowanie kodu w sposób pokazany w powyższym przykładzie ma jednak kilka wad:
\begin{itemize}
\item podczas kompilacji tworzony jest plik wykonywalny ze skompilowanym kodem
\item kompilowany kod musi być w pełni samodzielny, w szczególności musi zawierać funkcję {\tt Main}
\item skompilowany proces podczas uruchamiania tworzy nowe okno konsoli
\end{itemize}

Aby poradzić sobie z tymi problemami, po pierwsze zażyczymy sobie tworzenia kodu do pamięci zamiast
na dysk. Po drugie, skorzystamy z mechanizmu refleksji, dzięki któremu będziemy mogli obejrzeć
składniki skompilowanego do pamięci kodu. Po trzecie, wykorzystamy mechanizm pozwalający na tworzenie
delegatów z obiektów typu {\tt MethodInfo}, dzięki czemu będziemy mogli wybrać z kompilowanego kodu 
tylko te metody, które są interesujące.

Przygotujmy najpierw testowy plik z przykładowymi funkcjami:

\begin{scriptsize}
\begin{verbatim}
/* 
   test.cs 
   plik z przykładowymi funkcjami, który będzie dynamicznie kompilowany 
*/
using System;

namespace NSpace
{
  public class CMain
  {
    public static int A( int n )
    {
      return n+n; 
    }

    public static int B( int n )
    {
      return n*n; 
    }
  }
}
\end{verbatim}
\end{scriptsize}

A oto zmodyfikowany przykład dynamicznego tworzenia kodu:

\begin{scriptsize}
\begin{verbatim}
/* Wiktor Zychla, 2003 */
using System;
using System.Diagnostics;
using System.IO;
using System.CodeDom;
using System.CodeDom.Compiler;
using System.Reflection;

using Microsoft.CSharp;

namespace Example
{
  public class CExample 
  {
    public delegate int DF( int n );
    public static DF DummyDF = new DF( FDummy );
    public static int FDummy( int n )
    {
      return 0;
    }

    public static void Main(string[] args)
    {
      string sFileName;
      string sOutFileName;

      Console.Write( "Podaj nazwe pliku do skompilowania: " );
      sFileName    = Console.ReadLine();
      sOutFileName = 
        Path.GetFileNameWithoutExtension( sFileName ) + ".exe";

      if ( File.Exists( sFileName ) )
      {
        CSharpCodeProvider codeProvider = new CSharpCodeProvider();
        ICodeCompiler icc = codeProvider.CreateCompiler();

        CompilerParameters parameters = new CompilerParameters();
        parameters.GenerateExecutable = false;
        parameters.OutputAssembly     = sOutFileName;
     
        CompilerResults results = 
          icc.CompileAssemblyFromFile( parameters, sFileName );
      
        if (results.Errors.Count > 0)
        {
          foreach(CompilerError CompErr in results.Errors)
          {
            Console.WriteLine( "Linia: " + CompErr.Line + 
                               ", Numer: " + CompErr.ErrorNumber );
            Console.WriteLine( CompErr.ErrorText );
          }
        }
        else
        {
          try
          {
            Assembly  assembly = results.CompiledAssembly;

            Console.Write( "Podaj nazwę typu: " );
            Type t = assembly.GetType( Console.ReadLine() );
          
            Console.Write( "Podaj nazwę funkcji o prototypie int F(int): " );
            MethodInfo me = t.GetMethod( Console.ReadLine() );

            DF df = (DF)DF.CreateDelegate( DummyDF.GetType(), me  );
            Console.Write( "Podaj wartość parametru (int): " );

            int result = df( int.Parse( Console.ReadLine() ) ); 
            Console.WriteLine( result ); 
           }
           catch ( Exception ex )
           {
             Console.WriteLine( ex.Message );
           } 
        }
      }
    }
  }
}

C:\Example>example.exe
Podaj nazwe pliku do skompilowania: test.cs
Podaj nazwę typu: NSpace.CMain
Podaj nazwę funkcji o prototypie int F(int): A
Podaj wartość parametru (int): 24
48

C:\Example>example.exe
Podaj nazwe pliku do skompilowania: test.cs
Podaj nazwę typu: NSpace.CMain
Podaj nazwę funkcji o prototypie int F(int): B
Podaj wartość parametru (int): 25
625
\end{verbatim}
\end{scriptsize}

Cała siła tego kodu opiera się na linii

\begin{scriptsize}
\begin{verbatim}
DF df = (DF)DF.CreateDelegate( DummyDF.GetType(), me );
\end{verbatim}
\end{scriptsize}

Tworzony jest tutaj delegat typu {\em DF} za pomocą statycznej funkcji {\em CreateDelegate}, która w tej
(jednej z 4) wersji spodziewa się parametru określającego typ tworzonego delegata (tu: domyślnego
delegata typu {\tt DF} utworzonego w kodzie) oraz informacji o metodzie pobranej przez mechanizm refleksji.

Bardzo prosto napisać funkcję, która będzie mogła ewaluować wyrażenie dowolnego typu:

\begin{scriptsize}
\begin{verbatim}
using System;
using System.CodeDom;
using System.CodeDom.Compiler;
using Microsoft.CSharp;
using System.Text;
using System.Reflection;

namespace Example
{
  public class Evaluator
  {
    public static object Evaluate( Type type, string expression )
    {
      ICodeCompiler comp = (new CSharpCodeProvider().CreateCompiler());
      CompilerParameters cp = new CompilerParameters();
      cp.ReferencedAssemblies.Add("system.dll");
      cp.ReferencedAssemblies.Add("system.data.dll");
      cp.ReferencedAssemblies.Add("system.xml.dll");
      cp.GenerateExecutable = false;
      cp.GenerateInMemory = true;

      StringBuilder code = new StringBuilder();
      code.Append("using System; \n");
      code.Append("using System.Data; \n");
      code.Append("using System.Data.SqlClient; \n");
      code.Append("using System.Data.OleDb; \n");
      code.Append("using System.Xml; \n");
      code.Append("namespace _Evaluator { \n");
      code.Append("  public class _Evaluator { \n");
      code.AppendFormat("    public {0} Foo() ", type.Name );
      code.Append("{ ");
      code.AppendFormat("      return ({0}); ", expression);
      code.Append("}\n");
      code.Append("} }");

      CompilerResults cr = 
        comp.CompileAssemblyFromSource(cp, code.ToString());

      if (cr.Errors.HasErrors)
      {
        StringBuilder error = new StringBuilder();
        error.Append("Error Compiling Expression: ");
        foreach (CompilerError err in cr.Errors)
        {
          error.AppendFormat("{0}\n", err.ErrorText);
        }
        throw new Exception("Error Compiling Expression: " + 
	                     error.ToString());
      }
      Assembly    a = cr.CompiledAssembly;
      object      c = a.CreateInstance("_Evaluator._Evaluator");
      MethodInfo mi = c.GetType().GetMethod("Foo");
      return mi.Invoke( c, null );
    }
  }

  public class CMain
  {
    public static void Main()
    {
      Console.Write( "Wpisz wyrażenie arytmetyczne: " );
      Console.WriteLine( (int)Evaluator.Evaluate( typeof(int), 
                         Console.ReadLine() ) );
    }
  }
}

C:\Example>example
Wpisz wyrażenie arytmetyczne: (8*(4+6))-12
68

C:\Example>example
Wpisz wyrażenie arytmetyczne: 5+(

Unhandled Exception: System.Exception: Error Compiling Expression: Error Compili
ng Expression: Invalid expression term ')'
) expected

   at Example.Evaluator.Evaluate(Type type, String expression)
   at Example.CMain.Main()
\end{verbatim}
\end{scriptsize}

Zastosowania takiej metody tworzenia kodu mogą być bardzo szerokie. Można na przykład wyposażyć aplikację
w moduł skryptowy, który pozwoli użytkownikowi podkładać własne funkcje w miejsce dostarczanych z aplikacją.
Można zaprojektować całkowicie własny język skryptowy, dopisać prosty kompilator między
tym językiem a C\#, następnie kompilować kod w języku skryptowym najpierw do C\#, a
kod C\# dynamicznie dołączać do własnej aplikacji w czasie jej działania.

Można również wyobrazić sobie, że pewne ważne fragmenty aplikacji dostarczone są w postaci zaszyfrowanego kodu
źródłowego, do którego kod odszyfrowujący zna tylko użytkownik. Kod taki mógłby być odszyfrowywany i kompilowany
w czasie działania aplikacji, zaś użytkownik miałby pewność, że w przypadku kradzieży aplikacja byłaby
dla ewentualnego złodzieja bezużyteczna, gdyby nie znał hasła odszyfrowującego.

\subsection{Procesy, wątki}

\subsubsection{Procesy}

Dzięki bibliotece {\em System.Diagnostics} programista może kontrolować procesy działające w systemie.

\begin{scriptsize}
\begin{verbatim}
using System;
using System.Diagnostics;

namespace Example
{
  public class CMain
  {
    public static void Main()
    {
      Process[] pt = Process.GetProcesses();
      foreach ( Process p in pt )
      {
        Console.WriteLine( p.ProcessName.ToString() ); 
        Console.WriteLine( "\t"+p.PriorityClass.ToString() );
        Console.WriteLine( "\t"+p.MainModule.ModuleName );
        Console.WriteLine( "\t"+p.MainModule.ModuleMemorySize );
      }
    }
  }
}
\end{verbatim}
\end{scriptsize}

Procesy można nie tylko przeglądać, ale również tworzyć, zabijać, czekać na ich zakończenie oraz
korzystać z możliwości powłoki. Poniższy przykład jest C\#-owym odpowiednikiem przykładu ze strony 
\pageref{powlokaSystemu}.

\begin{scriptsize}
\begin{verbatim}
using System;
using System.Diagnostics;

namespace Example
{
  public class CMain
  {
    public static void Main()
    {
      Process p = new Process();

      p.StartInfo.UseShellExecute = true;
      p.StartInfo.Verb            = "print";
      p.StartInfo.FileName        = "plik.doc";
 
      p.Start();
      p.WaitForExit();
      p.Dispose();
      
      Console.WriteLine( "zakończono drukowanie" );
    }
  }
}
\end{verbatim}
\end{scriptsize}

\subsubsection{Wątki}

Biblioteka {\em System.Threading} udostępnia funkcje do tworzenia i synchronizacji wątków. Przekazywanie 
parametrów do wątków możliwe jest dzięki opakowywaniu ich w pomocnicze klasy:

\begin{scriptsize}
\begin{verbatim}
using System;
using System.Threading;

namespace Example
{
  public class CMyThread
  {
    string nazwa;
    int    k; 

    public void ThreadFunc()
    {
      for ( int i=0; i<k; i++ )
      {
        Console.WriteLine( nazwa );
        Thread.Sleep( 1 );
      }
    }

    public CMyThread( string nazwa, int k )
    {
      this.nazwa = nazwa;
      this.k     = k;
    }
  }


  public class CMain
  {   
    public static void Main()
    {
       string[] n = { "Jurek", "Ogórek", "Kiełbasa", "Sznurek" };
       int[]    m = { 4, 5, 2, 7 };
       
       for ( int i=0; i<n.Length; i++ )
       {
          CMyThread mt = new CMyThread( n[i], m[i] );
          Thread    t  = new Thread( new ThreadStart( mt.ThreadFunc ) );
          t.Start();
       }
    }
  }
}

C:\Example>example
Ogórek
Kiełbasa
Kiełbasa
Ogórek
Ogórek
Sznurek
Jurek
Sznurek
Ogórek
Jurek
Ogórek
Sznurek
Sznurek
Sznurek
Jurek
Sznurek
Jurek
Sznurek
\end{verbatim}
\end{scriptsize}

Najprostszy wariant synchronizacji wątków możliwy jest dzięki słowu kluczowemu C\# {\tt lock}. 
Objęcie jakiejś zmiennej tą klauzulą powoduje zablokowanie dostępu do tej zmiennej pozostałym wątkom na
tak długo, aż bieżący wątek opuści blok {\tt lock}, na przykład:

\begin{scriptsize}
\begin{verbatim}
JakisObiekt o; 
...
lock ( o )
{
  ...
}
\end{verbatim}
\end{scriptsize}

Do synchronizacji wątków można również wykorzystać m.in.:
\begin{itemize}
\item {\tt AutoResetEvent}
\item {\tt ManualResetEvent}
\item {\tt Monitor}
\item {\tt Mutex}
\end{itemize}

