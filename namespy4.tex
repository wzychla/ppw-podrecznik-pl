\subsection{XML}
\label{section_XML}

Tam, gdzie zachodzi konieczność wymiany danych, tam programiści muszą ustalić jakiś sposób 
ich przekazywania. Wyobraźmy sobie scenariusz, w którym aplikacja A powinna udostępniać jakiś
zbiór danych aplikacji B.

Najbardziej naiwnym podejściem byłoby dodanie do aplikacji B modułu pozwalającego na wczytywanie
danych bezpośrednio w formie, w jakiej składuje je aplikacja A. Takie rozwiązanie nie sprawdza się
jednak wtedy, gdy format danych aplikacji A ulegnie z jakiegoś powodu zmianie (najczęściej rozszerzeniu).

Wydawałoby się więc, że rozwiązaniem byłoby zaprojektowanie jakiegoś formatu pliku pozwalającego
na przekazywanie danych między aplikacjami. Takie rozwiązanie mogłoby być niezależne od wewnętrznych
formatów danych aplikacji A i B. Gdyby jednak przekazywać dane w formie binarnej, to oczywiście w razie
chęci rozszerzenia zakresu przekazywanych danych cała zabawa zaczyna się od początku.

Dość nieoczekiwanie rozwiązaniem tego i podobnych problemów okazało się zaprojektowanie
standardu XML (ang {\em eXtended Markup Language}), czyli pewnego standardu budowania plików
tekstowych do przesyłania dowolnego typu danych. XML przyjął się głównie dzięki temu, że 
uprościł przesyłanie danych w sieci, gdzie informacja w formie tekstowej okazuje się być często
jedynym wspólnym mianownikiem dla różnego rodzaju platform biorących udział w komunikacji.

Sam w sobie XML jest rozszerzeniem idei HTML - o ile jednak HTML narzuca ściśle zbiór możliwych nazw
węzłów jest ściśle ustalony, o tyle XML pozwala ustalać je dowolnie. Tak jak pliki HTML, tak i pliki XML
mogą być budowane nawet w zwykłym edytorze tekstów. Przeglądarki internetowe potrafią najczęściej przeglądać
takie pliki (była już o tym mowa na stronie \pageref{dokumentowanieKodu}).

Biblioteka {\em System.Xml} pozwala na manipulowanie plikami XML na 3 sposoby (a właściwie 2 i pół, bo
ostatni jest tylko rozszerzeniem przedostatniego): 
\begin{itemize}
\item za pomocą obiektów {\em System.Xml.XmlTextReader} i {\em System.Xml.XmlTextWriter}
\item za pomocą obiektów DOM (klasa {\em System.Xml.XmlDocument})
\item za pomocą obiektów z {\em System.Xml.XPath} 
\end{itemize}

Pierwsze dwie możliwości pozwalają na programowe czytanie i tworzenie plików XML, zaś obiekty {\em XPath}
wspomagają tylko odczyt struktury XML.

Przygotujmy najpierw prosty plik z danymi:
\begin{scriptsize}
\begin{verbatim}
<?xml version="1.0" encoding="windows-1250"?>
<ListaOsób nazwa="znajomi">
  <Osoba>
    <Imię>Jan</Imię>
    <Nazwisko>Kowalski</Nazwisko>
    <DataUr>1950-02-07</DataUr>
  </Osoba>
  <Osoba>
    <Imię>Tomasz</Imię>
    <Nazwisko>Malinowski</Nazwisko>
    <DataUr>1976-10-04</DataUr>
  </Osoba>
  <Osoba>
    <Imię>Adam</Imię>
    <Nazwisko>Nowak</Nazwisko>
    <DataUr>1984-02-17</DataUr>
  </Osoba>
</ListaOsób>
\end{verbatim}
\end{scriptsize}

\subsubsection{{\em XmlTextReader} i {\em XmlTextWriter}}

Z perspektywy obiektów {\em XmlTextReader} i {\em XmlTextWriter} pliki XML są strumieniami danych.
Odczyt zawartości pliku XML za pomocą obiektu {\em XmlTextReader} może wyglądać następująco:

\begin{scriptsize}
\begin{verbatim}
/* Wiktor Zychla, 2003 */
using System;
using System.IO;
using System.Xml;

namespace Example
{
  public class CMain
  {   
    public static void Main()
    {
       FileStream fs = new FileStream( "osoby.xml", FileMode.Open );
       XmlTextReader xr = new XmlTextReader( fs );

       while ( xr.Read() )
       {
         XmlNodeType t = xr.NodeType;
         Console.WriteLine( "Węzeł typu {0}", t );

         if ( t == XmlNodeType.Element )
         {
           Console.WriteLine( "\t<{0}>", xr.Name );
           if ( xr.HasAttributes )
             while ( xr.MoveToNextAttribute() )
               Console.WriteLine( "\t\t[{0}]{1}", xr.Name, xr.Value );                         
         }  
         if ( t == XmlNodeType.Text )
           Console.WriteLine( "Tekst: {0}", xr.Value );         
       }

       xr.Close();
    }
  }
}

C:\Example>example.exe
Węzeł typu XmlDeclaration
Węzeł typu Whitespace
Węzeł typu Element
	<ListaOsób>
		[nazwa]znajomi
Węzeł typu Whitespace
Węzeł typu Element
	<Osoba>
Węzeł typu Whitespace
Węzeł typu Element
	<Imię>
Węzeł typu Text
Tekst: Jan
Węzeł typu EndElement
Węzeł typu Whitespace
Węzeł typu Element
	<Nazwisko>
Węzeł typu Text
Tekst: Kowalski
Węzeł typu EndElement
Węzeł typu Whitespace
Węzeł typu Element
	<DataUr>
Węzeł typu Text
Tekst: 1950-02-07
...
\end{verbatim}
\end{scriptsize}

\subsubsection{Obiekty DOM}

Obiekty DOM najpierw wczytują zawartość XMLa do pamięci, a następnie budują drzewo
składniowe, które programista może przeglądać rekursywnie zgodnie z jego strukturą:

\begin{scriptsize}
\begin{verbatim}
/* Wiktor Zychla, 2003 */
using System;
using System.IO;
using System.Xml;

namespace Example
{
  public class CMain
  {   
    public static void InfoOWezle( XmlNode node )
    {
        Console.WriteLine( "Węzeł typu {0}", node.NodeType );

         if ( node.NodeType == XmlNodeType.Element )
         {
           Console.WriteLine( "\t<{0}>", node.Name );
           foreach ( XmlAttribute a in node.Attributes )
             Console.WriteLine( "\t\t[{0}]{1}", a.Name, a.Value );                         
         }  
         if ( node.NodeType == XmlNodeType.Text )
           Console.WriteLine( "Tekst: {0}", node.Value );    

         foreach ( XmlNode child in node.ChildNodes )
           InfoOWezle( child );
    }

    public static void Main()
    {
       XmlDocument xml = new XmlDocument();
       xml.Load( "osoby.xml" );

       InfoOWezle( xml );
    }
  }
}

C:\Example>example.exe
Węzeł typu Document
Węzeł typu XmlDeclaration
Węzeł typu Element
	<ListaOsób>
		[nazwa]znajomi
Węzeł typu Element
	<Osoba>
Węzeł typu Element
	<Imię>
Węzeł typu Text
Tekst: Jan
Węzeł typu Element
	<Nazwisko>
Węzeł typu Text
Tekst: Kowalski
Węzeł typu Element
	<DataUr>
Węzeł typu Text
Tekst: 1950-02-07
...
\end{verbatim}
\end{scriptsize}

\subsubsection{XPath}

Jeszcze jeden krok dalej idą obiekty z klasy {\em XPath}, za pomocą których można łatwiej
nawigować po strukturze obiektu {\em XmlDocument}. Obiekt {\em XPathNavigator} umożliwia
wybór konkretnego węzła, zaś {\em XPathNodeIterator} pozwala na nawigację po sąsiednich węzłach.

Aby uprościć przykład, przesuńmy informacje o osobach z podwęzłów do atrybutów węzłów.

\begin{scriptsize}
\begin{verbatim}
<?xml version="1.0" encoding="windows-1250"?>
<ListaOsób nazwa="znajomi">
  <Osoba Imię="Jan" Nazwisko="Kowalski" DataUr="1950-02-07"/>
  <Osoba Imię="Adam" Nazwisko="Nowak" DataUr="1992-12-23"/>
  <Osoba Imię="Tomasz" Nazwisko="Malinowski" DataUr="1979-10-01"/>
</ListaOsób>
\end{verbatim}
\end{scriptsize}

Dzięki możliwości nawigacji po strukturze można łatwiej odczytywać tylko interesujące obiekty.

\begin{scriptsize}
\begin{verbatim}
/* Wiktor Zychla, 2003 */
using System;
using System.IO;
using System.Xml;
using System.Xml.XPath;

namespace Example
{
  public class CMain
  {   
    public static void InfoOOsobie( XPathNavigator node )
    {
       Console.WriteLine( "Osoba" );

       Console.WriteLine( "Imię:\t\t" + node.GetAttribute( "Imię", "" ) );
       Console.WriteLine( "Nazwisko:\t" + node.GetAttribute( "Nazwisko", "" ) );
       Console.WriteLine( "Data ur.:\t" + node.GetAttribute( "DataUr", "" ) );
    }

    public static void Main()
    {
       XmlDocument xml = new XmlDocument();
       xml.Load( "osoby.xml" );

       XPathNavigator    n = xml.CreateNavigator();
       XPathNodeIterator i = n.Select( "//ListaOsób/Osoba" );

       while ( i.MoveNext() )
         InfoOOsobie( i.Current );
    }
  }
}

C:\Example>example
Osoba
Imię:           Jan
Nazwisko:       Kowalski
Data ur.:       1950-02-07
Osoba
Imię:           Adam
Nazwisko:       Nowak
Data ur.:       1992-12-23
Osoba
Imię:           Tomasz
Nazwisko:       Malinowski
Data ur.:       1979-10-01
\end{verbatim}
\end{scriptsize}

\subsubsection{Automatyczne tworzenie kodu do ładowania XML}

Pliki XML przechowujące dane mają najczęściej prostą strukturę. Czy nie można wykorzystać tego
spostrzeżenia do zautomatyzowania tworzenia kodu parsującego plik XML?

Okazuje się, że jest to możliwe. W skład programów narzędziowych, udostępnianych z {\bf .NET Framework SDK}
znajduje się niepozorne narzędzie o nazwie {\bf xsd.exe}. To ono właśnie pozwala znacznie uprościć
proces tworzenia kodu dla danych XML. 

Punktem wyjścia będą dane. Zaczynamy od zaprojektowania struktury o dowolnej pojemności informacyjnej. 
Na przykład takiej (nazwimy te dane {\bf dane.xml}):

\begin{scriptsize}
\begin{verbatim}
<?xml version="1.0" encoding="windows-1250"?>
<ListaOsob xmlns='MojeDane'>
   <Osoba obyw_polskie="T">
    <Imie>Jan</Imie>
    <Nazwisko>Kowalski</Nazwisko>
    <DataUr>1950-02-07</DataUr>
  </Osoba>
  <Osoba obyw_polskie="T">
    <Imie>Tomasz</Imie>
    <Nazwisko>Malinowski</Nazwisko>
    <DataUr>1976-10-04</DataUr>
  </Osoba>
  <Osoba obyw_polskie="T">
    <Imie>Adam</Imie>
    <Nazwisko>Nowak</Nazwisko>
    <DataUr>1984-02-17</DataUr>
  </Osoba>
</ListaOsob>
\end{verbatim}
\end{scriptsize}

Nowym elementem w tej definicji jest atrybut {\bf xmlns} w głównym węźle danych, który wszystkie dane
przypisuje do przestrzeni nazw {\bf MojeDane}. Będzie to ważne wtedy, kiedy będziemy weryfikowali poprawność
otrzymanych danych.

Następny krok polega na uruchomieniu {\bf xsd.exe} i wskazaniu pliku XML jako parametru. Efektem będzie plik 
XSD, zawierający w sobie informację o strukturze podanego pliku XML.

\begin{scriptsize}
\begin{verbatim}
<?xml version="1.0" encoding="utf-8"?>
<xs:schema id="ListaOsob" targetNamespace="MojeDane" 
xmlns:mstns="MojeDane" xmlns="MojeDane" xmlns:xs="http://www.w3.org/2001/XMLSchema" 
xmlns:msdata="urn:schemas-microsoft-com:xml-msdata" 
attributeFormDefault="qualified" elementFormDefault="qualified">
  <xs:element name="ListaOsob" msdata:IsDataSet="true" msdata:Locale="pl-PL">
    <xs:complexType>
      <xs:choice maxOccurs="unbounded">
        <xs:element name="Osoba">
          <xs:complexType>
            <xs:sequence>
              <xs:element name="Imie" type="xs:string" minOccurs="0" msdata:Ordinal="0" />
              <xs:element name="Nazwisko" type="xs:string" minOccurs="0" msdata:Ordinal="1" />
              <xs:element name="DataUr" type="xs:string" minOccurs="0" msdata:Ordinal="2" />
            </xs:sequence>
            <xs:attribute name="obyw_polskie" form="unqualified" type="xs:string" />
          </xs:complexType>
        </xs:element>
      </xs:choice>
    </xs:complexType>
  </xs:element>
</xs:schema>
\end{verbatim}
\end{scriptsize}

Tak przygotowany opis struktury danych może posłużyć do automatycznego wygenerowania kodu, który
będzie potrafił wczytywać pliki XML. Wystarczy ponownie uruchomić {\bf xsd.exe}, tym razem z przełącznikiem
{\bf classes} i jako parametr podać plik XSD. Efektem będzie plik zawierający w sobie
kod C\#.

\begin{scriptsize}
\begin{verbatim}
//------------------------------------------------------------------------------
// <autogenerated>
//     This code was generated by a tool.
//     Runtime Version: 1.0.3705.0
//
//     Changes to this file may cause incorrect behavior and will be lost if 
//     the code is regenerated.
// </autogenerated>
//------------------------------------------------------------------------------

// 
// This source code was auto-generated by xsd, Version=1.0.3705.0.
// 
using System.Xml.Serialization;

/// <remarks/>
[System.Xml.Serialization.XmlTypeAttribute(Namespace="MojeDane")]
[System.Xml.Serialization.XmlRootAttribute("ListaOsob", 
                             Namespace="MojeDane", IsNullable=false)]
public class ListaOsob {
    
    /// <remarks/>
    [System.Xml.Serialization.XmlElementAttribute("Osoba")]
    public ListaOsobOsoba[] Items;
}

/// <remarks/>
[System.Xml.Serialization.XmlTypeAttribute(Namespace="MojeDane")]
public class ListaOsobOsoba {
    
    /// <remarks/>
    public string Imie;
    
    /// <remarks/>
    public string Nazwisko;
    
    /// <remarks/>
    public string DataUr;
    
    /// <remarks/>
    [System.Xml.Serialization.XmlAttributeAttribute(
     Form=System.Xml.Schema.XmlSchemaForm.Unqualified)]
    public string obyw_polskie;
}
\end{verbatim}
\end{scriptsize}

Jak widać kod ten zawiera klasy {\bf ListaOsob} i {\bf ListaOsobOsoba}, których składowe odpowiadają
relacjom między odpowiednimi węzłami opisanymi w strukturze XML. Wystarczy jedynie dopisać kod do 
deserializacji obiektu\footnote{Mechanizm serializacji danych opisano dokładniej w 
rozdziale \ref{csSerializacja}}:

\begin{scriptsize}
\begin{verbatim}
using System;
using System.IO;
using System.Xml;
using System.Xml.Serialization;

public class CMain
{
  public static void Main()
  {
    XmlSerializer xS = new XmlSerializer(typeof(ListaOsob));

    FileStream fs = new FileStream( "dane.xml", FileMode.Open );
    ListaOsob ds = (ListaOsob)xS.Deserialize( fs );

    foreach ( ListaOsobOsoba o in ds.Items )
    {  
      Console.WriteLine( String.Format( "{0}, {1}, {2}", 
      o.Imie, o.Nazwisko, o.DataUr ) );
    }
  }
}
\end{verbatim}
\end{scriptsize}

Całość można już skompilować i uruchomić:

\begin{scriptsize}
\begin{verbatim}
C:\Example>csc.exe example.cs dane.cs
Microsoft (R) Visual C# .NET Compiler version 7.10.3052.4
for Microsoft (R) .NET Framework version 1.1.4322
Copyright (C) Microsoft Corporation 2001-2002. All rights reserved.


C:\Example>example
Jan, Kowalski, 1950-02-07
Tomasz, Malinowski, 1976-10-04
Adam, Nowak, 1984-02-17
\end{verbatim}
\end{scriptsize}

Podsumujmy: dysponując danymi w postaci XML mogliśmy automatycznie wygenerować opis struktury danych w postaci
pliku XSD, zaś ten posłużył do automatycznego wygenerowania kodu C\#. Po dopisaniu kilku linijek kodu
deserializującego obiekt, otrzymaliśmy możliwość łatwego ładowania pliku XML i przenoszenia jego
zawartości do programu. 

\subsubsection{Weryfikacja poprawności danych XML}

Okazuje się, że schemat XSD nadaje się nie tylko do automatycznego generowania kodu odpowiednich klas do
przechowywania danych w programie, ale może być wykorzystany do dynamicznej walidacji poprawności danych XML.

Do tego celu wykorzystamy obiekt typu {\bf XmlValidatingReader}. Podczas ładowania dokumentu sprawdza on
poprawność danych, a ściśle - zgodność ze specyfikacją zawartą w strukturze XSD. Ewentualne błędy
lub ostrzeżenia są zgłaszane za pomocą zdarzenia {\bf ValidationEventHandler}. 

\begin{scriptsize}
\begin{verbatim}
using System;
using System.IO;
using System.Xml;
using System.Xml.Schema;
 
namespace Example
{
  public class CMain
  {   
    public static void Main()
    {
       XmlDocument xml = new XmlDocument();

       XmlTextReader tr = new XmlTextReader( "dane.xml" );
       XmlValidatingReader reader = new XmlValidatingReader(tr);

       reader.ValidationType = ValidationType.Schema;
       reader.ValidationEventHandler += new ValidationEventHandler (ValidationHandler);

       xml.Load( reader );

       Console.WriteLine( "Wczytano plik." );
    }

     public static void ValidationHandler(object sender, ValidationEventArgs args)
     {
        Console.WriteLine("*Błąd walidacji*");
        Console.WriteLine("\tWaznosc: {0}", args.Severity);
        Console.WriteLine("\tInfo:    {0}", args.Message);
     }
  }
}
\end{verbatim}
\end{scriptsize}

Efekt uruchomienia powyższego programu może być w pierwszej chwili dość zaskakujący: każda linia
pliku z danymi powoduje zgłoszenie ostrzeżenia o braku schematu do walidacji. Dzieje się tak dlatego, że
nigdzie nie skojarzyliśmy pliku z danymi (dane.xml) z plikiem zawierającym schemat struktury (dane.xsd).
Poza tym, że fizycznie mogą znajdować się w jednym folderze, nic tych plików nie łączy. 

Aby plik XML był walidowany przy użyciu schematu XSD, należy odpowiednią informację zaszyć
we wnętrzu pliku XML. 

\begin{scriptsize}
\begin{verbatim}
<?xml version="1.0" encoding="windows-1250"?>
<ListaOsob xmlns='MojeDane' 
           xmlns:MojWalidator='http://www.w3.org/2001/XMLSchema-instance'
           MojWalidator:schemaLocation='MojeDane dane.xsd'>
  ...
</ListaOsob>
\end{verbatim}
\end{scriptsize}

Tak osygnowany plik XML będzie już poprawnie walidowany i każde odstępstwo od schematu będzie
wyłapywane jako błąd. 

Dokładne zapoznanie się z możliwościami schematów XSD umożliwia bardzo szczegółowe
zapanowanie nad regułami walidacji dokumentów XML. Jest to bardzo przydatne w sytuacji, kiedy
producent danych nie jest znany, nie jest wiarygodny bądź po prostu struktura danych ulega modyfikacji
w czasie życia aplikacji.