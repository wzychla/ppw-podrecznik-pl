\subsection{Biblioteka funkcji matematycznych}

Funkcje matematyczne zostały w C\# umieszczone jako statyczne w klasie {\tt System.Math}.
Programista znajdzie tam takie funkcje, jak m.in.: Abs, Asin, Atan, Cos, Cosh, E (stała), 
Exp, Floor, Log, Max, Min, PI (stała), Pow, Sign, Sin, Sinh, Tan, Sqrt.

\begin{scriptsize}
\begin{verbatim}
/* Wiktor Zychla, 2003 */
using System;
using System.Collections;

namespace Example
{
  public class CExample 
  {
    public static void Main(string[] args)
    {
      Console.WriteLine( "E  = {0}", Math.E );
      Console.WriteLine( "Pi = {0}", Math.PI );

      Console.WriteLine( "E^PI  = {0}", Math.Pow( Math.E, Math.PI ) );
      Console.WriteLine( "Pi^E  = {0}", Math.Pow( Math.PI, Math.E ) );
    }
  }
}

C:\Example>example.exe
E  = 2,71828182845905
Pi = 3,14159265358979
E^PI  = 23,1406926327793
Pi^E  = 22,459157718361
\end{verbatim}
\end{scriptsize}

Podobnie łatwo dzięki klasie {\tt Random} można uzyskać liczby losowe. 

\begin{scriptsize}
\begin{verbatim}
/* Wiktor Zychla, 2003 */
using System;
using System.Collections;

namespace Example
{
  public class CExample 
  {
    public static void Main(string[] args)
    {
      Random r = new Random();

      Console.WriteLine( "Sekwencja losowych liczb całkowitych: " );
      for ( int i=0; i<10; i++ )
        Console.WriteLine( r.Next() );

      Console.WriteLine( "Sekwencja losowych liczb zmiennoprzecinkowych: " );
      for ( int i=0; i<10; i++ )
        Console.WriteLine( r.NextDouble() );
    }
  }
}

C:\Example>example.exe
Sekwencja losowych liczb całkowitych:
1537211907
1960381545
1107103792
1638000156
206550390
4349299
1902247774
493693260
357003656
1461388247
Sekwencja losowych liczb zmiennoprzecinkowych:
0,359765542372952
0,615490057792277
0,185565924358352
0,0885926261956769
0,67311383582331
0,495275298829784
0,700976026105218
0,496796116464211
0,58237144610955
0,495158822506274
\end{verbatim}
\end{scriptsize}

\subsection{Biblioteki wejścia/wyjścia}

Tyle ile różnych języków - tyle różnych podejść do zagadnienia obsługi wejścia / wyjścia. 
Projektanci języka stają przed trudnym zadaniem zaprojektowania przystępnego iterfejsu programowania
do obsługi różnego rodzaju obiektów (pliki, konsola, połączenia sieciowe itd.) i różnego rodzaju 
rodzajów przekazywania danych (tekstowy, binarny, buforowany, dostęp sekwencyjny i swobodny itd.).

\subsubsection{Struktura systemu plików}

Zanim przejdziemy do przekazywania danych z i do strumieni reprezentujących obiekty wejścia / wyjścia, 
zajmiemy się operacjami na strukturze systemu plików. Biblioteka udostępnia tu 3 klasy: {\bf File}, 
{\bf Directory} i {\bf Path}. Żadna z tych klas nie pozwala tworzyć swojej instancji, udostępniają 
one tylko statyczne metody, z których korzysta programista.

Klasa {\bf Directory} służy do bezpośrednich operacji na plikach i katalogach. Udostępnia m.in.
następujące metody:
\begin{description}
\item [CreateDirectory] Tworzenie katalogu.
\item [Delete] Usuwanie katalogu.
\item [Exists] Sprawdzanie czy katalog istnieje.
\item [GetCurrentDirectory] Zwraca bieżący katalog.
\item [GetFiles] Zwraca listę nazw plików w katalogu.
\item [GetDirectories] Zwraca listę podkatalogów w katalogu.
\item [GetFileSystemEntries] Zwraca listę wszystkich elementów w katalogu.
\item [GetParent] Zwraca nazwę katalogu poziom wyżej niż wskazany.
\item [GetLogicalDrives] Zwraca listę dysków logicznych w systemie.
\item [Move] Przesuwa katalog w systemie plików.
\item [SetCurrentDirectory] Ustawia bieżący katalog.
\end{description}

Klasa {\bf File} udostępnia metody do operacji na plikach, m.in.:
\begin{description}
\item [Copy] Kopiowanie plików.
\item [Create] Tworzenie nowych plików.
\item [Delete] Usuwanie plików. 
\item [Exists] Sprawdzanie czy plik istnieje.
\item [GetAttributes] Zwraca atrybuty pliku.
\item [Open] Otwiera plik.
\item [SetAttributes] Ustawia atrybuty pliku.
\end{description}

Klasa {\bf Path} udostępnia metody do obsługi nazw plików w systemie plików, m.in.:
\begin{description}
\item [ChangeExtension] Zmiana rozszerzenia nazwy pliku.
\item [GetDirectoryName] Część określająca nazwę katalogu w ścieżce.
\item [GetExtension] Rozszerzenie pliku.
\item [GetFileName] Nazwa pliku (bez ścieżki).
\item [GetFileNameWithoutExtension] Nazwa pliku (bez ścieżki i rozszerzenia).
\item [GetFullPath] Pełna nazwa pliku.
\item [GetTempName] Nazwa tymczasowego pliku.
\item [DirectorySeparatorChar] Separator katalogów w nazwach plików (w Windows "\").
\item [PathSeparator] Separator ścieżek w nazwach plików (w Windows ";").
\item [VolumeSeparatorChar] Separator woluminu w nazwach plików (w Windows ":")
\end{description}

Dodatkową, usługową funkcję spełniają dwie klasy, {\bf FileInfo} i {\bf DirectoryInfo}. Za pomocą
obiektów tych klas można uzyskać szczegółowe informacje na temat plików i folderów.

Przykład prostego programu:

\begin{scriptsize}
\begin{verbatim}
/* Wiktor Zychla, 2003 */
using System;
using System.IO;

namespace Example
{
  public class CExample 
  {
    public static void Main(string[] args)
    {
       string[] fileNames = 
         Directory.GetFiles( Directory.GetCurrentDirectory(), "*.exe" );
       foreach ( string s in fileNames )
       {
         FileInfo fi = new FileInfo( s );

         Console.WriteLine( fi.FullName );
         Console.WriteLine( " rozmiar\t{0}", fi.Length );
         Console.WriteLine( " utworzony\t{0}", fi.CreationTime );
         Console.WriteLine( " atrybuty\t{0}", fi.Attributes );
       }
    }
  }
}

C:\Example>example.exe
C:\Example\example.exe
 rozmiar        3584
 utworzony      2003-03-22 19:47:55
 atrybuty       Archive
\end{verbatim}
\end{scriptsize}

\subsubsection{Obsługa danych w strumieniach}

Interfejsy nowoczesnych języków programowania zwykle używają abstrakcyjnej reprezentacji
danych przesyłanych do i z urządzeń wejścia / wyjścia w postaci {\em strumieni}. Udaną próbę
zbudowania jednolitego interfejsu strumieni podjęto przy projektowaniu C++.

\begin{table}[ht]
	\begin{center}

	\framebox[12cm]{MemoryStream, FileStream, NetworkStream}  \\ $\bigtriangledown$ \\
	\framebox[12cm]{CryptoStream, BufferedStream}             \\ $\bigtriangledown$ \\
	\framebox[12cm]{StreamReader, StreamWriter, BinaryReader, BinaryWriter}  

	\end{center}
\caption{Składanie różnych funkcji strumieni} 
\label{tab:Streams}
\end{table}

W C\# istnieje klasa {\em Stream}, która, oprócz udostępniania kilku prostych metod, spełnia
funkcję klasy bazowej dla specjalizowanych klas do obsługi różnych strumieni:

\begin{description}
\item [MemoryStream] dostarcza mechanizmów do operacji na danych w pamięci
\item [FileStream] dostarcza mechanizmów do operacji na plikach
\item [IsolatedStorageFileStream] wirtualny system plików z kontrolą dostępu
\item [NetworkStream] dostarcza mechanizmów do operacji sieciowych
\end{description}

Dodatkowe strumienie mogą stanowić warstwę pośrednią w komunikacji z wyżej wymienionymi
strumieniami:

\begin{description}
\item [CryptoStream] pozwala szyfrować i deszyfrować dane przesyłane z i do strumienia
\item [BufferedStream] pozwala przyspieszyć dostęp do strumienia przez wysyłanie większych porcji danych
\end{description}

W zależności od tego jakiego rodzaju dostępu do strumienia oczekuje programista, może on wybierać między:

\begin{description}
\item [StreamReader, StreamWriter] pozwala na dostęp do strumieni traktowanych jako napisy
\item [BinaryReader, BinaryWriter] pozwala na dostęp do strumieni traktowanych jak bajty
\end{description}

Te trzy rodzaje funkcji tworzą niejako trzy niezależne warstwy obsługi strumieni, zaś
programista może dowolnie składać funkcje z kolejnych warstw. Oznacza to, że tak naprawdę
istnieje kilkadziesiąt różnych możliwości ich składania (tabela \ref{tab:Streams}).

Warstwa pierwsza udostępnia najprostszy interfejs, w którym do strumienia można kierować i czytać
tylko pojedyńcze bajty. Warstwa druga umożliwia nałożenie szyfrowania lub buforowania na strumień.
Warstwa trzecia pozwala na wysyłanie do strumienia całych napisów, liczb i innych obiektów.

Spróbujmy więc na przykład utworzyć strumień plikowy, na niego nałożyć funkcję zapisu tekstu
w Unicode i zapisać do pliku jakiś tekst. 

\begin{scriptsize}
\begin{verbatim}
/* Wiktor Zychla, 2003 */
using System;
using System.Text;
using System.IO;

namespace Example
{
  public class CExample 
  {
    public static void Main(string[] args)
    {
      FileStream fs = new FileStream( "plik.txt", FileMode.Create );
      StreamWriter sw = new StreamWriter( fs, Encoding.Unicode );

      sw.WriteLine( "Chrząsz brzmi w Żyrardówku" );
      
      sw.Close();
      fs.Close();
    }
  }
}
\end{verbatim}
\end{scriptsize}

\subsubsection{Szyfrowanie strumieni w locie}

Biblioteka wejścia / wyjścia .NET pozwala na umieszczenie strumienia szyfrującego 
między strumieniem, a obiektem pozwalającym czytać bądź pisać dane tego strumienia. 
Jest to naprawdę proste i wygodne - z perspektywy programisty zachowanie się strumienia
jest nadal takie samo, mimo to dane trafiają do strumienia po przejściu przez warstwę szyfrującą.

\begin{center}
$$\begin{array}{ll}
\mbox{Typ} & \mbox{Nazwa}  \\
\hline 
\mbox{Symetryczny} & \mbox{DES} \\
\mbox{Symetryczny} & \mbox{TripleDES} \\
\mbox{Symetryczny} & \mbox{RC2} \\
\mbox{Symetryczny} & \mbox{Rijndael} \\
\mbox{Asymetryczny} & \mbox{DSA} \\
\mbox{Asymetryczny} & \mbox{RSA} 
\end{array}$$
\end{center}

Udostępniono kilka znanych protokołów kryptograficznych: {\em symetryczne} używają tego samego
klucza do szyfrowania i deszyfrowania, podczas gdy {\em asymetryczne} szyfrują za pomocą kluczy
publicznych, zaś do odszyfrowania potrzebują kluczy prywatnych. Biblioteka krytpograficzna
udostępnia metody do wspomagania tworzenia kluczy dla obu typów protokołów.

W przykładzie najpierw utworzymy strumień do zapisu danych z pośrednim strumieniem 
szyfrującym, a następnie zdekodujemy tekst z pliku. Gdyby podczas próby dekodowania użyto
niepoprawnego hasła, to oczywiście operacja nie powiodłaby się. Oczywiście zawartość pliku z
zaszyfrowaną informacją w żaden sposób nie nadaje się do odczytania bez zdeszyfrowania.

\begin{scriptsize}
\begin{verbatim}
/* Wiktor Zychla, 2003 */
using System;
using System.IO;
using System.Security.Cryptography; 
using System.Text;

namespace Example
{
  public class CExample 
  {

    static string CzytajHaslo()
    {
      Console.Write( "podaj haslo do szyfrowania: " );
      string passwd = Console.ReadLine();  
      if ( passwd.Length != 8 ) 
      {
        Console.WriteLine( "Haslo musi miec 8 znakow" );
        Environment.Exit(0);
      }
      return passwd;
    }

    public static void Main(string[] args)
    {
      string password    = CzytajHaslo();       
      UnicodeEncoding UE = new UnicodeEncoding();
      byte[] key         = UE.GetBytes(password);

      // zapis zaszyfrowanych danych
      // cs jest strumieniem pośrednim
      FileStream      fs = new FileStream( "plik.txt", FileMode.Create );
      RijndaelManaged RMCrypto = new RijndaelManaged();
      CryptoStream    cs = new CryptoStream(fs,
                           RMCrypto.CreateEncryptor(key, key),   
                           CryptoStreamMode.Write);      
      StreamWriter sw = new StreamWriter( cs, Encoding.Unicode );  

      sw.WriteLine( "Chrząsz brzmi w Żyrardówku" );      
      sw.Close();

      // odczyt zaszyfrowanych danych
      // gs jest strumieniem pośrednim 
      FileStream      gs = new FileStream( "plik.txt", FileMode.Open );
      RijndaelManaged RMCryptp = new RijndaelManaged();
      CryptoStream    ds = new CryptoStream(gs,
                           RMCryptp.CreateDecryptor(key, key),   
                           CryptoStreamMode.Read);      
      StreamReader    sq = new StreamReader( ds, Encoding.Unicode );  
 
      Console.WriteLine( sq.ReadLine() );
      sq.Close(); 
    }
  }
}

C:\Example>example.exe
podaj haslo do szyfrowania: qwertyui
Chrząsz brzmi w Żyrardówku
\end{verbatim}
\end{scriptsize}

\subsubsection{Strumienie konsoli}

Obiekt reprezentujący konsolę dysponuje informacją o strumieniach wejścia, wyjścia i błędu. Obiekty te
({\tt Console.In, Console.Out, Console.Error}) są strumieniami typów {\bf TextReader} i
{\bf TextWriter} (klasy bazowe dla odpowiednio StreamReader, StringReader i StreamWriter, StringWriter).
Oznacza to, że strumieni tych można użyć w każdym kontekście, w którym używa się strumieni pochodnych.

Strumienie te mogą być przekierowane za pomocą metod {\bf SetIn}, {\bf SetOut} i {\bf SetError}.

