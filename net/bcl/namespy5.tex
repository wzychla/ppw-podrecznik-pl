\subsection{Komunikacja między procesami}

Biblioteka {\em System.Net} udostępnia komplet funkcji wspomagających komunikację za pomocą
mechanizmów sieciowych. Programista znajdzie tu nie tylko obiektowe interfejsy dla gniazd, ale również
wyspecjalizowane funkcje do bezpośredniej komunikacji z wybranymi usługami sieciowymi.

\begin{scriptsize}
\begin{verbatim}
/* Wiktor Zychla, 2003 */
using System;
using System.IO
using System.Net;

namespace SimpleHttpReader
{
  public class CMain
  {
    public static void Main()
    {
      Uri          uri = new Uri( "http://www.ii.uni.wroc.pl" );
      WebRequest  req  = WebRequest.Create( uri );
      WebResponse resp = req.GetResponse();
      Stream    stream = resp.GetResponseStream();
      StreamReader  sr = new StreamReader( stream );

      string s = sr.ReadToEnd();
      Console.Write( s );    
    }
  }
}
\end{verbatim}
\end{scriptsize}

Aby przekonać się jak łatwo oprogramować gniazda, napiszmy .NETowe odpowiedniki serwera i klienta z
rozdziału \ref{apiKomunikacja}.

Kod serwera:

\begin{scriptsize}
\begin{verbatim}
/* Wiktor Zychla, 2003 */
// prosty moduł serwera
// użycie: server.exe
using System;
using System.IO;
using System.Net;
using System.Net.Sockets;
using System.Threading;

namespace TcpSever
{
  public class CClientThread
  {
    const  string serverMsg = "Tu serwer. Odpowiadam.";
    Socket clientSocket;

    public CClientThread( Socket clientSocket )
    {
      this.clientSocket = clientSocket;
    }

    public void ClientThreadFunc()
    {
      NetworkStream nS = new NetworkStream( clientSocket );
      StreamReader  sR = new StreamReader( nS );
      StreamWriter  sW = new StreamWriter( nS );

      while( true )
      {
        // odbierz wiadomość
        string clientMessage = sR.ReadLine();
        if ( clientMessage != null )
          Console.WriteLine( "Otrzymano wiadomość: {0}", clientMessage );
        else
          break;

        // odeślij odpowiedź
        sW.WriteLine( serverMsg );
        sW.Flush();
      }

      sR.Close();
      sW.Close();
    }
  }

  public class CServer
  {
    public const int DEFAULT_PORT   = 5000;
 
    public static void Main()
    {     
      TcpListener tcpListener = new TcpListener( DEFAULT_PORT );
      tcpListener.Start() ;
	 
      Console.WriteLine( "Serwer nasłuchuje." );
      Console.WriteLine( "Adres: {0}, port: {1}", Dns.GetHostName(), DEFAULT_PORT );

      while ( true )
      {
        Thread.Sleep(1);
        if ( tcpListener.Pending() )
        {
          Socket socketForClient = tcpListener.AcceptSocket();
          string clientName = 
              String.Format( "{0} [{1}]",
                Dns.GetHostByAddress( ((IPEndPoint)socketForClient.RemoteEndPoint).Address ).HostName,
                ((IPEndPoint)socketForClient.RemoteEndPoint).Address.ToString()
				            );
          Console.WriteLine( "Zaakceptowano połączenie: serwer {0}", clientName );

          CClientThread ct = new CClientThread( socketForClient );
          Thread         t = new Thread( new ThreadStart( ct.ClientThreadFunc ) );
          t.Start();
        }
      }
    }
  }
}
\end{verbatim}
\end{scriptsize}

Kod klienta:

\begin{scriptsize}
\begin{verbatim}
/* Wiktor Zychla, 2003 */
// prosty moduł klienta
// użycie: klient.exe -s:IP
using System;
using System.IO;
using System.Net;
using System.Net.Sockets;
using System.Threading;

namespace TcpClientNS
{
  public class CClient
  {
    const int DEFAULT_COUNT=5;
    const int DEFAULT_PORT =5000;
    const string DEFAULT_MESSAGE="Tu klient. Witam.";

    static string szServer;

    static void sposob_uzycia()
    {
      Console.WriteLine( "client.exe -s:IP" );
      Environment.Exit(1);
    }

    static void WalidacjaLiniiPolecen( string[] args )
    {
      int i;

      if ( args.Length < 1 )
      {
        sposob_uzycia();
      }
	  
      for ( i=0; i<args.Length; i++ )
      {
        if ( args[i][0] == '-' )
        {
          switch( args[i][1].ToString().ToLower() )
          {
            case "s" : if ( args[i].Length > 3 )
                       {
                         szServer = args[i].Substring(3); 					     
                       };
                       break;
            default  : sposob_uzycia(); break;
          }
        }
      }
    }

    public static void Main( string[] args )
    {
      WalidacjaLiniiPolecen( args );

      TcpClient socketForServer = new TcpClient( szServer, DEFAULT_PORT );

      NetworkStream nS = socketForServer.GetStream();
      StreamWriter  sW = new StreamWriter( nS );
      StreamReader  sR = new StreamReader( nS );

      for ( int i=0; i<DEFAULT_COUNT; i++ )
      {
        sW.WriteLine( DEFAULT_MESSAGE );
        sW.Flush();
        string message = sR.ReadLine();
        if ( message != null )
          Console.WriteLine( "Serwer odpowiada: {0}", message ); 
      }

      sR.Close();
      sW.Close();
    }
  }  
}
\end{verbatim}
\end{scriptsize}

Jak widać, opakowanie gniazd w obiektowe klasy {\em TcpListener} i {\em TcpClient} (przeznaczone
do komunikacji przez TCP/IP), znacznie upraszcza kod odpowiedzialny za wysyłanie i odbieranie danych.

\subsection{Wyrażenia regularne}

Biblioteki do obsługi wyrażeń regularnych są standardowo dołączane do współczesnych języków
programowania. Szczycą się nimi zwłaszcza języki skryptowe, ale jak zobaczymy w kolejnych przykładach,
wszystko zależy od dobrej biblioteki.

\subsubsection{Czym są {\em wyrażenia regularne}}

Pojęcie "{\em wyrażenia regularne}" związane jest z przetwarzaniem tekstu. Wyrażenia
regularne tworzą pewien szczególny język, niezależny od żadnego języka programowania, za pomocą którego
definiujemy wzorce, wykorzystywane następnie do wyszukiwania, usuwania, czy zamieniania fragmentów tekstu.
Wyrażenia regularne znakomicie upraszają proces parsowania na przykład stron HTML, plików XML, logów
systemowych itd.

\subsubsection{Język wyrażeń regularnych}

Poniższa tabela podsumowuje wybrane zasady budowania wyrażeń regularnych.

\begin{center}
$$\begin{array}{ll}
\mbox{Wyrażenie} & \mbox{Opis}  \\
\hline 
\mbox{.}        & \mbox{Każdy znak za wyjątkiem $\backslash n$} \\
\mbox{[znaki]}  & \mbox{Pojedyńcze znaki z listy} \\
\mbox{[znakA-znakB]} & \mbox{Znaki z podanego zakresu} \\
\mbox{$\backslash$ w}            & \mbox{Znak tworzący słowo, równoważnie [a-zA-Z\_0-9]} \\
\mbox{$\backslash$ W}            & \mbox{Znak nie tworzący słowa} \\ 
\mbox{$\backslash$ s}            & \mbox{Znak biały, równoważnie [$\backslash$n$\backslash$r$\backslash$t$\backslash$f]} \\
\mbox{$\backslash$ S}            & \mbox{Znak inny niż biały} \\
\mbox{$\backslash$ d}            & \mbox{Cyfra, równoważnie [0-9]} \\
\mbox{$\backslash$ D}            & \mbox{Nie-cyfra} \\
\mbox{$\backslash$ b}            & \mbox{Na granicy słowa} \\
\mbox{$\backslash$ B}            & \mbox{Nie na granicy słowa} \\
\mbox{*}             & \mbox{Zero lub więcej} \\
\mbox{+}             & \mbox{Jeden lub więcej} \\
\mbox{?}             & \mbox{Zero lub jeden} \\
\mbox{$\{n\}$}       & \mbox{Dokładnie $n$-razy} \\ 
\mbox{$\{n,\}$}      & \mbox{Co najmniej $n$-razy} \\ 
\mbox{$\{n,m\}$}     & \mbox{Co najmniej $n$ ale nie więcej niż $m$}\\ 
\mbox{( )}         & \mbox{Podwyrażenie} \\
\mbox{(?$<$nazwa$>$)}  & \mbox{Podwyrażenie jako nazwa} \\
\mbox{$\vert$}         & \mbox{Alternatywa} \\ 
\end{array}$$
\end{center}

Tak naprawdę zaprojektowanie odpowiedniego często nie jest łatwe i wymaga po prostu trochę wprawy.  

\subsubsection{Dzielenie tekstu}

Pierwszym przykładem zastosowania wyrażeń regularnych jest dzielenie tekstu. Tekst jest dzielony
w miejscach, które dopasowują się do zadanego wyrażenia regularnego.

\begin{scriptsize}
\begin{verbatim}
/* Wiktor Zychla, 2003 */
using System;
using System.Text;
using System.Text.RegularExpressions;

class CExample 
{
  public static void Main()
  {
    string sSplit = "Ala  ma kota a    kot ma   Ale";
    Regex  r = new Regex( "[ ]+" );

    foreach( string s in r.Split( sSplit ) )
      Console.Write( s+"," );
  }  
}

C:\example>example
Ala,ma,kota,a,kot,ma,Ale,
\end{verbatim}
\end{scriptsize}

\subsubsection{Wyszukiwanie wzorca}

Wyszukiwanie zadanego wyrażeniem regularnym wzorca w zadanym tekście możliwe jest dzięki
obiektom {\bf Match} i {\bf MatchCollection}.

\begin{scriptsize}
\begin{verbatim}
/* Wiktor Zychla, 2003 */
using System;
using System.Text;
using System.Text.RegularExpressions;

class CExample 
{
  public static void Main()
  {
    string sFind = "Dobrze jest dojsc do domu radosnie i wydobrzec do rana";
    Regex  r = new Regex( "(do)|(a)" );

    for ( Match m = r.Match( sFind ); m.Success; m = m.NextMatch() )
      Console.Write( "'{0}' na pozycji {1}\n", m.Value, m.Index );
  }  
}

C:\example>example
'do' na pozycji 12
'do' na pozycji 18
'do' na pozycji 21
'a' na pozycji 27
'do' na pozycji 28
'do' na pozycji 39
'do' na pozycji 47
'a' na pozycji 51
'a' na pozycji 53
\end{verbatim}
\end{scriptsize}

\subsubsection{Edycja, usuwanie tekstu}

Dzięki metodzie {\bf Replace} wyrażeń regularnych można użyć do zastępowania tekstu.

\begin{scriptsize}
\begin{verbatim}
/* Wiktor Zychla, 2003 */
using System;
using System.Text;
using System.Text.RegularExpressions;

class CExample 
{
  public static void Main()
  {
    string sFind = "Dobrze jest dojsc do domu radosnie i wydobrzec do rana";
    Regex  r = new Regex( "(do)|(a)" );

    Console.Write( r.Replace( sFind, "" ) );
  }  
}

C:\example>example
Dobrze jest jsc  mu rsnie i wybrzec  rn
\end{verbatim}
\end{scriptsize}

\subsection{Serializacja}
\label{csSerializacja}

O {\em serializacji} mówimy wtedy, gdy instancja obiektu jest składowana na nośniku zewnętrznym.
Mechanizm ten wykorzystywany jest również do transferu zawartości obiektów między odległymi środowiskami.
Oczywiście łatwo wyobrazić sobie mechanizm zapisu zawartości obiektu przygotowany przez programistę,
ale serializacja jest mechanizmem niezależnym od postaci obiektu i od tego, czy programista przewidział
możliwość zapisu zawartości obiektu czy nie. 

\subsubsection{Serializacja binarna}

Aby zawartość obiektu mogła być składowana w postaci binarnej, klasa musi spełniać kilka warunków:
\begin{itemize}
\item Musi być oznakowana artybutem {\em Serializable}
\item Musi implementować interfejs {\em ISerializable}
\item Musi mieć specjalny konstruktor do deserializacji
\end{itemize}

\begin{scriptsize}
\begin{verbatim}
/* Wiktor Zychla, 2003 */
using System;
using System.IO;
using System.Runtime.Serialization;
using System.Runtime.Serialization.Formatters.Binary;
using System.Runtime.Serialization.Formatters.Soap;

namespace NExample
{
  [Serializable()]
  public class CObiekt : ISerializable
  {
    int      v;
    DateTime d;
    string   s;
	 
    public CObiekt( int v, DateTime d, string s )
    {
      this.v = v; this.d = d; this.s = s;
    }

	// konstruktor do deserializacji
    public CObiekt(SerializationInfo info, StreamingContext context) 
    {
      v = (int)info.GetValue("v", typeof(int));
      d = (DateTime)info.GetValue("d", typeof(DateTime));
      s = (string)info.GetValue("s", typeof(string));
    }


    // serializacja
    public void GetObjectData(SerializationInfo info, 
	                          StreamingContext context)
    {
      info.AddValue("v", v);
      info.AddValue("d", d);
      info.AddValue("s", s);
    }
    
    public override string ToString()
    {
      return String.Format( "{0}, {1:d}, {2}", v, d, s );
    }
  }

  public class CMain
  {
    static void SerializujBinarnie()
    {
      Console.WriteLine( "Serializacja binarna" );

      CObiekt o = new CObiekt( 5, DateTime.Now, "Ala ma kota" );
      Console.WriteLine( o );

      // serializuj
      Stream          s = File.Create( "binary.dat" );
      BinaryFormatter b = new BinaryFormatter();
      b.Serialize( s, o );
      s.Close();

      // deserializuj
      Stream          t = File.Open( "binary.dat", FileMode.Open );
      BinaryFormatter c = new BinaryFormatter();
      CObiekt         p = (CObiekt)c.Deserialize( t );
      t.Close();

      Console.WriteLine( "Po deserializacji: " + p.ToString() );
    }

    public static void Main()
    {
      SerializujBinarnie();
    }
  }
}

c:\Example>example.exe
Serializacja binarna
5, 2003-04-24, Ala ma kota
Po deserializacji: 5, 2003-04-24, Ala ma kota
\end{verbatim}
\end{scriptsize}

\subsubsection{Serializacja SOAP}

Serializacja binarna ma jak widać wady (wymaga specjalnie przygotowanej klasy), ma również zalety
(jest szybka, plik wynikowy zajmuje niewiele miejsca). 

Alternatywne podejście możliwe jest dzięki
mechanizmom SOAP ({\em Simple Object Access Protocol}). SOAP jest protokołem do wymiany danych,
opartym o nośnik XML, niezależny od systemu operacyjnego. Serializacja SOAP jest wolniejsza niż
serializacja binarna, wynik zajmuje więcej miejsca (w końcu to plik XML), jednak w ten sposób
można serializować dowolne obiekty.

\begin{scriptsize}
\begin{verbatim}
/* Wiktor Zychla, 2003 */
using System;
using System.IO;
using System.Runtime.Serialization;
using System.Runtime.Serialization.Formatters.Binary;
using System.Runtime.Serialization.Formatters.Soap;

namespace NExample
{
  [Serializable()]
  public class CObiekt 
  {
    int      v;
    DateTime d;
    string   s;
	 
    public CObiekt( int v, DateTime d, string s )
    {
      this.v = v; this.d = d; this.s = s;
    }
    
    public override string ToString()
    {
      return String.Format( "{0}, {1:d}, {2}", v, d, s );
    }
  }

  public class CMain
  {
    static void SerializujSOAP()
    {
      Console.WriteLine( "Serializacja SOAP" );

      CObiekt o = new CObiekt( 5, DateTime.Now, "Ala ma kota" );
      Console.WriteLine( o );

      // serializuj
      Stream          s = File.Create( "binary.soap" );
      SoapFormatter   b = new SoapFormatter();
      b.Serialize( s, o );
      s.Close();

      // deserializuj
      Stream          t = File.Open( "binary.soap", FileMode.Open );
      SoapFormatter   c = new SoapFormatter();
      CObiekt         p = (CObiekt)c.Deserialize( t );
      t.Close();

      Console.WriteLine( "Po deserializacji: " + p.ToString() );

    }

    public static void Main()
    {
      SerializujSOAP();
    }
  }
}

c:\Example>example.exe
Serializacja binarna
5, 2003-04-24, Ala ma kota
Po deserializacji: 5, 2003-04-24, Ala ma kota
\end{verbatim}
\end{scriptsize}

\subsection{Wołanie kodu niezarządzanego}

Współpraca z już istniejącymi bibliotekami jest bardzo ważnym elementem platformy .NET. Programista
może nie tylko wołać funkcje z natywnych bibliotek, ale również korzystać z bibliotek obiektowych COM.

\begin{scriptsize}
\begin{verbatim}
/* Wiktor Zychla, 2003 */
using System;
using System.Runtime.InteropServices;

namespace NExample
{
  public class CMain
  {
    [DllImport("user32.dll", EntryPoint="MessageBox")]
    public static extern int MsgBox(int hWnd, String text, 
                                    String caption, uint type);

    public static void Main()
    {
      MsgBox( 0, "Witam", "", 0 );
    }
  }
}
\end{verbatim}
\end{scriptsize}

Wygląda to dość prosto, jednak w rzeczywistości wymaga starannego przekazania parametrów
do funkcji napisanej najczęściej w C, a następnie odebrania wyników. Każdy typ w świecie .NET ma 
domyślnie swojego odpowiednika w kodzie niezarządzanym, który będzie używany w komunikacji między oboma
światami. Na przykład domyślny sposób przekazywana zmiennej zadeklarowanej jako {\em string} to LPSTR
(wskaźnik na tablicę znaków). Programista może dość szczegółowo zapanować nad domyślnymi konwencjami dzięki
atrybutowi {\em MarshalAs}.

\begin{scriptsize}
\begin{verbatim}
/* Wiktor Zychla, 2003 */
using System;
using System.Runtime.InteropServices;

namespace NExample
{
  public class CMain
  {
    [DllImport("user32.dll", EntryPoint="MessageBox")]
    public static extern int MsgBox(int hWnd, 
                                    [MarshalAs(UnmanagedType.LPStr)]
                                    String text, 
	                            String caption, uint type);

    public static void Main()
    {
      MsgBox( 0, "Witam", "", 0 );
    }
  }
}
\end{verbatim}
\end{scriptsize}

Aby ustalić w ten sposób typ wartości zwracanej z funkcji należałoby napisać:

\begin{scriptsize}
\begin{verbatim}
...
[DllImport("user32.dll", EntryPoint="MessageBox")]
[return: MarshalAs(UnmanagedType.I4)]
public static extern int MsgBox(int hWnd, ...
...
\end{verbatim}
\end{scriptsize}

Możliwość tak dokładnego wpływania na postać parametrów jest szczególnie przydatna w typowym
przypadku przekazywania jakiejś struktury do jakiejś funkcji, na przykład z Win32API.

Przykładowa struktura z Win32API

\begin{scriptsize}
\begin{verbatim}
typedef struct tagLOGFONT 
{ 
   LONG lfHeight; 
   LONG lfWidth; 
   LONG lfEscapement; 
   LONG lfOrientation; 
   LONG lfWeight; 
   BYTE lfItalic; 
   BYTE lfUnderline; 
   BYTE lfStrikeOut; 
   BYTE lfCharSet; 
   BYTE lfOutPrecision; 
   BYTE lfClipPrecision; 
   BYTE lfQuality; 
   BYTE lfPitchAndFamily; 
   TCHAR lfFaceName[LF_FACESIZE]; 
} LOGFONT; 
\end{verbatim}
\end{scriptsize}

powinna być przetłumaczona tak, aby zachować kolejność ułożenia pól oraz ograniczoną długość napisu.

\begin{scriptsize}
\begin{verbatim}
[StructLayout(LayoutKind.Sequential)]
public class LOGFONT 
{ 
    public const int LF_FACESIZE = 32;
    public int lfHeight; 
    public int lfWidth; 
    public int lfEscapement; 
    public int lfOrientation; 
    public int lfWeight; 
    public byte lfItalic; 
    public byte lfUnderline; 
    public byte lfStrikeOut; 
    public byte lfCharSet; 
    public byte lfOutPrecision; 
    public byte lfClipPrecision; 
    public byte lfQuality; 
    public byte lfPitchAndFamily;
    [MarshalAs(UnmanagedType.ByValTStr, SizeConst=LF_FACESIZE)]
    public string lfFaceName; 
}
\end{verbatim}
\end{scriptsize}

Czasami nawet konieczne jest dokładne wyznaczenie położenia wszystkich pól
struktury.

\begin{scriptsize}
\begin{verbatim}
[StructLayout(LayoutKind.Explicit, Size=16, CharSet=CharSet.Ansi)]
public class MySystemTime 
{
   [FieldOffset(0)]public ushort wYear; 
   [FieldOffset(2)]public ushort wMonth;
   [FieldOffset(4)]public ushort wDayOfWeek; 
   [FieldOffset(6)]public ushort wDay; 
   [FieldOffset(8)]public ushort wHour; 
   [FieldOffset(10)]public ushort wMinute; 
   [FieldOffset(12)]public ushort wSecond; 
   [FieldOffset(14)]public ushort wMilliseconds; 
}
\end{verbatim}
\end{scriptsize}

\subsubsection{Funkcje zwrotne}

Funkcje Win32Api, które zwracają więcej niż jeden element, najczęściej korzystają z mechanizmu funkcji
zwrotnych. Programista przekazuje wskaźnik na funkcję zwrotną, która jest wywoływana dla każdego elementu
na liście wyników (tak działa na przykład {\bf EnumWindows}, czy {\bf EnumDesktops}).

\begin{scriptsize}
\begin{verbatim}
BOOL EnumDesktops {
  HWINSTA hwinsta,  
  DESKTOPENUMPROC lpEnumFunc,
  LPARAM lParam
}
\end{verbatim}
\end{scriptsize}

Parametr typu {\tt HWINSTA} można przekazać jako {\tt IntPtr}, zaś {\tt LPARAM} jako {\tt int}.
Wskaźnik na funkcję

\begin{scriptsize}
\begin{verbatim}
BOOL CALLBACK EnumDesktopProc (
  LPTSTR lpszDesktop, LPARAM lParam
)
\end{verbatim}
\end{scriptsize}

należy zamienić na delegata

\begin{scriptsize}
\begin{verbatim}
delegate bool EnumDesktopProc(
  [MarshalAs(UnmanagedType.LPTStr)]
  string desktopName, int lParam
)
\end{verbatim}
\end{scriptsize}

Definicja funkcji {\bf EnumDesktops} będzie więc wyglądać tak:

\begin{scriptsize}
\begin{verbatim}
[DllImport("user32.dll"), CharSet = CharSet.Auto)]
static extern bool EnumDesktops (
  IntPtr windowStation,
  EnumDesktopProc callback,
  int lParam
)
\end{verbatim}
\end{scriptsize}

\subsection{Odśmiecacz}

Mechanizm odśmiecania funkcjonuje samodzielnie, bez kontroli programisty. W szczególnych
sytuacjach odśmiecanie może być wymuszone przez wywołanie metody obiektu odśmiecacza:

\begin{scriptsize}
\begin{verbatim}
GC.Collect();
\end{verbatim}
\end{scriptsize}

Należy pamiętać o tym, że destruktory obiektów są wykonywane w osobnym wątku, dlatego zakończenie
metody {\bf Collect} nie oznacza, że wszystkie destruktory są już zakończone. Można oczywiście wymusić
oczekiwanie na zakończenie się wszystkich czekających destruktorów:

\begin{scriptsize}
\begin{verbatim}
GC.Collect();
GC.WaitForPendingFinalizers();
\end{verbatim}
\end{scriptsize}

Działanie odśmiecacza jest dość proste. W momencie, w którym aplikacji brakuje pamięci, odśmiecacz
rozpoczyna przeglądanie wszystkich referencji od zmiennych statycznych, globalnych i lokalnych, oznaczając
kolejne obiekty jako używane. Wszystkie obiekty, które nie zostaną oznaczone, mogą zostać usunięte, bowiem
żaden aktualnie aktywny obiekt z nich nie korzysta.

Taki sposób postępowania, mimo że poprawny, byłby dość powolny. Dlatego w rzeczywistości wykorzystuje się
dodatkowo pojęcie tzw. {\em generacji}. Chodzi o to, że obiekt tuż po wykreowaniu 
należy do zerowej generacji obiektów, czyli obiektów "najmłodszych". Po "przeżyciu" odśmiecania,
obiektom inkrementuje się numery generacji. Kiedy odśmiecacz zabiera się za przeglądanie obiektów,
zaczyna od obiektów najmłodszych, dopiero jeśli okaże się, że pamięci nadal jest zbyt mało, usuwa obiekty
coraz starsze.

Idea ta ma proste uzasadnienie - obiektami najmłodszymi najcześciej będą na przykład zmienne lokalne funkcji
czy bloków kodu. Te zmienne powinny być usuwane najszybciej. Zmienne statyczne, kilkukrotnie wykorzystane 
w czasie działania programu, będą usuwane najpóźniej.

\begin{scriptsize}
\begin{verbatim}
using System;

public class CObiekt 
{
    private string name;
    public  CObiekt(string name) { this.name = name; }
    override public string ToString() { return name; }
}

namespace Example
{
  public class CMainForm
  {  
    const int IL = 3;
 
    public static void Main()
    {    
      Console.WriteLine( "Maksymalna generacja odsmiecacza " + GC.MaxGeneration );
      
      CObiekt[] t = new CObiekt[IL];

      Console.WriteLine( "Tworzenie obiektow." );
      for ( int i=0; i<IL; i++ )
      {     
        t[i] = new CObiekt( "obiekt " + i );
        Console.WriteLine( "{0}, generacja {1}", t[i], GC.GetGeneration( t[i] ) );
      } 

      // spróbuj usuwać nieużywane obiekty
      GC.Collect();
      GC.WaitForPendingFinalizers(); 

      Console.WriteLine( "Usuwanie obiektow." );
      for ( int i=0; i<IL; i++ )
      {
        Console.WriteLine( "{0}, generacja {1}", t[i], GC.GetGeneration( t[i] ) );
        t[i] = null;

        GC.Collect();
        GC.WaitForPendingFinalizers(); 
      }       
    }
  }
}

C:\Example>example
Maksymalna generacja odsmiecacza 2
Tworzenie obiektow.
obiekt 0, generacja 0
obiekt 1, generacja 0
obiekt 2, generacja 0
Usuwanie obiektow.
obiekt 0, generacja 1
obiekt 1, generacja 2
obiekt 2, generacja 2
\end{verbatim}
\end{scriptsize}
