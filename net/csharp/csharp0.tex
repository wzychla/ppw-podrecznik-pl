\section{Programowanie obiektowe}

Programowanie obiektowe jest naturalnym rozwinięciem programowania strukturalnego, możliwego chociażby 
w języku C. Program strukturalny składa się z szeregu {\em struktur danych} oraz {\em funkcji} operujących na 
tych strukturach. Struktury danych reprezentują {\em wiedzę} (dane), funkcje reprezentują {\em algorytmy}.

\begin{scriptsize}
\begin{verbatim}
struct Foo
{
  int Bar;
} foo;

int DoSomething( struct Foo f ) 
{
  return f.Bar + 1;
}

int main()
{
  printf( "%d", DoSomething( foo ) );
}
\end{verbatim}
\end{scriptsize}

Na język obiektowy można patrzeć jak na pewną {\em konwencję programowania strukturalnego}, prowadzącą do specyficznego 
uporządkowania kodu. Ta konwencja wypływa naprzeciw pewnej obserwacji - dość często jest tak, że algorytm można
przypisać jednej określonej strukturze danych na której operuje. 

Takie jednoznaczne przypisanie prowadzi do powstania pojęcia {\em klasy}, która łączy ze sobą dane i algorytmy w taki
szczególny sposób, w którym funkcja operująca na danych {\bf nie musi} otrzymywać tych danych w postaci jawnego argumentu
({\tt struct Foo f} w powyższym przykładzie), ponieważ ma do danych "swojego" (tego, który w wersji nieobiektowej
musiałby być przekazany jako jawny argument) obiektu zawsze dostęp za pomocą
zarezerwowanego słowa kluczowego {\tt this}:

\begin{scriptsize}
\begin{verbatim}
class Foo
{
  int Bar;

  int DoSomething() 
  {
    return this.Bar + 1;
  }
};

int main()
{
  Foo foo; 
  printf( "%d", DoSomething( foo ) );
}
\end{verbatim}
\end{scriptsize}

Skądinąd, jest to dość interesujące od strony technicznej, bowiem oznacza że metody klasy, które przynależą
do wystąpień obiektów (metody niestatyczne), mają w językach obiektowych ten jeszcze jeden jeden, niejawny argument 
i że rzeczywista liczba argumentów metody typu

\begin{scriptsize}
\begin{verbatim}
class Foo
{
  void DoSomething( int a, int b ) 
  {
  }
};
\end{verbatim}
\end{scriptsize}

to nie dwa ({\tt a} i {\tt b}) ale trzy ({\tt this}, {\tt a} i {\tt b}), co znaczy że tak naprawdę to jest to właśnie

\begin{scriptsize}
\begin{verbatim}
struct  Foo
{
}

void DoSomething( struct Foo this, int a, int b ) 
{
}
\end{verbatim}
\end{scriptsize}

Ta konwencja ma wiele ważnych pozytywnych konsekwencji, wśród których najważniejszą wydaje się zmiana sposobu
myślenia - o klasie zaczyna się myśleć jako o abstrakcie pewnego {\bf pojęcia}, często odpowiadającego
jakiemuś pojęciu ze świata rzeczywistego. Można o tych pojęciach mówić bardziej formalnie, wprowadzać elementy
{\em analizy obiektowej} i budować całe {\em modele pojęciowe}, które z kolei można mniej lub bardziej formalnie 
analizować itd.

Istnieją jednak również oczywiście takie konsekwencje przyjęcia konwencji obiektowej, które 
(przynajmniej przez pewien czas) mogą wydawać się niewygodne. 

Przede wszystkim - w przypadku funkcji/algorytmów operujących na danych więcej niż jednej struktury,
wybór tej właśnie jednej do której algorytm zostanie przypisany, nie wydaje się możliwy do formalizacji bez odniesienia
się do konkretnego przypadku. Mówiąc inaczej, mając

\begin{scriptsize}
\begin{verbatim}
struct Foo
{
  int foo;
};

struct Bar
{
  int bar;
}

int DoSomething( struct Foo f, struct Bar b ) 
{
  return f.foo + b.bar; 
}
\end{verbatim}
\end{scriptsize}

niekoniecznie łatwo wyczuć czy funkcja {\tt DoSomething} bardziej należy do {\tt Foo} czy do {\tt Bar} i do której struktury
przywiązać tę funkcję dokonując konwersji kodu strukturalnego do wymogów konwencji obiektowej.

Nie wgłębiając się bardzo w ten problem, powiedzmy tylko że nie ma takich reguł projektowania obiektowego, które rozstrzygałyby
tego rodzaju wątpliwości obiektywnie i deterministycznie. W takich przypadkach jak ten, przypisanie funkcji do jednej tych 
z kilku klas, które są argumentami funkcji, bywa zależne od "stylu" programisty i dla kodu

\begin{scriptsize}
\begin{verbatim}
struct Ser
{
}

struct Mysz
{
}

void MyszZjadaSer( Mysz mysz, Ser ser )
{
}
\end{verbatim}
\end{scriptsize}

następujące dwa podejścia są równoważne
(jeśli nie ma żadnych innych dodatkowych przesłanek za którymkolwiek z nich):

\begin{scriptsize}
\begin{verbatim}
class Ser
{
}

class Mysz
{
   void Zjada( Ser ser );  
}
\end{verbatim}
\end{scriptsize}

oraz

\begin{scriptsize}
\begin{verbatim}
class Mysz
{
  
}

class Ser
{
  void JestZjadanyPrzez( Mysz mysz );
}
\end{verbatim}
\end{scriptsize}

Kolejna oczywiste ograniczenie paradygmatu obiektowego pojawia się w językach {\em ściśle} obiektowych, 
takich w których {\bf każda funkcja musi} być przypisana do {\bf jakiejś} klasy. Tak jest na przykład w Javie czy C\#,
z kolei C++ nie nakłada takich ograniczeń. U niedoświadczonych programistów to wymaganie rodzi pewne wątpliwości,
no bo jakże to taką prostą metodę przypisać do jakiejś struktury:

\begin{scriptsize}
\begin{verbatim}
int Sum( int x, int y )
{
  return x + y;
}
\end{verbatim}
\end{scriptsize}

?

Choć wydaje się, że kandydatem na bycie "właścicielem" takiego algorytmu ma szansę być struktura (klasa) {\tt int},
w wielu językach zwyczajnie nie ma możliwości rozszerzania klas biblioteki standardowej o własne, niestandardowe metody.
W takich przypadkach zwykle stosuje się regułę {\bf Pure Fabrication}, jedną z formalnych reguł projektowania obiektowego 
GRASP, która mówi tyle że w przypadku algorytmów które nie przynależą w naturalny sposób do żadnej struktury należy
po prostu utworzyć dla nich osobną klasę w taki sposób, żeby nie naruszyć dwóch innych zasad GRASP, Low Coupling i 
High Cohesion. 

W tym konkretnym przypadku mogłoby to być na przykład:

\begin{scriptsize}
\begin{verbatim}
class Math
{
  static int Sum( int x, int y )
  {
    return x + y;
  }
}
\end{verbatim}
\end{scriptsize}

gdzie {\tt static} dodatkowo podkreśla brak zależności od instancji (wystąpienia) danej klasy.