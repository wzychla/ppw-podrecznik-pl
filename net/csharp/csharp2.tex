\subsection{Struktura kodu, operatory}

Kod C\#-owy najbardziej przypomina kod Javy. Wszystkie podstawowe konstrukcje językowe takie jak
deklaracje zmiennych, operatory, instrukcje warunkowe czy pętle działają dokładnie tak jak w Javie.
Dzięki temu programiści znający C, C++ czy Javę bardzo szybko odnajdą się w nowym języku.

Przykład:

\begin{scriptsize}
\begin{verbatim}
/* Wiktor Zychla, 2003 */
using System;

namespace Example
{
  public class CMain
  {
    public static void Main()
    {
      int i, j=0, n;

      Console.Write( "Podaj liczbe naturalna: " );
      n = int.Parse( Console.ReadLine() );

      for ( i=1; i<n; i++ )
        j+=i;   

      Console.WriteLine( "Suma wynosi " + j.ToString() );
    }
  }
}
\end{verbatim}
\end{scriptsize}

Kompilator dość restrykcyjnie traktuje powszechnie popełniane przez programistów pomyłki, zwykle
sygnalizując błąd tam, gdzie kompilator C czy C++ poprzestaje na ostrzeżeniu. Na przykład w powyższym
przykładzie deklaracja

\begin{scriptsize}
\begin{verbatim}
int i, j, n; // brak przypisania j=0
\end{verbatim}
\end{scriptsize}

spowoduje błąd kompilacji

\begin{scriptsize}
\begin{verbatim}
Microsoft (R) Visual C# .NET Compiler version 7.00.9466
for Microsoft (R) .NET Framework version 1.0.3705
Copyright (C) Microsoft Corporation 2001. All rights reserved.

example.cs(17,9): error CS0165: Use of unassigned local variable 'j'
\end{verbatim}
\end{scriptsize}

W przeciwieństwie do C, kompilator C\# nie pozwala redefiniować zmiennych na kolejnych poziomach
zagnieżdżenia kodu, uznano bowiem że jest to źródłem zbyt dużej ilości niezamierzonych pomyłek.

\begin{scriptsize}
\begin{verbatim}
/* Wiktor Zychla, 2003 */
using System;

class Example
{
  public static void Main()
  {
    bool b=true;
    int  i;

    while ( b )
    {
      int i;
    }
  }
}

example.cs(12,8): error CS0136: A local variable named 'i' cannot be declared in
        this scope because it would give a different meaning to 'i', which is
        already used in a 'parent or current' scope to denote something else
\end{verbatim}
\end{scriptsize}

Niespodziewanie, lecz konsekwentnie, taka konstrukcja nie jest możliwa również wtedy, gdy
deklaracja bardziej zagnieżdżona {\em poprzedza} deklarację mniej zagnieżdżoną.

\begin{scriptsize}
\begin{verbatim}
/* Wiktor Zychla, 2003 */
using System;

class Example
{
  public static void Main()
  {
    bool b=true;
    while ( b )
    {
      int i;
    }
    int  i;
  }
}
\end{verbatim}
\end{scriptsize}

\subsection{System typów, model obiektowy}

W obiektowych językach programowania zwykle funkcjonują dwie rozłączne klasy bytów, na których można
operować: typy proste (liczby całkowite, zmiennoprzecinkowe, napisy) oraz typy złożone (klasy). Istnienie
dwóch rozłącznych światów rodzi mnóstwo problemów i niejednokrotnie zmusza programistę do pisania
"brzydkiego" kodu. Na przykład w sytuacji, kiedy potrzebna jest funkcja operująca na wartości dowolnego
typu, istnienie typów prostych zmusza programistę do przeciążania tej funkcji tyle razy ile różnych
jej wariantów będzie potrzebował. W C++ pewnym sposobem na przezwyciężanie takich trudności są 
szablony, jednak nie istnieje sposób na stosowanie szablonu do typu nieznanego kompilatorowi podczas kompilacji.
Taka cecha języka sprawia, że trudno nazwać C++ językiem w pełni obiektowym. Nawet Java dzieli typy
na proste i złożone, bowiem stosowanie typów prostych znacząco poprawia wydajność kodu.

Model obiektowy C\# to model z pojedyńczym dziedziczeniem\footnote{Sposobem na pokonanie
ograniczeń pojedyńczego dziedziczenia są tzw. {\em interfejsy}.}. Zakłada się istnienie jednej 
wspólnej klasy {\em object} dla wszystkich obiektów. Choć nadal istnieje podział na typy proste i typy
złożone, to z punktu widzenia systemu typów wszystko jest obiektem, co więcej typ obiektu jest możliwy do
odzyskania w trakcie działania programu.

Dzięki takiej konstrukcji systemu typów możliwe jest zdefiniowanie pewnej funkcjonalności już na poziomie
klasy {\em object}. Ta funkcjonalność jest dziedziczona na klasy potomne. Najważniejsze dwie metody
wirtualne zdefiniowane w klasie {\em object} to:

\begin{description}
\item[string ToString()] Domyślnie ta metoda zwraca nazwę typu obiektu. Przeciążona może zwracać opis
zawartości obiektu w postaci przyjaznej dla użytkownika.
\item[Type GetType()] Zwraca zmienną typu {\em Type}, która określa typ obiektu. 
\end{description}

\begin{scriptsize}
\begin{verbatim}
/* Wiktor Zychla, 2003 */
using System;

namespace Example
{
  public class Klasa1 {}
  public class Klasa2 
  {
    public override string ToString()
    {
      return "Jestem obiektem klasy Klasa2";
    } 
  }

  public class CMain
  {
    public static void Main()
    {
      Klasa1 k1 = new Klasa1();
      Klasa2 k2 = new Klasa2();
      
      Console.WriteLine( k1.ToString() );
      Console.WriteLine( k2.ToString() );

      Console.WriteLine( k1.GetType().ToString() );
      Console.WriteLine( k2.GetType().ToString() );

      Console.WriteLine( k1.GetType().GetType().ToString() );
    }
  }
}

C:\Example>example.exe
Example.Klasa1
Jestem obiektem klasy Klasa2
Example.Klasa1
Example.Klasa2
System.RuntimeType
\end{verbatim}
\end{scriptsize}

Jak widać wartość wyrażenia
\begin{scriptsize}
\begin{verbatim}
object_value.GetType()
\end{verbatim}
\end{scriptsize}

sama jest wartością typu {\em Type}, można więc zapytać o jej typ

\begin{scriptsize}
\begin{verbatim}
object_value.GetType().GetType()
\end{verbatim}
\end{scriptsize}

i poprosić o jego reprezentację 

\begin{scriptsize}
\begin{verbatim}
object_value.GetType().GetType().ToString()
\end{verbatim}
\end{scriptsize}

Zalety zunifikowanego systemu typów ({\em CTS, Common Type System}) to jednak nie tylko elegancja języka.
CTS gra główną rolę przy łączeniu przez środowisko uruchomieniowe kodu napisanego w różnych językach. 
Każdy kompilator musi umieć posługiwać się zdefiniowanymi w CTS typami prostymi, musi także 
definiować własne typy wpasowując je w określoną przez CTS hierarchię typów. Sam CTS jest częścią
szerszej specyfikacji zwanej CLS ({\em Common Language Specification}), która dodatkowo określa 
inne istotne wymagania dla kompilatora (takie jak sposób zarządzania pamięcią, obsługi wyjątków). 

Największą zaletą CTS jest jednak bezpieczeństwo jakie oferuje tak zaprojektowany system typów.
\begin{itemize}
\item{Typ każdego obiektu może być jednoznacznie określony, nawet dynamicznie, czyli w trakcie działania programu.}
\item{Nie ma możliwości oszukania systemu typów przez próbę przekonania go że jakiś obiekt ma inny typ
niż jego prawdziwy typ.}
\item Dostęp do składowych każdego obiektu (publiczny, prywatny) jest określony na poziomie definicji klasy i 
nie jest możliwe działanie wbrew określonym prawom dostępu. To znaczy, że na przykład jeśli składowa klasy 
jest prywatna, to system typów i środowisko uruchomieniowe chronią ją przed dostępem z 
zewnątrz\footnote{W C++ kwalifikatory dostępu to tak naprawdę mechanizm którym programista chroni się
sam przed sobą. Można bowiem zrzutować obiekt klasy z polami prywatnymi na inną klasę z polami publicznymi, w
ten sposób obchodząc mechanizm kwalifikatorów.}.  
\end{itemize}

\subsection{Typy proste a typy referencyjne, boxing i unboxing}

System typów, w którym każdy byt jest obiektem nie jest pomysłem nowym. Tak jest na przykład
w SmallTalku. Niestety, SmallTalk płaci za tę cechę języka cenę efektywności - tam gdzie mamy
do czynienia na przykład z liczbami całkowitymi czy zmiennoprzecinkowymi (na których procesor
potrafi przecież dokonywać szybkich obliczeń) konieczność opakowywania ich w struktury obiektowe
dramatycznie zmniejsza wydajność.

Projektanci języka C\# rozwiązali ten problem dzieląc świat wszystkich obiektów na dwie kategorie:
obiekty proste i obiekty referencyjne. Obiekty proste są tworzone na stosie i reprezentowane są
przez aktualną wartość obiektu. Obiekty proste nie mogą więc mieć wartości {\em null}, która
związania jest z pustym wskazaniem - one zawsze mają wartość. 
Obiekty referencyjne tworzone są na stercie, zaś na stosie znajduje
się referencja do obiektu na stercie. 

\begin{center}
$$
\begin{array}{rlll}
           & \mbox{Stos} & & \mbox{Sterta} \\
             \hline            \\
\mbox{int} & \framebox[3cm]{17} & & \\
\mbox{string} & \framebox[3cm]{adres} & \longleftrightarrow & \framebox[3cm]{"Ala ma kota"} \\

\end{array}
$$
\end{center}

Obiektami prostymi są w C\# na przykład arytmetyczne typy wbudowane, enumeracje i typy zdefiniowane
jako struktury (patrz \ref{subsubsection:klasy}). Obiektami referencyjnymi są pozostałe klasy, tablice,
delegaci, interfejsy.

Obiekt typu prostego jest inicjowany bezpośrednio po zadeklarowaniu. Oznacza to, że nie ma
potrzeby jawnego wywołania konstruktora, na przykład:

\begin{scriptsize}
\begin{verbatim}
int i;

for ( i=0; i<20; i++ )
  ...
\end{verbatim}
\end{scriptsize}

Programista może przekształcać obiekty o typach prostych do postaci referencyjnej za pomocą
tzw. opakowywania (ang. {\em boxing}), a następnie z powrotem do postaci prostem (odpakowywanie, {\em unboxing}).

\begin{scriptsize}
\begin{verbatim}
int    i = 1;
object o = i;
int    newi = (int)o;
\end{verbatim}
\end{scriptsize}

\begin{center}
$$
\begin{array}{rlll}
            & \mbox{Stos} & & \mbox{Sterta} \\
\hline                                      \\
\mbox{i}    & \framebox[3cm]{1} & & \\
\mbox{o}    & \framebox[3cm]{adres} & \longleftrightarrow & \framebox[3cm]{1} \\
\mbox{newi} & \framebox[3cm]{1} & & \\

\end{array}
$$
\end{center}
