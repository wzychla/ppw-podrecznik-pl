\subsection{Klasy}
\label{subsubsection:klasy}

Pojęcie klasy jest fundamentalnym pojęciem programowania obiektowego. Pojedyńcza klasa opisuje cechy 
jakiegoś konkretnego obiektu - jego właściwości i możliwe akcje. 

Najprostsza definicja klasy w C\# mogłaby wyglądać tak:

\begin{scriptsize}
\begin{verbatim}
class cOsoba
{
  public int wiek; 
}
\end{verbatim}
\end{scriptsize}

Taka definicja pozwala konstruować obiekty opisanego typu i odwoływać się do jedynego {\em 
pola} obiektów tej klasy:

\begin{scriptsize}
\begin{verbatim}
/* Wiktor Zychla, 2003 */
using System;

namespace Example
{
  class cOsoba 
  {
    public int wiek; 
  } 

  public class CMain
  {
    public static void Main()
    {
      cOsoba o = new cOsoba();
      o.wiek = 13;

      Console.WriteLine( "wiek osoby to " + o.wiek.ToString() + " lat" );
    }
  }
}
\end{verbatim}
\end{scriptsize}

Ponieważ w C\# nie ma znanych z C i C++ plików nagłówkowych, w których umieszczało się deklaracje funkcji,
cała definicja klasy musi znajdować się w jednym pliku. Nie ma również potrzeby pisania deklaracji
wyprzedzających - klasa może być używana w dowolnym miejscu kodu bez względu na miejsce jej definicji, na przykład:

\begin{scriptsize}
\begin{verbatim}
/* Wiktor Zychla, 2003 */
using System;

class Example
{
  class A
  {
    B b;
  }
  class B
  {
    A a;
  }
  public static void Main()
  {
  }
}
\end{verbatim}
\end{scriptsize}

Definicja klasy składa się z definicji {\em elementów składowych} 
określających jej funkcjonalność. W C\# istnieje 7 możliwych
rodzajów elementów składowych:

\begin{description}

\item [pola] Pole jest elementem składowym klasy, który przechowuje jakąś wartość.
\item [metody] Metoda jest funkcją, która najczęściej w jakiś sposób operuje na wartościach przechowywanych
w polach.
\item [właściwości (propercje)] Propercje są metodami, które z punktu widzenia klientów klasy wyglądają jak pola.
\item [stałe] Stałe są polami, których wartość nie może ulegać zmianom.
\item [indeksery] Indeksery są konstrukcjami językowymi, które pozwalają na dostęp do danych klasy tak, jakby
były one umieszczone w tablicy, choć wewnętrzna reprezentacja może być zupełnie inna.
\item [zdarzenia] Zdarzenia powodują wykonywanie się jakiegoś kodu. Zdarzenia mają swoje listy słuchaczy, a
zaistnienie zdarzenia powoduje wykonanie wszystkich funkcji na liście słuchaczy.
\item [operatory] W C\# istnieje możliwość przeciążania kilku standardowych operatorów.

\end{description}

Każdy element składowy (z drobnymi wyjątkami) może być opatrzony odpowiednim {\em kwalifikatorem dostępu}.

\begin{description}

\item [public] Brak ograniczeń w dostępie do składowej.
\item [protected] Dostęp jest ograniczony do składowych danej klasy i klas potomnych.
\item [internal] Dostęp jest ograniczony do bieżącego modułu.
\item [protected internal] Dostęp jest ograniczony do składowych danej klasy i klas potomnych bieżącego modułu.
\item [private] Dostęp jest ograniczony do składowych danej klasy.

\end{description}

W przeciwieństwie do C++ każda składowa klasy musi być jawnie opatrzona odpowiednim kwalifikatorem dostępu, 
zaś jego brak oznacza domyślnie kwalifikator {\em private}, na przykład:

\begin{scriptsize}
\begin{verbatim}
/* Wiktor Zychla, 2003 */
using System;

class Example
{
  class A
  {
    public string s;
    int i;
  }
  public static void Main()
  {
    A a = new A();
    a.s = "Ala ma kota";
    a.i = 5; // błąd
  }
}

example.cs(14,2): error CS0122: 'Example.A.i' is inaccessible due to its
        protection level
\end{verbatim}
\end{scriptsize}

\subsubsection{Pola}

Projektując obiekty dla swojej aplikacji, programista zwykle stoi przed zadaniem zbudowania zbioru
klas tak, aby jak najlepiej opisać problem, który rozwiązywać ma aplikacja. Stąd naturalne są konstrukcje,
w których polami klasy opisującej osobę byłyby jej atrybuty takie jak imię, nazwisko, data urodzenia itp.,
klasa opisująca pozycję w bibliotece mogłaby zawierać pola opisujące rodzaj pozycji, jej tytuł, autora i datę
wydania itp.

\begin{scriptsize}
\begin{verbatim}
class COsoba
{
  public string Imie;
  public string Nazwisko;
  public DateTime data_urodzenia;
}
\end{verbatim}
\end{scriptsize}

Przypomnijmy sobie, że w C++ istnieją dwie możliwości utworzenia obiektu:
\begin{scriptsize}
\begin{verbatim}
COsoba  osoba1;
COsoba* osoba2 = new COsoba();
...
\end{verbatim}
\end{scriptsize}

W C\# klasa opisana tak jak wyżej będzie typem referencyjnym, to znaczy że użycie obiektu będzie 
wymagało jego jawnego utworzenia:

\begin{scriptsize}
\begin{verbatim}
...
COsoba osoba = new COsoba();
...
\end{verbatim}
\end{scriptsize}

Sam term {\em osoba} funkcjonuje w CTS jako referencja do obiektu na stercie programu. Odwołania do jego
składowych odbywają się za pomocą operatora ".", na przykład

\begin{scriptsize}
\begin{verbatim}
...
osoba.Imie = "Xawery";
...
\end{verbatim}
\end{scriptsize}

Programista może w klasie zdefiniować również pola {\em statyczne}. Mają one własność przynależenia do klasy,
a nie do żadnego konkretnego obiektu klasy. Intuicyjnie można więc rozumieć pola statyczne jako 
odpowiednik zmiennych globalnych, występujących w innych językach programowania.

\begin{scriptsize}
\begin{verbatim}
/* Wiktor Zychla, 2003 */
using System;

namespace Example
{
  class COsoba 
  {
    public static int IloscOsob; 
  } 

  public class CMain
  {
    public static void Main()
    {
      COsoba.IloscOsob = 17;
      Console.WriteLine( COsoba.IloscOsob.ToString() );
    }
  }
}
\end{verbatim}
\end{scriptsize}

\subsubsection{Metody}

Metody zawierają w sobie kod wykonywany podczas działania programu. Na przykład:

\begin{scriptsize}
\begin{verbatim}
/* Wiktor Zychla, 2003 */
using System;

namespace Example
{
  class COsoba 
  {
    public string Imie;
    public string Nazwisko;

    public string ImieNazwisko()
    {
      return Imie+" "+Nazwisko;
    }
  } 

  public class CMain
  {
    public static void Main()
    {
      COsoba o   = new COsoba();

      o.Imie     = "Xawery";
      o.Nazwisko = "Xawerowski";

      Console.WriteLine( o.ImieNazwisko() );
    }
  }
}
\end{verbatim}
\end{scriptsize}

Specjalne znaczenie mają metody statyczne, które podobnie jak pola statyczne nie są przypisane do
konkretnej instancji obiektu danej klasy, tylko do klasy jako takiej. Podobnie jak pola, metody statyczne
są intuicyjnymi odpowiednikami funkcji globalnych z C czy C++.

\begin{scriptsize}
\begin{verbatim}
/* Wiktor Zychla, 2003 */
using System;

namespace Example
{
  class CInfo
  {
    public static string GetInfo()
    {
      return "Info";
    }
  } 

  public class CMain
  {
    public static void Main()
    {
      Console.WriteLine( CInfo.GetInfo() );
    }
  }
}
\end{verbatim}
\end{scriptsize}

W standardowy sposób {\em przeciąża się} metody, tak aby akceptowały różne listy wywołania:

\begin{scriptsize}
\begin{verbatim}
/* Wiktor Zychla, 2003 */
using System;

namespace Example
{
  public class CMain
  {    
    static void Metoda( int i, string s )
    {
	  Console.WriteLine( String.Format( "Liczba '{0}', napis '{1}' ", i, s ) );
    }
    static void Metoda( int i )
    {
      Metoda( i, "jakis napis" );
    }
    static void Metoda( string s )
    {
      Metoda( 17, s );
    }

    public static void Main()
    {
      Metoda( 5, "Ala ma kota" );
      Metoda( 13 );
      Metoda( "kot ma Ale" );
    }
  }
}

C:\Example>example.exe
Liczba '5', napis 'Ala ma kota'
Liczba '13', napis 'jakis napis'
Liczba '17', napis 'kot ma Ale'
\end{verbatim}
\end{scriptsize}

Istnieje również możliwość poinformowania kompilatora o tym, że metoda może być wołana z nieznaną
w czasie kompilacji liczbą parametrów:

\begin{scriptsize}
\begin{verbatim}
/* Wiktor Zychla, 2003 */
using System;

namespace Example
{
  public class CMain
  {
    static void VariableParList( params int[] iInfo )
    {
      Console.Write( "parametry: " );
      for ( int i=0; i<iInfo.GetLength(0); i++ )
        Console.Write( iInfo[i].ToString()+"," );
      Console.WriteLine();
    }

    public static void Main()
    {
      VariableParList( 1 );
      VariableParList( 1, 2 );
      VariableParList( 1, 2, 3, 4, 5 );
    }
  }
}
\end{verbatim}
\end{scriptsize}

Kompilator nie pozwoli jednak na skompilowanie kodu, w którym ze względu na przeciążenie
funkcji intencje programisty są niejasne.

\begin{scriptsize}
\begin{verbatim}
/* Wiktor Zychla, 2003 */
using System;

namespace Example
{
  class CMain 
  {
    static void A( int a, params int[] tab )
    {
      Console.WriteLine( "A1" ); 
    }
  
    static void A( int a, int b, params int[] tab )
    {
      Console.WriteLine( "A2" ); 
    }

    public static void Main()
    {
      A( 1, 2, 3 );
    }
  }
}
\end{verbatim}
\end{scriptsize}

Czy wołając funkcję {\bf A} programista miał na myśli wersję pierwszą, z jednym parametrem jawnym, czy drugą,
z dwoma parametrami jawnymi? Cokolwiek myślał programista, kompilator ma własne zdanie na temat takiego
kodu:

\begin{scriptsize}
\begin{verbatim}
example.cs(19,8): error CS0121: The call is ambiguous between the following methods
        or properties: 'Example.CMain.A(int, int, params int[])' and
        'Example.CMain.A(int, params int[])'
\end{verbatim}
\end{scriptsize}

Specjalną rolę wśród metod w klasie pełni metoda Main(), która określa punkt startowy 
aplikacji\footnote{O możliwości
umieszczenia wielu alternatywnych metod Main() w kodzie aplikacji napisano więcej na stronie 
\pageref{main_wielerazy}.}. Parametry startowe programu są przekazane jako tablica do metody Main():

\begin{scriptsize}
\begin{verbatim}
/* Wiktor Zychla, 2003 */
using System;

namespace Example
{
  public class CMain
  {
    public static void Main( string[] args )
    {
      Console.WriteLine( "Oto kolejne argumenty wywolania programu: " );
      foreach ( string s in args )
        Console.WriteLine( s );
    }
  }
}

C:\Example>example.exe 17 "napis napis"
Oto kolejne argumenty wywolania programu:
17
napis napis
\end{verbatim}
\end{scriptsize}

\subsubsection{Przekazywanie parametrów do metod}

Sposób przekazania parametru do metody zależy od tego, czy zmienna jest typu prostego czy typu
referencyjnego. Jeśli zmienna jest typu prostego, jak {\em int}, to do metody zostanie przekazana wartość,
jeśli zmienna jest typu referencyjnego, to do metody zostanie przekazana referencja.

Oznacza to, że wołana metoda nie ma możliwości zmiany, w przypadku typów prostych - wartości zmiennej,
w przypadku typów referencyjnych - referencji do zmiennej. 

\begin{scriptsize}
\begin{verbatim}
/* Wiktor Zychla, 2003 */
using System;

namespace Example
{
  public class CMain
  {    
    public static void Zmien( int i, string s )
    {
      i = 1;
      s = "Ala ma kota";
    }
    public static void Main()
    {
      int    i = 0;
      string s = "Kot ma Ale";

      Console.WriteLine( "{0}, {1}", i, s );
      Zmien( i, s );
      Console.WriteLine( "{0}, {1}", i, s );
    }
  }
}

C:\Example>example.exe
0, Kot ma Ale
0, Kot ma Ale
\end{verbatim}
\end{scriptsize}

To, że referencja do przekazywanego obiektu nie może ulegać zmianie nie oznacza, 
że wartość obiektu referencyjnego nie może być zmodyfikowana - wprost przeciwnie,
metoda ma możliwość zmiany właściwości obiektu przekazanego przez referencję.

\begin{scriptsize}
\begin{verbatim}
/* Wiktor Zychla, 2003 */
using System;
using System.Collections;

namespace Example
{
  public class CMain
  {    
    public static void Zmien( ArrayList a )
    {
      a.Add( 0 );
    }
    public static void Main()
    {
      ArrayList a = new ArrayList();
      Console.WriteLine( "elementow na liscie {0}", a.Count );
      Zmien( a );
      Console.WriteLine( "elementow na liscie {0}", a.Count );      
    }
  }
}

C:\Example>example.exe
elementow na liscie 0
elementow na liscie 1
\end{verbatim}
\end{scriptsize}

Jeśli intencją programisty jest zmiana wartości przekazywanej do funkcji, może zażądać przekazania do
funkcji referencji do obiektu (w przypadku typu referencyjnego będzie to referencja do referencji) za
pomocą słowa kluczowego {\em ref}. Jest to dosłowny odpowiednik przekazywania parametrów do funkcji przez
referencje, znany z C++.

\begin{scriptsize}
\begin{verbatim}
/* Wiktor Zychla, 2003 */
using System;

namespace Example
{
  public class CMain
  {    
    public static void Zmien( ref int i, ref string s )
    {
      i = 1;
      s = "Ala ma kota";
    }
    public static void Main()
    {
      int    i = 0;
      string s = "Kot ma Ale";

      Console.WriteLine( "{0}, {1}", i, s );
      Zmien( ref i, ref s );
      Console.WriteLine( "{0}, {1}", i, s );
    }
  }
}

C:\Example>example.exe
0, Kot ma Ale
1, Ala ma kota
\end{verbatim}
\end{scriptsize}

Wydawać by się mogło, że w taki sposób należy również przekazywać parametry, które miałyby służyć do
przekazywania wyników do funkcji. Naiwnie możnaby więc spróbować napisać coś takiego:

\begin{scriptsize}
\begin{verbatim}
/* Wiktor Zychla, 2003 */
using System;

namespace Example
{
  public class CMain
  {    
    public static void Oblicz( ref int wynik )
    {
      wynik = 1;
    }
    public static void Main()
    {
      int    wynik;

      Oblicz( ref wynik );
      Console.WriteLine( "wynik: {0}", wynik );
    }
  }
}
\end{verbatim}
\end{scriptsize}

jednak kompilator takiej konstrukcji nie przyjmie

\begin{scriptsize}
\begin{verbatim}
example.cs(16,19): error CS0165: Use of unassigned local variable 'wynik'
\end{verbatim}
\end{scriptsize}

Istnieją dwa możliwe rozwiązania takiego problemu:
\begin{itemize}
\item Przed wykonaniem obliczeń przypisać jakąś wartość zmiennej {\em wynik}. 
Nie jest to rozwiązanie eleganckie, skoro {\em wynik} ma dopiero otrzymać wartość w wyniku obliczeń.
\item Zadeklarować parametr funkcji jako {\em out int wynik} zamiast {\em ref int wynik}. Słowo
kluczowe {\em out} jest równoważne {\em ref}, przy czym zmienna nie musi otrzymać wartości przed
wykorzystaniem jej jako parametru do funkcji.
\end{itemize}

\begin{scriptsize}
\begin{verbatim}
/* Wiktor Zychla, 2003 */
using System;

namespace Example
{
  public class CMain
  {    
    public static void Oblicz( out int wynik )
    {
      wynik = 1;
    }
    public static void Main()
    {
      int    wynik;

      Oblicz( out wynik );
      Console.WriteLine( "wynik: {0}", wynik );
    }
  }
}
\end{verbatim}
\end{scriptsize}

\subsubsection{Konstruktory}

Konstruktory są w pewnym sensie specjalnymi metodami, które zawierają kod inicjujący obiekty. 
Konstruktory definiuje się dokładnie tak samo jak w C++ czy w Javie: przez utworzenie
kodu pseudo-metody o nazwie takiej jak nazwa klasy. 
Podstawowa różnica między C\# a na przykład C++ jest taka, że, podobnie jak w Javie, środowisko uruchomieniowe
za pomocą {\em odśmiecacza} zajmuje się oczyszczaniem pamięci z nieużywanych już obiektów\footnote{Istnieje
specjalny interfejs {\em IDisposable} przygotowany na użytek klas, które potrzebują jawnie wykonać akcje przy
niszczeniu obiektu przez odśmiecacz.}.

Konstruktory można oczywiście przeładowywać, można również korzystać z konstruktorów klas bazowych lub
z innych konstruktorów już określonych w klasie za pomocą wyrażeń inicjujących {\bf base(...)} oraz
{\bf this(...)}, na przykład:

\begin{scriptsize}
\begin{verbatim}
/* Wiktor Zychla, 2003 */
using System;
using System.Threading;

namespace Example
{
  public class CExample
  {
    DateTime d;
    int i=0, j=0;

    public CExample()
    {
      d = DateTime.Now;
    }

    public CExample( int I, int J ) : this()
    {
      this.i = I; 
      this.j = J;
    }

    public override string ToString()
    {
      return String.Format( "[{0},{1}], utworzone {2}", i, j, d );
    }
  }
  public class CMain
  {    
    public static void Main()
    {
      CExample e1 = new CExample();
      Thread.Sleep( 1000 );
      CExample e2 = new CExample( 13, 17 );

      Console.WriteLine( e1 );
      Console.WriteLine( e2 );
    }
  }
}

C:\Example>example.exe
[0,0], utworzone 2003-03-18 21:48:08
[13,17], utworzone 2003-03-18 21:48:10
\end{verbatim}
\end{scriptsize}

C\# pozwala zdefiniować konstruktor statyczny, który będzie wywołany przed skonstruowaniem
pierwszego obiektu klasy. Statyczny konstruktor może być tylko jeden, bez żadnego kwalifikatora dostępu i
z pustą listą parametrów.

\begin{scriptsize}
\begin{verbatim}
class CExample
{
  static CExample()
  {
    ...
  }
}
\end{verbatim}
\end{scriptsize}

\subsubsection{Propercje}

Propercje pozwalają ukryć implementację metody tak, aby z punktu widzenia klienta klasy 
wyglądała ona jak pole. Propercje stosuje się tam, gdzie istnieje konieczność nadania
lub pobrania wartości pola, a przy tym wykonać jakieś dodatkowe operacje. Za pomocą propercji można
także ograniczyć dostęp do jakiegoś pola, czyniąc je tylko do odczytu lub tylko do zapisu.

\begin{scriptsize}
\begin{verbatim}
/* Wiktor Zychla, 2003 */
using System;

namespace Example
{
  public class COsoba
  {
    private int      m_wiek;
    private DateTime m_dataUrodzenia;
	
    public int wiek
    {
      get { return m_wiek; }
    }

    public DateTime dataUrodzenia
    {
      get { return m_dataUrodzenia; }
      set 
      {
        m_dataUrodzenia = value;
        m_wiek = DateTime.Now.Year-m_dataUrodzenia.Year;
      }
    }
  }

  public class CMain
  {    
    public static void Main()
    {
      COsoba o = new COsoba();
      o.dataUrodzenia = new DateTime( 1950, 3, 8 );

      Console.WriteLine( "Osoba:\r\nUrodzona\t{0:d}\r\nWiek\t\t{1}", 
                         o.dataUrodzenia, o.wiek );
    }
  }
}

C:\Example>example.exe
Osoba:
Urodzona        1950-03-08
Wiek            53
\end{verbatim}
\end{scriptsize}

\subsubsection{Stałe}

Stałe można zadeklarować w klasie przez opatrzenie deklaracji pola kwalifikatorem {\em const}.

\begin{scriptsize}
\begin{verbatim}
public const string sKraj = "Polska";
\end{verbatim}
\end{scriptsize}

Co jednak zrobić, gdy wartość stałej jest znana dopiero {\bf po} uruchomieniu programu? W C\# taki problem
rozwiązuje kwalifikator {\em readonly}, który oznacza pole zawierające stałą, przy czym wartość takiego
pola można modyfikować {\bf tylko} w konstruktorze klasy.

\begin{scriptsize}
\begin{verbatim}
/* Wiktor Zychla, 2003 */
using System;

namespace Example
{
  public class COsoba
  {
    public readonly DateTime dataI;
    public COsoba()
    {
      dataI = DateTime.Now;
    }
  }
  public class CMain
  {    
    public static void Main()
    {
      COsoba o = new COsoba();
      Console.WriteLine( o.dataI );
    }
  }
}
\end{verbatim}
\end{scriptsize}

\subsubsection{Indeksery}

Indeksery pozwalają klientom klasy traktować obiekt tak, jakby był on tablicą, bez względu na 
reprezentację pól obiektu. Indeksery podobne są trochę do propercji - podobnie jak propercje indeksery
mogą pobierać wartość {\em get} lub sposób ustalać wartość {\em set}.

\begin{scriptsize}
\begin{verbatim}
/* Wiktor Zychla, 2003 */
using System;

namespace Example
{
  public class COsoba
  {
    public int this[int i, int j]
    {
      get { return i+j; }
    }
  }
  public class CMain
  {    
    public static void Main()
    {
      COsoba o = new COsoba();
      Console.WriteLine( o[5,17] );
    }
  }
}
\end{verbatim}
\end{scriptsize}

Idea indekserów narodziła się z chęci ułatwienia programistom dostępu do składowych obiektu, który
w jakiś sposób opisuje strukturę tablicopodobną. Na przykład klasy opisujące okna potomne, takie jak
ComboBox czy ListView, z pewnością jako jedno z pól będą zawierać jakąś tablicę elementów przechowywanych
w wewnętrznej liście obiektu. Indekser umożliwia w takim przypadku dostęp do takiej listy bezpośrednio
przez indeksowanie obiektu, a nie jego pola, tzn. na przykład zamiast

\begin{scriptsize}
\begin{verbatim}
ComboBox comboBox = new ComboBox();
...
comboBox.Items[5] = ...
\end{verbatim}
\end{scriptsize}

moglibyśmy pisać (mając zdefiniowany odpowiedni indekser)

\begin{scriptsize}
\begin{verbatim}
ComboBox comboBox = new ComboBox();
...
comboBox[5] = ...
\end{verbatim}
\end{scriptsize}

\subsubsection{Przeciążanie operatorów}

Przeciążanie operatorów nie wnosi do języków programowania nic, poza czystością i elegancją kodu. Z
technicznego punktu widzenia przeciążone operatory są jakimiś metodami, które biorą określoną ilość 
parametrów i zwracają wyniki. 

C\# pozwala przeciążać operatory za pomocą składni

\begin{scriptsize}
\begin{verbatim}
public static retval operatorop (object1 [, object2])
\end{verbatim}
\end{scriptsize}

\begin{scriptsize}
\begin{verbatim}
/* Wiktor Zychla, 2003 */
using System;

namespace Example
{
  public class CVec
  {
    public int x;
    public int y;

    public CVec( int X, int Y )
    {
      x = X; y = Y;    
    }

    public static CVec operator+( CVec v1, CVec v2 )
    {
      return new CVec( v1.x+v2.x, v1.y+v2.y );
    }

    public override string ToString()
    {
      return String.Format( "[{0},{1}]", x, y );
    }
  }
  public class CMain
  {    
    public static void Main()
    {
      CVec u = new CVec( 2, 3 );
      CVec v = new CVec( 1, 1 );
      Console.WriteLine( "{0}+{1}={2}", u, v, u+v );
    }
  }
}

C:\Example>example.exe
[2,3]+[1,1]=[3,4]
\end{verbatim}
\end{scriptsize}

Obowiązują następujące zasady przy przeciążaniu operatorów w C\#:
\begin{itemize}
\item można przeciążać operatory unarne +, -, !, ~, ++, --, {\em true} i {\em false} oraz binarne
+, -, *, /, %, &, -, ^, <<, >>, ==, !=, >, <, >= i <=
\item nie można przeciążać operatora [], można jednak zdefiniować indekser, który pozwala traktować obiekt jak tablicę
\item nie można przeciążać operatora (), z wyjątkiem definiowania własnych jawnych konwersji
\item operatory warunkowe (\&\&, --, i ?:) nie mogą być przeciążane
\item operatory nie występujące w C\# nie mogą być przeciążane
\item operatory zdefiniowane przez Framework (., =, {\em new}) nie mogą być przeciążane   
\item operatory (== i !=) mogą być przeciążane, wymaga to jednak przeciążenia metod {\em Equals} i {\em GetHashCode}
\item przeładowanie niektórych operatorów binarnych (na przykład +) powoduje automatyczne przeładowanie pewnych
innych operatorów (w tym przypadku +=)
\item operatory $<$ i $>$ muszą być przeładowywane jednocześnie
\end{itemize}

\subsection{Struktury}

Typy proste w C\# deklaruje się tak samo jak typy referencyjne, zastępując słowo kluczowe {\em class}
słowem kluczowym {\em struct}. Tak jak typy referencyjne nazywamy klasami, tak typy proste nazywamy
strukturami.

Deklaracja struktury może zawierać dowolną ilość konstruktorów, z wyjątkiem konstruktora bezparametrowego,
który jest tworzony domyślnie i powoduje wyzerowanie (nadanie wartości domyślnych) wartości wszystkich
pól struktury. W przypadku typów prostych, które nie zawierają pól, korzystanie ze zmiennych możliwe jest więc
bez wywołania konstruktora (jak na przykład w przypadku typu {\em int}).

\begin{scriptsize}
\begin{verbatim}
/* Wiktor Zychla, 2003 */
using System;

namespace Example
{
  struct RGB
  {
    public int r;
    public int g;
    public int b;
    
    public RGB( int R, int G, int B )
    {
      r = R; g = G; b = B;
    }

    public override string ToString()
    {
      return String.Format( "[R:{0}, G:{1}, B:{2}]", r, g, b );
    }
  }
  public class CMain
  {    
    public static void Main()
    {
       RGB rgb  = new RGB(); 
       RGB rgb2 = new RGB( 1, 2, 3 );
       Console.WriteLine( rgb );
       Console.WriteLine( rgb2 );
    }
  }
}

C:\Example>example.exe
[R:0, G:0, B:0]
[R:1, G:2, B:3]
\end{verbatim}
\end{scriptsize}

\subsection{Dziedziczenie}

Model obiektowy C\# udostępnia pojedyńcze dziedziczenie. Oznacza to, że każda klasa może mieć co najwyżej
jedną klasę bazową. O relacjach między klasami programista informuje kompilator za pomocą składni

\begin{scriptsize}
\begin{verbatim}
class <klasaPotomna> : <klasaBazowa>
\end{verbatim}
\end{scriptsize}

\begin{scriptsize}
\begin{verbatim}
/* Wiktor Zychla, 2003 */
using System;

namespace Example
{
  class A
  {
    public int ID;
  }
  class B : A
  {
    public int ID2;
  }
  public class CMain
  {    
    public static void Main()
    {
      A a = new A();
      B b = new B();
      
      a.ID  = 1;
      b.ID  = 2;
      b.ID2 = 3;
    }
  }
}
\end{verbatim}
\end{scriptsize}

Jeśli jakaś metoda występuje w klasie potomnej i w klasie bazowej, to kompilator zakłada, że 
metoda z klasy potomnej {\em przykrywa} definicję metody z klasy bazowej (ominięcie obowiązkowego
w takim przypadku kwalifikatora {\em new} zostanie przez kompilator wykryte i programista zostanie ostrzeżony
o jego braku).

Co się jednak stanie, jeśli obiekt klasy potomnej zostanie najpierw zrzutowany na obiekt klasy
macierzystej, a następnie zostanie wywołana metoda?

\begin{scriptsize}
\begin{verbatim}
/* Wiktor Zychla, 2003 */
using System;

namespace Example
{
  class A
  {
    public void DajGlos()
    {
      Console.WriteLine( "A" );
    }
  }
  class B : A
  {
    new public void DajGlos()
    {
      Console.WriteLine( "B" );
    }
  }
  public class CMain
  {    
    public static void Main()
    {
      A a = new A();
      B b = new B();

      a.DajGlos();
      b.DajGlos();
      ((A)b).DajGlos();
    }
  }
}

C:\Example>example.exe
A
B
A
\end{verbatim}
\end{scriptsize}

Jak widać, kompilator {\em statycznie} wyznaczył typ obiektu i wymusił zawołanie funkcji właściwej dla
wyznaczonego typu. 

Programista może jednak zażądać {\em polimorficznego} traktowania przeciążonych w klasach potomnych metod, 
to znaczy {\em dynamicznego wyznaczania} typu obiektu i wołania odpowiedniej funkcji. Służą do tego
kwalifikatory {\em virtual} i {\em override} umieszczone przy odpowiednich deklaracjach.

\begin{scriptsize}
\begin{verbatim}
/* Wiktor Zychla, 2003 */
using System;

namespace Example
{
  class A
  {
    virtual public void DajGlos()
    {
      Console.WriteLine( "A" );
    }
  }
  class B : A
  {
    override public void DajGlos()
    {
      Console.WriteLine( "B" );
    }
  }
  public class CMain
  {    
    public static void Main()
    {
      A a = new A();
      B b = new B();

      a.DajGlos();
      b.DajGlos();
      ((A)b).DajGlos();
    }
  }
}

C:\Example>example
A
B
B
\end{verbatim}
\end{scriptsize}

Programista może zabezpieczyć klasę przed dziedziczeniem z niej za pomocą kwalifikatora {\em sealed}, na przykład

\begin{scriptsize}
\begin{verbatim}
/* Wiktor Zychla, 2003 */
using System;

namespace Example
{
  sealed class A
  {
  }
  class B : A
  {
  }
  public class CMain
  {    
    public static void Main()
    {
    }
  }
}

example.cs(9,9): error CS0509: 'Example.B' : cannot inherit from sealed class
        'Example.A'
\end{verbatim}
\end{scriptsize}

\subsection{Niszczenie obiektów}

Niszczeniem obiektów w C\# zajmuje się odśmiecacz. Co pewien czas odśmiecacz przegląda stertę w poszukiwaniu
obiektów do których brak już referencji i zwalnia przydzieloną im pamięć. Programista nie ma więc kontroli
nad niszczeniem obiektów takiej jak na przykład w C++. Istnieją jednak dwa aspekty niszczenia obiektów, 
których programista musi być świadomy.

\subsubsection{Destruktory}

Zadziwiające, ale destruktory istnieją w C\#. Ich działanie zdecydowanie różni się od działania
destruktorów w C++, bardziej zaś przypomina działanie metod typu {\em Finalize} z Javy.

Chodzi o to, że programista nie ma żadnej kotroli nad tym, kiedy destruktor zostanie wywołany. 
Odśmiecacz wykona kod zawarty w destruktorze tuż przed usunięciem obiektu, jednak sam moment usuwania
obiektu jest z punktu widzenia programisty nie możliwy do określenia. 

Poniższy przykład pokazuje, że destruktory są wywoływane tuż przed zakończeniem programu. Okazuje się jednak, że
przerwanie działania programu za pomocą CTRL+C spowoduje, że destruktory nie wykonają się!


\begin{scriptsize}
\begin{verbatim}
/* Wiktor Zychla, 2003 */
using System;
using System.Threading;

namespace Example
{
  public class CExample
  {
    public int numer;

    public  CExample( int Numer )
    {
      numer = Numer;
      Console.WriteLine( "Utworzono obiekt {0}", numer );
    }

    ~CExample()
    {
      Console.WriteLine( "Zniszczono obiekt {0}", numer );
    }

    public static void Main(string[] args)
    {
      int objectNo = 10;
      
      CExample[] examples = new CExample[ objectNo ];
      for ( int i=0; i<10; i++ )
        examples[i] = new CExample( i ); 

      Thread.Sleep( 5000 );
    }
  }
}

C:\Example>example.exe
Utworzono obiekt 0
Utworzono obiekt 1
Utworzono obiekt 2
Utworzono obiekt 3
Utworzono obiekt 4
Utworzono obiekt 5
Utworzono obiekt 6
Utworzono obiekt 7
Utworzono obiekt 8
Utworzono obiekt 9
Zniszczono obiekt 9
Zniszczono obiekt 8
Zniszczono obiekt 7
Zniszczono obiekt 6
Zniszczono obiekt 5
Zniszczono obiekt 4
Zniszczono obiekt 3
Zniszczono obiekt 2
Zniszczono obiekt 1
Zniszczono obiekt 0
\end{verbatim}
\end{scriptsize}

\subsubsection{Interfejs {\em IDisposable}}

Z innego rodzaju problemem mamy do czynienia, kiedy obiekt C\# inicjuje zasoby innego 
rodzaju niż pamięć, na przykład obiekty systemowe takie jak gniazda, połączenia do baz danych, obiekty GDI.
Brak kontroli nad zwalnianiem tych zasobów (na przykład uchwytów GDI), może wręcz zakłócić pracę systemu!
Aby uniknąć takich problemów, należy zaimplementować w klasie interfejs {\em IDisposable}, który ma
jedną metodę: {\em Dispose()}, w której programista może jawnie zniszczyć zasoby przyznane obiektowi.

\begin{scriptsize}
\begin{verbatim}
/* Wiktor Zychla, 2003 */
using System;
using System.Threading;

namespace Example
{
  public class CExample : IDisposable
  {
    public int numer;

    public  CExample( int Numer )
    {
      numer = Numer;
      Console.WriteLine( "Utworzono obiekt {0}", numer );
    }

    public void Dispose()
    {
      Console.WriteLine( "Zniszczono obiekt {0}", numer );
    }

    public static void Main(string[] args)
    {
      int objectNo = 10;
      
      CExample[] examples = new CExample[ objectNo ];
      for ( int i=0; i<10; i++ )
        examples[i] = new CExample( i ); 

      Thread.Sleep( 5000 );

      for ( int i=0; i<10; i++ )
        examples[i].Dispose(); 
    }
  }
}
\end{verbatim}
\end{scriptsize}

Niestety, programista musi sam pamiętać o wywołaniu metody {\em Dispose} gdy obiekt przestaje być potrzebny.
W przypadku konstrukcji pojedyńczego obiektu, przydatny okazuje się {\em cukierek syntaktyczny}, polegający
na umieszczeniu konstrukcji obiektu w klauzuli {\em using}. W chwili zakończenia następującego po niej
bloku kodu, metoda {\em Dispose()} wołana jest automatycznie.

\label{disposeSyntaxSugar}

\begin{scriptsize}
\begin{verbatim}
/* Wiktor Zychla, 2003 */
using System;
using System.Threading;

namespace Example
{
  public class CExample : IDisposable
  {
    public int numer;

    public  CExample( int Numer )
    {
      numer = Numer;
      Console.WriteLine( "Utworzono obiekt {0}", numer );
    }

    public void Dispose()
    {
      Console.WriteLine( "Zniszczono obiekt {0}", numer );
    }

    public static void Main(string[] args)
    {
      using ( CExample example = new CExample( 0 ) )
      {
        Thread.Sleep( 5000 );
      }
    }
  }
}

C:\Example>example.exe
Utworzono obiekt 0
Zniszczono obiekt 0
\end{verbatim}
\end{scriptsize}

\subsection{Interfejsy}

Kiedy mówimy o klasach, mamy na myśli pewne właściwości. Interfejsy są odpowiednikami klas, przy czym
dotyczą one jakiejś określonej funkcjonalności. Interfejsy są odpowiednikami klas abstrakcyjnych, znanych
z innych języków programowania - kiedy klasa {\em implementuje interfejs} (czasem mówimy też {\em dziedziczy
z interfejsu}), czyli implementuje wszystkie metody interfejsu, klienci tej klasy mogą być pewni istnienia
w klasie wszystkich definiowanych przez interfejs metod.

Interfejsy są sposobem na przezwyciężenie ograniczenia pojedyńczego dziedziczenia - o ile
klasa może mieć tylko jedną klasę bazową, o tyle ta sama klasa może implementować dowolną ilość interfejsów.
Interfejs nie może zawierać pól, tylko specyfikacje metod przy czym w definicji interfejsu metody nie
mogą mieć żadnych kwalifikatorów dostępu, zaś w implementacji interfejsu w klasie implementowane metody
muszą być publiczne (inaczej nie miałoby sensu umieszczanie ich w publicznym intefejsie, który klasa
rzekomo miałaby spełniać).

\begin{scriptsize}
\begin{verbatim}
/* Wiktor Zychla, 2003 */
using System;

namespace Example
{
  interface I
  {
     void DajGlos();
  }
  class A : I
  {
     public void DajGlos()
     {
        Console.WriteLine( "A" );
     }
  }
  public class CMain
  {    
    public static void Main()
    {
      A a = new A();
      a.DajGlos();
    }
  }
}
\end{verbatim}
\end{scriptsize}

Jeśli klasa nie implementuje wszystkich wymaganych funkcji, to kompilator podczas kompilacji zgłosi błąd.

Z punktu widzenia programisty interfejs zachowuje się jak klasa, to znaczy można obiekty rzutować
na interfejs, statycznie i dynamicznie\footnote{O konwersji między typami można przeczytać na stronie
\pageref{KonwersjeMiedzyTypami}.}. Można także definiować zmienne których typem jest konkretny interfejs.

\begin{scriptsize}
\begin{verbatim}
/* Wiktor Zychla, 2003 */
using System;

namespace Example
{
  interface I
  {
     void DajGlos();
  }
  class A : I
  {
     public void DajGlos()
     {
        Console.WriteLine( "A" );
     }
  }
  public class CMain
  {    
    public static void Main()
    {
      A a = new A();
      I i = a as I;
      i.DajGlos();
    }
  }
}
\end{verbatim}
\end{scriptsize}

Programista może w czasie wykonania programu dowiedzieć się czy klasa implementuje jakiś interfejs
za pomocą słowa kluczowego {\em is}. Jest to przydatne zwłaszcza wtedy, kiedy obiekty znajdują się w jakimś
kontenerze, w którym wszystkie są zrzutowane do typu bazowego {\em object}.

\begin{scriptsize}
\begin{verbatim}
/* Wiktor Zychla, 2003 */
using System;

namespace Example
{
  interface I
  {
     void DajGlos();
  }
  class A : I
  {
     public void DajGlos()
     {
        Console.WriteLine( "A" );
     }
  }
  class B 
  {
     public void DajGlos()
     {
        Console.WriteLine( "A" );
     }
  }
  public class CMain
  {    
    public static void Main()
    {
       A a = new A();
       B b = new B();

       if ( a is I ) Console.WriteLine( "A implementuje I" );
       if ( b is I ) Console.WriteLine( "B implementuje I" );
    }
  }
}

C:\Example>example.exe
A implementuje I
\end{verbatim}
\end{scriptsize}

Interfejsy nie są domyślnie dziedziczone z klasy bazowej do klas potomnych. Jeśli programista 
zrzutuje obiekt klasy potomnej na interfejs implementowany w klasie bazowej, to nawet jeśli klasa
potomna zawiera odpowiednie metody, zostanie zawołana metoda z klasy bazowej. 

\begin{scriptsize}
\begin{verbatim}
/* Wiktor Zychla, 2003 */
using System;

namespace Example
{
  interface I
  {
     void DajGlos();
  }
  class A : I
  {
     public void DajGlos()
     {
        Console.WriteLine( "A" );
     }
  }
  class B : A
  {
     new public void DajGlos()
     {
        Console.WriteLine( "B" );
     }
  }
  public class CMain
  {    
    public static void Main()
    {
      B b = new B();
      b.DajGlos();

      ((I)b).DajGlos();
    }
  }
}

C:\Example>example.exe
B
A
\end{verbatim}
\end{scriptsize}

Jeśli klasa potomna miałaby implementować interfejs, to należy o tym poinformować kompilator.

\begin{scriptsize}
\begin{verbatim}
/* Wiktor Zychla, 2003 */
using System;

namespace Example
{
  interface I
  {
     void DajGlos();
  }
  class A : I
  {
     public void DajGlos()
     {
        Console.WriteLine( "A" );
     }
  }
  class B : A, I
  {
     new public void DajGlos()
     {
        Console.WriteLine( "B" );
     }
  }
  public class CMain
  {    
    public static void Main()
    {
      B b = new B();
      b.DajGlos();

      ((I)b).DajGlos();
    }
  }
}

C:\Example>example.exe
B
B
\end{verbatim}
\end{scriptsize}

Interfejsy mogą implementować inne interfejsy.

\begin{scriptsize}
\begin{verbatim}
interface I
{
   void I();
}
interface J
{
   void J();
}
interface IJ : I, J
{
   void IJ();
}
\end{verbatim}
\end{scriptsize}
