\subsection{Delegaty i zdarzenia}
\label{delegaciZdarzenia}

\subsubsection{Delegaty}

{\em Delegat} jest typem referencyjnym, który w elegancki sposób przechowuje wskaźnik na funkcję. 
CTS za pomocą delegatów pozwala przekazywać funkcje jako parametry do innych funkcji, kontrolując 
jednocześnie zgodność typów - kompilator nie pozwoli na utworzenie delegata z funkcji o nieodpowiednim
prototypie.

Sama deklaracja delegata przypomina deklarację typu wskaźnika na funkcję w języku C:

\begin{scriptsize}
\begin{verbatim}
typedef int(*pfPInt)(); // definicja typu wskaźnika na funkcję w C
\end{verbatim}
\end{scriptsize}

\begin{scriptsize}
\begin{verbatim}
delegate int pfPInt(); // definicja delegata - wskaźnika na funkcję w C#
\end{verbatim}
\end{scriptsize}

Podkreślmy - delegaty są obiektami, konstruuje się je więc w standardowy sposób za pomocą {\em new}.

\begin{scriptsize}
\begin{verbatim}
/* Wiktor Zychla, 2003 */
using System;
using System.Text;

namespace Example
{ 
  public class MathClass
  {
  	public static int Kwadrat( int n )
  	{
  	  return n*n;
  	}
  	
  	public static int Dwukrotnosc( int n )
  	{
  	  return n+n;
  	}
  }
	
  public class CMain
  {    
    public delegate int MathDelegate( int n );
  	
    public static int Oblicz( int n, MathDelegate m )
    {
      return m( n );
    }
  	
    public static void Main()
    {
      int n1 = Oblicz( 11, new MathDelegate( MathClass.Kwadrat ) );
      int n2 = Oblicz( 11, new MathDelegate( MathClass.Dwukrotnosc ) );
    	
      Console.WriteLine( "{0}, {1}", n1, n2 );
    }
  }
}
\end{verbatim}
\end{scriptsize}

Dowolna liczba delegatów może być złożona do jednego delegata za pomocą operatora +. 
Wykonanie takiego złożonego delegata powoduje wykonanie powoduje wykonanie całej sekwencji. 

Jeśli delegaty zwracają wyniki, to wynik złożonego delegata jest wynikiem
ostatnio wykonanego delegata.

\begin{scriptsize}
\begin{verbatim}
/* Wiktor Zychla, 2003 */
using System;
using System.Text;

namespace Example
{ 
  public class InfoClass
  {
  	public static void WypiszKwadrat( int n )
  	{
  	  Console.WriteLine( n*n );
  	}
  	
  	public static void WypiszDwukrotnosc( int n )
  	{
  	  Console.WriteLine( n+n );
  	}
  }
	
  public class CMain
  {    
  	public delegate void InfoDelegate( int n );
  	
  	public static void Oblicz( int n, InfoDelegate m )
  	{
  	  m( n );
  	}
  	
    public static void Main()
    {
      InfoDelegate d1 = new InfoDelegate( InfoClass.WypiszKwadrat );
      InfoDelegate d2 = new InfoDelegate( InfoClass.WypiszDwukrotnosc );

      InfoDelegate d3 = d1+d2;

      Oblicz( 11, d3 );    	    	
    }
  }
}
\end{verbatim}
\end{scriptsize}

W dalszej części rozdziału zostanie pokazane w jaki sposób można obejść to ograniczenie i otrzymywać
wyniki ze wszystkich kolejno wykonujących się delegatów jak również - jak wykonywać delegaty w sposób asynchroniczny.

\subsubsection{Zdarzenia}

Kiedy zachodzi konieczność poinformowania jakiegoś obiektu o zajściu jakiegoś zdarzenia, bardzo przydatny
okazuje się mechanizm zdarzeń. Obiekt, który jest sprawcą pojawienia się zdarzenia przechowuje listę
delegatów, którzy zostaną wykonani kiedy zdarzenie ma zostać ogłoszone światu. Każdy inny obiekt, który
jest zainteresowany otrzymaniem powiadomienia, po prostu dopisuje swojego delegata do listy delegatów
zdarzenia.

\begin{scriptsize}
\begin{verbatim}
/* Wiktor Zychla, 2003 */
using System;
using System.Threading;

namespace DelegaciIZdarzenia
{
  public delegate void CZdarzenieDelegate( CZdarzenieEventArgs e );

  // Argumenty zdarzenia
  public class CZdarzenieEventArgs
  {
    public int informacja1;
    public int informacja2;

    private CZdarzenieEventArgs() {}
    public  CZdarzenieEventArgs( int Informacja1, int Informacja2 )
    {
      informacja1 = Informacja1;
      informacja2 = Informacja2;
    }
  }
	
  // Obiekt, który będzie wysyłał zdarzenie
  public class CObiekt
  {
    public event CZdarzenieDelegate Zdarzenie;

    public void ZdarzenieZaszlo( CZdarzenieEventArgs e )
    {
      this.Zdarzenie(e);
    }

    public CObiekt() {}
  }

  class CMain
  {
    // Reakcja na zdarzenie
    static void Reakcja1( CZdarzenieEventArgs e )
    {
      Console.WriteLine( String.Format( "Reakcja 1: {0},{1}", 
                         e.informacja1, e.informacja2 ) );
    }

    static void Reakcja2( CZdarzenieEventArgs e )
    {
      Console.WriteLine( String.Format( "Reakcja 2: {0},{1}", 
                         e.informacja1, e.informacja2 ) );
    }

    public static void Main()
    {
      CObiekt obiekt = new CObiekt();
      obiekt.Zdarzenie += new CZdarzenieDelegate( Reakcja1 );
      Thread.Sleep(1000);
      obiekt.ZdarzenieZaszlo( new CZdarzenieEventArgs( 1, 2 ) );
      Thread.Sleep(1000);
      obiekt.Zdarzenie += new CZdarzenieDelegate( Reakcja2 );
      obiekt.ZdarzenieZaszlo( new CZdarzenieEventArgs( 3, 4 ) );
      Thread.Sleep(1000);
      Console.WriteLine( "koniec" );
    }
  }
}

C:\Example>example.exe
Reakcja 1: 1,2
Reakcja 1: 3,4
Reakcja 2: 3,4
koniec
\end{verbatim}
\end{scriptsize}

\subsubsection{Uczeń Czarnoksiężnika}

Możliwości C\# w zakresie delegatów i zdarzeń podsumujmy bajką o uczniu czarnoksiężnika\footnote{Oryginał,
historia pracownika biurowego, dostępny jest pod adresem 
{\em http://www.sellsbrothers.com/writing/default.aspx?content=delegates.htm}.}. 

{\em
Dawno dawno temu, za siedmioma górami i siedmioma rzekami, mieszkał potężny Czarnoksiężnik. Czarnoksiężnik miał
Ucznia, który bardzo chciał kiedyś być tak mądry jak Czarnoksiężnik. Póki co, Czarnoksiężnik wymyślał
swojemu Uczniowi kolejne, coraz bardziej skomplikowane zadania, a Uczeń skrupulatnie je wykonywał.

Taki układ trwał i trwał, aż w końcu Czarnoksiężnik uznał, że nie musi już cały czas doglądać pracy swojego 
Ucznia. "Uczniu!" - rzekł któregoś dnia - "Jesteś już na tyle samodzielny, że w czasie kiedy pracujesz mógłbym
zająć się swoimi sprawami. Po prostu informuj mnie o tym, kiedy skończysz pracę." 

\begin{scriptsize}
\begin{verbatim}
using System;
using System.Threading;

namespace UczenCzarnoksieznika
{	
  class Uczen
  {
    public void PoradzSie( Czarnoksieznik czarnoksieznik )
    {
      _czarnoksieznik = czarnoksieznik;  	
    }
    public void Pracuj()
    {
      Console.WriteLine( "Uczen: rozpoczynam prace." );
      if ( _czarnoksieznik != null ) _czarnoksieznik.PracaRozpoczeta();
  		
      Console.WriteLine( "Uczen: pracuje." );
      if ( _czarnoksieznik != null ) _czarnoksieznik.PracaTrwa();
  	  
      Console.WriteLine( "Koncze prace." );
      if ( _czarnoksieznik != null ) 
      {
        int ocena = _czarnoksieznik.PracaZakonczona();
        Console.WriteLine( "Ocena : {0}", ocena );
      }
    }
    private Czarnoksieznik _czarnoksieznik;
  }
  
  class Czarnoksieznik
  {
    public void PracaRozpoczeta() { }
    public void PracaTrwa()       { }
    public int PracaZakonczona() 
    {
      Console.WriteLine( "Ach, znakomicie!" );
      return 2;
    }
  }
  
  class Uniwersum
  {
    public static void Main()
    {
      Uczen uczen = new Uczen();
      Czarnoksieznik czarnoksieznik = new Czarnoksieznik();
  		
      uczen.PoradzSie( czarnoksieznik );  		
      uczen.Pracuj();
    }
  }
}
\end{verbatim}
\end{scriptsize}

Wydawało się, że wszystko funkcjonuje jak należy, jednak Uczeń zaczął zastanawiać się co by się
stało, gdyby nie tylko Czarnoksiężnik, ale ktoś inny był również zainteresowany jego postępami
(na przykład Uczennica pewnej zaprzyjaźnionej z Czarnoksiężnikiem Wróżki). W pierwszej chwili Uczeń
trochę się zmartwił, bo wyobraził sobie, że jak dużo różnych metod musiałby znać, aby o swoich postępach
informować innych. W końcu każdy mógłby chcieć być informowany w trochę inny sposób.
Trochę się jednak uspokoił kiedy pomyślał o rozdzieleniu listy możliwych powiadomień
od implementacji tych powiadomień. Zaprojektował więc odpowiedni interfejs.

\begin{scriptsize}
\begin{verbatim}
using System;
using System.Threading;

namespace UczenCzarnoksieznika
{	
  interface IUczenPowiadomi
  {
    void PracaRozpoczeta();
    void PracaTrwa();
    int  PracaZakonczona();
  }
  
  class Uczen
  {
    public void PoradzSie( IUczenPowiadomi uczenpowiadomi )
    {
      _uczenpowiadomi = uczenpowiadomi;  	
    }
    public void Pracuj()
    {
      Console.WriteLine( "Uczen: rozpoczynam prace." );
      if ( _uczenpowiadomi != null ) _uczenpowiadomi.PracaRozpoczeta();
  		
      Console.WriteLine( "Uczen: pracuje." );
      if ( _uczenpowiadomi != null ) _uczenpowiadomi.PracaTrwa();
  	  
      Console.WriteLine( "Koncze prace." );
      if ( _uczenpowiadomi != null ) 
      {
        int ocena = _uczenpowiadomi.PracaZakonczona();
        Console.WriteLine( "Ocena : {0}", ocena );
      }
    }
    private IUczenPowiadomi _uczenpowiadomi;
  }
  
  class Czarnoksieznik : IUczenPowiadomi
  {
    public void PracaRozpoczeta() { }
    public void PracaTrwa()       { }
    public int PracaZakonczona() 
    {
      Console.WriteLine( "Ach, znakomicie!" );
      return 3;
    }
  }
  
  class Uniwersum
  {
    public static void Main()
    {
      Uczen uczen = new Uczen();
      Czarnoksieznik czarnoksieznik = new Czarnoksieznik();
  		
      uczen.PoradzSie( czarnoksieznik );  		
      uczen.Pracuj();
    }
  }
}
\end{verbatim}
\end{scriptsize}

Przekonanie Czarnoksiężnika do zaimplementowania interfejsu trochę trwało i choć na razie nikt 
inny nie był implementowaniem jego interfejsu zainteresowany, to Uczeń był z siebie bardzo zadowolony.
"W końcu teraz" - pomyślał - "każdy zainteresowany będzie mógł łatwo dowiedzieć się jak sobie radzę."

Czarnoksiężnik nie był jednak zachwycony. "Uczniu!" - zagrzmiał - "Dlaczego zadajesz sobie
tyle trudu informując mnie o rozpoczęciu Twojej pracy i o jej trwaniu? Nie jestem tym zainteresowany.
Nie dość, że zmuszasz mnie do implementowania odpowiednich metod w interfejsie, 
to jeszcze tracisz swój czas czekając
aż zauważę Twoje poczynania. Przecież gdybym akurat gdzieś wybył, to musiałbyś bardzo długo czekać
na zakończenie wykonania moich metod. Zrób coś z tym, Uczniu!"

Chcąc nie chcąc, zganiony przez swojego Mistrza, Uczeń uznał, 
że interfejsy są owszem użyteczne w wielu przypadkach, jednak
niespecjalnie nadają się do implementowania zdarzeń. Pomyślał, że rzeczywiście byłoby właściwie informować
zainteresowanych tylko o tych wydarzeniach, którymi są oni zainteresowani. Zamiast interfejsu stworzył
więc delegatów do odpowiednich funkcji.

\begin{scriptsize}
\begin{verbatim}
using System;
using System.Threading;

namespace UczenCzarnoksieznika
{	
  delegate void PracaRozpoczeta();
  delegate void PracaTrwa();
  delegate int  PracaZakonczona();
  
  class Uczen
  {
    public void Pracuj()
    {
      Console.WriteLine( "Uczen: rozpoczynam prace." );
      if ( rozpoczynaprace != null ) rozpoczynaprace();
  		
      Console.WriteLine( "Uczen: pracuje." );
      if ( pracuje != null ) pracuje();
  	  
      Console.WriteLine( "Koncze prace." );
      if ( zakonczylprace != null ) 
      {
        int ocena = zakonczylprace();
        Console.WriteLine( "Ocena : {0}", ocena );
      }
    }
    public PracaRozpoczeta rozpoczynaprace;
    public PracaTrwa       pracuje;
    public PracaZakonczona zakonczylprace;
  }
  
  class Czarnoksieznik 
  {
    public int PracaZakonczona() 
    {
      Console.WriteLine( "Ach, znakomicie!" );
      return 4;
    }
  }
  
  class Uniwersum
  {
    public static void Main()
    {
      Uczen uczen = new Uczen();
      Czarnoksieznik czarnoksieznik = new Czarnoksieznik();
  		
      uczen.zakonczylprace = new PracaZakonczona( czarnoksieznik.PracaZakonczona );  		
      uczen.Pracuj();
    }
  }
}
\end{verbatim}
\end{scriptsize}

W ten sposób Uczeń przestał zajmować Czarnoksiężnika zdarzeniami, którymi tamten nie był zainteresowany.
W międzyczasie okazało się, że samo Uniwersum zainteresowało się poczynaniami Ucznia i chciało być informowane o
rozpoczęciu i zakończeniu przez niego pracy.

\begin{scriptsize}
\begin{verbatim}
using System;
using System.Threading;

namespace UczenCzarnoksieznika
{	
  delegate void PracaRozpoczeta();
  delegate void PracaTrwa();
  delegate int  PracaZakonczona();
  
  class Uczen
  {
    public void Pracuj()
    {
      Console.WriteLine( "Uczen: rozpoczynam prace." );
      if ( rozpoczynaprace != null ) rozpoczynaprace();
  		
      Console.WriteLine( "Uczen: pracuje." );
      if ( pracuje != null ) pracuje();
  	  
      Console.WriteLine( "Koncze prace." );
      if ( zakonczylprace != null ) 
      {
        int ocena = zakonczylprace();
        Console.WriteLine( "Ocena : {0}", ocena );
      }
    }
    public PracaRozpoczeta rozpoczynaprace;
    public PracaTrwa       pracuje;
    public PracaZakonczona zakonczylprace;
  }
  
  class Czarnoksieznik 
  {
    public int PracaZakonczona() 
    {
      Console.WriteLine( "Ach, znakomicie!" );
      return 4;
    }
  }
  
  class Uniwersum
  {  	
    static void UczenRozpoczalPrace()
    {
      Console.WriteLine( "Uniwersum zauwaza, ze Uczen rozpoczal prace." );      
    }
    static int UczenZakoczylPrace()
    {
      Console.WriteLine( "Uniwersum zauwaza, ze Uczen zakonczyl prace." );      
      return 7;
    }
    public static void Main()
    {
      Uczen uczen = new Uczen();
      Czarnoksieznik czarnoksieznik = new Czarnoksieznik();
  		
      uczen.rozpoczynaprace = new PracaRozpoczeta( Uniwersum.UczenRozpoczalPrace );    	
      uczen.zakonczylprace = new PracaZakonczona( czarnoksieznik.PracaZakonczona );  		
      uczen.zakonczylprace = new PracaZakonczona( Uniwersum.UczenZakoczylPrace );    	
      uczen.Pracuj();
    }
  }
}
\end{verbatim}
\end{scriptsize}

Katastrofa! Okazało się, że uczynienie delegatów publicznymi polami w swojej klasie, było błędem Ucznia. 
Uniwersum, w swoim uniwersalnym wymiarze, przysłoniło powiadomienie o zakończeniu pracy skierowane 
do Czarnoksiężnika swoim własnym. 

Uczeń postanowił, że musi coś na to poradzić. Zdał sobie sprawę, że potrzebuje jakiegoś
mechanizmu rejestrowania i wyrejestrowywania delegatów, tak aby słuchacze zdarzeń mogli 
dodawać i usuwać swoje funkcje do powiadomień, ale nie mogli zniszczyć całej listy funkcji powiadomień.
Uczeń skorzystał więc ze {\em zdarzeń}, o których wiedział że automatycznie tworzą odpowiednie propercje
związane z obsługą delegatów, tak że słuchacze mogli być dodawani i usuwani za pomocą operatorów
+= i -=.

\begin{scriptsize}
\begin{verbatim}
using System;
using System.Threading;

namespace UczenCzarnoksieznika
{	
  delegate void PracaRozpoczeta();
  delegate void PracaTrwa();
  delegate int  PracaZakonczona();
  
  class Uczen
  {
    public void Pracuj()
    {
      Console.WriteLine( "Uczen: rozpoczynam prace." );
      if ( rozpoczynaprace != null ) rozpoczynaprace();
  		
      Console.WriteLine( "Uczen: pracuje." );
      if ( pracuje != null ) pracuje();
  	  
      Console.WriteLine( "Koncze prace." );
      if ( zakonczylprace != null ) 
      {
        int ocena = zakonczylprace();
        Console.WriteLine( "Ocena : {0}", ocena );
      }
    }
    public event PracaRozpoczeta rozpoczynaprace;
    public event PracaTrwa       pracuje;
    public event PracaZakonczona zakonczylprace;
  }
  
  class Czarnoksieznik 
  {
    public int PracaZakonczona() 
    {
      Console.WriteLine( "Ach, znakomicie!" );
      return 4;
    }
  }
  
  class Uniwersum
  {  	
    static void UczenRozpoczalPrace()
    {
      Console.WriteLine( "Uniwersum zauwaza, ze Uczen rozpoczal prace." );      
    }
    static int UczenZakoczylPrace()
    {
      Console.WriteLine( "Uniwersum zauwaza, ze Uczen zakonczyl prace." );      
      return 7;
    }
    public static void Main()
    {
      Uczen uczen = new Uczen();
      Czarnoksieznik czarnoksieznik = new Czarnoksieznik();
  		
      uczen.rozpoczynaprace += new PracaRozpoczeta( Uniwersum.UczenRozpoczalPrace );    	
      uczen.zakonczylprace += new PracaZakonczona( czarnoksieznik.PracaZakonczona );  		
      uczen.zakonczylprace += new PracaZakonczona( Uniwersum.UczenZakoczylPrace );    	
      uczen.Pracuj();
    }
  }
}
\end{verbatim}
\end{scriptsize}

Po tym wszystkim Uczeń odetchnął z ulgą. W elegancki sposób poradził sobie z zaspokojeniem
potrzeb wszystkich zainteresowanych jego postępami, a sam nie musiał się specjalnie
przejmować wewnętrznymi implementacjami ich metod. Zauważył jedynie, że choć zarówno Czarnoksiężnik jak i 
Uniwersum oceniają jego poczynania, to do niego dociera tylko jedna ocena. Uczeń chciał zaś znać
oceny, które wystawiają mu wszyscy zainteresowani zakończeniem przez niego pracy. 
Na szczęście był w stanie przeglądać listę słuchaczy zdarzenia i zbierać wyniki od wszystkich
po kolei.

\begin{scriptsize}
\begin{verbatim}
using System;
using System.Threading;

namespace UczenCzarnoksieznika
{	
  delegate void PracaRozpoczeta();
  delegate void PracaTrwa();
  delegate int  PracaZakonczona();
  
  class Uczen
  {
    public void Pracuj()
    {
      Console.WriteLine( "Uczen: rozpoczynam prace." );
      if ( rozpoczynaprace != null ) rozpoczynaprace();
  		
      Console.WriteLine( "Uczen: pracuje." );
      if ( pracuje != null ) pracuje();
  	  
      Console.WriteLine( "Koncze prace." );
      if ( zakonczylprace != null ) 
      {
        foreach ( PracaZakonczona pz in zakonczylprace.GetInvocationList() )      	
        {
          int ocena = pz();
          Console.WriteLine( "Ocena : {0}", ocena );
        }
      }
    }
    public event PracaRozpoczeta rozpoczynaprace;
    public event PracaTrwa       pracuje;
    public event PracaZakonczona zakonczylprace;
  }
  
  class Czarnoksieznik 
  {
    public int PracaZakonczona() 
    {
      Console.WriteLine( "Ach, znakomicie!" );
      return 4;
    }
  }
  
  class Uniwersum
  {  	
    static void UczenRozpoczalPrace()
    {
      Console.WriteLine( "Uniwersum zauwaza, ze Uczen rozpoczal prace." );      
    }
    static int UczenZakoczylPrace()
    {
      Console.WriteLine( "Uniwersum zauwaza, ze Uczen zakonczyl prace." );      
      return 7;
    }
    public static void Main()
    {
      Uczen uczen = new Uczen();
      Czarnoksieznik czarnoksieznik = new Czarnoksieznik();
  		
      uczen.rozpoczynaprace += new PracaRozpoczeta( Uniwersum.UczenRozpoczalPrace );    	
      uczen.zakonczylprace += new PracaZakonczona( czarnoksieznik.PracaZakonczona );  		
      uczen.zakonczylprace += new PracaZakonczona( Uniwersum.UczenZakoczylPrace );    	
      uczen.Pracuj();
    }
  }
}
\end{verbatim}
\end{scriptsize}

Uczeń usiadł na chwilę zadowolony ze swoich pomysłów. Niestety, okazało się że zarówno Uniwersum jak
i Czarnoksiężnik mają mnóstwo własnych zajęć i zauważenie postępów ucznia zajmuje im coraz
więcej czasu.

\begin{scriptsize}
\begin{verbatim}
  class Czarnoksieznik 
  {
    public int PracaZakonczona() 
    {
      Thread.Sleep( 5000 );
      Console.WriteLine( "Ach, znakomicie!" );
      return 4;
    }
  }
  
  class Uniwersum
  {  	
    static int UczenZakoczylPrace()
    {
      Thread.Sleep( 7000 );
      Console.WriteLine( "Uniwersum zauwaza, ze Uczen zakonczyl prace." );      
      return 7;
    }
    ...
  }
}
\end{verbatim}
\end{scriptsize}

Dla Ucznia oznaczało to, że zamiast pilnie pracować musi po wywołaniu słuchacza zdarzenia czekać
w nieskończoność na jego zakończenie. Postanowił więc chwilowo przestać przejmować się ocenami, za to
wołać odpowiednich słuchaczy asynchronicznie.

\begin{scriptsize}
\begin{verbatim}
    public void Pracuj()
    {
      ...
      Console.WriteLine( "Koncze prace." );
      if ( zakonczylprace != null ) 
      {
        foreach ( PracaZakonczona pz in zakonczylprace.GetInvocationList() )      	
        {
          pz.BeginInvoke( null, null );        	
        }
      }
    }
\end{verbatim}
\end{scriptsize}

Dzięki temu Uczeń mógł natychmiast po wywołaniu powiadomienia zająć się z powrotem swoimi sprawami. 
Brakowało mu jednak tego, że ktoś docenia jego pracę. 
Postanowił więc nadal wołać słuchaczy asynchronicznie i 
co jakiś czas sprawdzać, czy jego praca jest już oceniona.

\begin{scriptsize}
\begin{verbatim}
  class Uczen
  {
    public void Pracuj()
    {
      ...   	  
      Console.WriteLine( "Koncze prace." );
      if ( zakonczylprace != null ) 
      {
        foreach ( PracaZakonczona pz in zakonczylprace.GetInvocationList() )      	
        {
          IAsyncResult res = pz.BeginInvoke( null, null );        	
          while ( !res.IsCompleted ) Thread.Sleep(1);
          int ocena = pz.EndInvoke(res);
          Console.WriteLine( "Ocena : {0}", ocena );                   	
        }
      }
    }
\end{verbatim}
\end{scriptsize}

W ten sposób, niestety, Uczeń co prawda mógł kontynuować swoją pracę natychmiast po
zawołaniu funkcji-słuchacza, jednak w osobnym wątku co chwila
zaglądał przez ramię czy zawołany słuchacz zdarzenia już zakończył pracę. Uczeń nie był z tej konieczności
zadowolony, postanowił więc zatrudnić własnego delegata który powiadamiałby go o zakończeniu pracy przez 
asynchronicznego delegata.

\begin{scriptsize}
\begin{verbatim}
using System;
using System.Threading;

namespace UczenCzarnoksieznika
{	
  delegate void PracaRozpoczeta();
  delegate void PracaTrwa();
  delegate int  PracaZakonczona();
  
  class Uczen
  {
    public void Pracuj()
    {
      Console.WriteLine( "Uczen: rozpoczynam prace." );
      if ( rozpoczynaprace != null ) rozpoczynaprace();
  		
      Console.WriteLine( "Uczen: pracuje." );
      if ( pracuje != null ) pracuje();
  	  
      Console.WriteLine( "Koncze prace." );
      if ( zakonczylprace != null ) 
      {
        foreach ( PracaZakonczona pz in zakonczylprace.GetInvocationList() )      	
        {
          pz.BeginInvoke( new AsyncCallback(OcenPrace), pz );
        }
      }
    }
    
    private void OcenPrace( IAsyncResult res )
    {
      PracaZakonczona pz = (PracaZakonczona)res.AsyncState;
      int ocena = pz.EndInvoke(res);
      Console.WriteLine( "Ocena : {0}", ocena );                   	          	
    }
    
    public event PracaRozpoczeta rozpoczynaprace;
    public event PracaTrwa       pracuje;
    public event PracaZakonczona zakonczylprace;
  }
  
  class Czarnoksieznik 
  {
    public int PracaZakonczona() 
    {
      Thread.Sleep( 5000 );
      Console.WriteLine( "Ach, znakomicie!" );
      return 4;
    }
  }
  
  class Uniwersum
  {  	
    static void UczenRozpoczalPrace()
    {
      Console.WriteLine( "Uniwersum zauwaza, ze Uczen rozpoczal prace." );      
    }
    static int UczenZakoczylPrace()
    {
      Thread.Sleep( 7000 );
      Console.WriteLine( "Uniwersum zauwaza, ze Uczen zakonczyl prace." );      
      return 7;
    }
    public static void Main()
    {
      Uczen uczen = new Uczen();
      Czarnoksieznik czarnoksieznik = new Czarnoksieznik();
  		
      uczen.rozpoczynaprace += new PracaRozpoczeta( Uniwersum.UczenRozpoczalPrace );    	
      uczen.zakonczylprace += new PracaZakonczona( czarnoksieznik.PracaZakonczona );  		
      uczen.zakonczylprace += new PracaZakonczona( Uniwersum.UczenZakoczylPrace );    	
      uczen.Pracuj();
      Thread.Sleep( 20000 );
    }
  }
}
\end{verbatim}
\end{scriptsize}

Teraz wszyscy byli zadowoleni. Czarnoksiężnik i Uniwersum byli powiadamiani o zdarzeniach, które ich
interesowały. Uczeń mógł powiadamiać wszystkich zainteresowanych, a sam nie musiał czekać na zakończenie 
metod implementujących powiadomienia. Mógł za to asynchronicznie zbierać wyniki tych metod.

Zmęczony całym dniem ciężkiej pracy, Uczeń mógł w końcu iść spać...
}

