\section{C\# 2.0}

\subsection{Typy generyczne}

\subsubsection{Motywacja}

Rozważmy przykład klasy wektora dwuwymiarowego o składowych całkowitoliczbowych:

\begin{scriptsize}
\begin{verbatim}
public class Vector
{
    public Vector()
    {
    }

    public Vector( int x, int y )
    {
        this.X = x;
        this.Y = y;
    }

    public int X { get; set; }
    public int Y { get; set; }
}

...
Vector v1 = new Vector(1, 0);
Vector v2 = new Vector(2, 1);
\end{verbatim}
\end{scriptsize}

{\em Przykład ten wykorzystuje skrócony sposób definiowania właściwości, o których wcześniej była mowa
w podrozdziale~\ref{properties}. Ten nowy skrócony sposób zostanie dokładniej omówiony w rozdziale~\ref{cs3}}. 

Taką konstrukcję można powtórzyć dla co kilku innych typów ({\tt double}, {\tt string}) otrzymując wektory liczb
zmiennoprzecinkowych czy wektory słów. Niestety, powtórzenie oznacza, że pojawiają się kolejne typy, niezależne od poprzednich,
nie bardzo też jest jak zastosować tu dziedziczenie w celu osiągnięcia reużywalności kodu - wszak składowe wektora miałyby
różne typy w różnych wariantach tej klasy.

Pierwszym, oczywistym pomysłem na uogólnienie tego typu konstrukcji, jest użycie najbardziej ogólnego typu do
reprezentowania składowych, w tym przypadku typu {\tt object}:

\begin{scriptsize}
\begin{verbatim}
public class Vector
{
    public Vector()
    {
    }

    public Vector(object x, object y)
    {
        this.X = x;
        this.Y = y;
    }

    public object X { get; set; }
    public object Y { get; set; }
}
...
Vector v1 = new Vector(1, 0);
Vector v2 = new Vector(2.0, 1.1);
Vector v3 = new Vector("foo", "bar");
\end{verbatim}
\end{scriptsize}

Niestety, ceną jaką przychodzi zapłacić za takie otwarcie jest utrata {\bf statycznej} kontroli typów. W szczególności
kompilator nie zaprotestuje w żaden sposób przeciwko

\begin{scriptsize}
\begin{verbatim}
Vector v = new Vector(1, "foo");
\end{verbatim}
\end{scriptsize}

co być może miewa sens, ale w typowym przypadku wygląda na dość poważny błąd, zwykle możliwy do wykrycia dopiero
w trakcie działania aplikacji a nie jej kompilacji. 

Cóż, pierwotna wersja, w której typy składowych pojawiały się w definicji klasy jawnie, nie miała takiej wady -
tam istniała statyczna kontrola typów podczas kompilacji.

Programowanie generyczne przy pomocy {\bf typów generycznych} pozwala w elegancki sposób {\bf uogólnić} definicję
klasy, pozostawiając w niej definicje niektórych składowych w postaci {\em otwartej}, której {\em ukonkretnienie}
odbywa się dopiero w miejscu użycia. Takie podejście rozwiązuje zaobserwowany wcześniej problem utraty kontroli
nad statycznym typowaniem.

\begin{scriptsize}
\begin{verbatim}
public class Vector<T>
{
    public Vector()
    {
    }

    public Vector( T x, T y )
    {
        this.X = x;
        this.Y = y;
    }

    public T X { get; set; }
    public T Y { get; set; }
}

...
Vector<int>    v1 = new Vector<int>(1, 0);
Vector<double> v2 = new Vector<double>(2.0, 1.1);
Vector<string> v3 = new Vector<string>("foo", "bar");

\end{verbatim}
\end{scriptsize}

Tu próba obejścia statycznej kontroli typów nie udaje się:

\begin{scriptsize}
\begin{verbatim}
Vector<int> v = new Vector<int>(1, "foo");
...
Error	CS1503	Argument 2: cannot convert from 'string' to 'int'
\end{verbatim}
\end{scriptsize}

Z punktu widzenia programisty sprawa jest więc prosta - jedna i ta sama klasa może występować w wielu postaciach, za każdym
razem parametr typowy ukonkretnia się tak, jak wynika to z kontekstu użycia. Typy generyczne mogą być argumentami klas

\begin{scriptsize}
\begin{verbatim}
public class Foo<T>
{
    ...
}
\end{verbatim}
\end{scriptsize}

metod w klasach 

\begin{scriptsize}
\begin{verbatim}
public class Foo
{
    public void Bar<T>( T t )
    {
        ...
    }
}
\end{verbatim}
\end{scriptsize}

lub interfejsów

\begin{scriptsize}
\begin{verbatim}
public interface IFoo<T>
{
    void Bar(T t);
}

public interface Qux
{
    void Baz<T>(T t);
}
\end{verbatim}
\end{scriptsize}

Nie ma również ograniczeń na liczbę parametrów typowych, może ich być wiele, w tym - część może być na poziomie typu
(klasy lub interfejsu) a część na poziomie metody

\begin{scriptsize}
\begin{verbatim}
public class Foo<U,V>
{
    public void Bar<S,T>(U u, V v, S s, T t)
    {

    }
}
...
Foo<int, string> foo = new Foo<int, string>();
foo.Bar<double, char>(1, "foo", 2.1, 'c');
\end{verbatim}
\end{scriptsize}

W tym ostatnim przypadku nie ma nawet potrzeby ukonkretniania parametrów typowych przy wywołaniu metody,
ponieważ kompilator może odtworzyć typy automatycznie z kontekstu użycia

\begin{scriptsize}
\begin{verbatim}
// to
foo.Bar(1, "foo", 2.1, 'c');

// jest równoważne temu
foo.Bar<double, char>(1, "foo", 2.1, 'c');
\end{verbatim}
\end{scriptsize}

\subsubsection{Ograniczenia na typy generyczne}

Może się wydawać, że mechanizm typów generycznych to całkiem sporo nowych możliwości. Czar pryska w chwili
kiedy oprócz danych, klasa musi zawierać również metody.

Klasa wektora generycznego z poprzedniego podrozdziału jest bardzo prosta - zawiera wyłącznie dane.
Próba implementacji jakiegokolwiek prostego algorytmu jest ilustracją wspomnianej trudności. Niech tym prostym
algorytmem będzie dodawanie wektorów; do bieżącego wektora ({\tt this}) będzie dodawany wektor przesunięcia, a wynik
ma być sumą obu wektorów:

\begin{scriptsize}
\begin{verbatim}
public class Vector<T>
{
    public Vector()
    {
    }

    public Vector( T x, T y )
    {
        this.X = x;
        this.Y = y;
    }

    public T X { get; set; }
    public T Y { get; set; }

    public Vector<T> Add( Vector<T> v )
    {
        return new Vector<T>(this.X + v.X, this.Y + v.Y); // <- czy tak można?
    }
}
...
Error	CS0019	Operator '+' cannot be applied to operands of type 'T' and 'T'	
\end{verbatim}
\end{scriptsize}

Sytuacja jest szczególnie zaskakująca dla programistów mających doświadczenie z klasami szablonowymi w C++, gdzie
analogiczna konstrukcja przechodzi bez problemów:

\begin{scriptsize}
\begin{verbatim}
template <class T> 
class Vector
{
public:

    Vector()
    {

    }
    Vector(T x, T y)
    {
        this->X = x;
        this->Y = y;
    }

    T X;
    T Y;

    Vector Add(Vector const& v)
    {
        return Vector(this->X + v.X, this->Y + v.Y);
    }
};

int main()
{
    Vector<int> v1(1, 0);
    Vector<int> v2(2, 1);

    // w C++ ok
    Vector<int> v3 = v1.Add(v2);
}
\end{verbatim}
\end{scriptsize}

Zaobserwowana różnica ilustruje {\bf fundamentalną różnicę} w sposobie implementacji typów generycznych między
C\# a C++.

W języku C++ klasa szablonowa jest traktowana podobnie jak {\bf makrodefinicja} i w trakcie kompilacji samej klasy
generycznej kompilator nie próbuje rozstrzygać czy napotkany w wyrażeniu arytmetycznym operator {\tt +}
ma sens czy nie w ogólnym przypadku.

Dopiero w trakcie ukonkretnienia szablonu typem, w tym przypadku {\tt int}, kompilator {\em rozwija definicję klasy szablonowej}
i napotyka na wyrażenie w którym po obu stronach {\tt +} występują wartości typu {\tt int}.

W C++ problemem byłoby dopiero

\begin{scriptsize}
\begin{verbatim}
class Person
{
    Person()
    {

    }
};

Vector<Person> p1;
Vector<Person> p2;

Vector<Person> p3 = p1.Add(p2);
...
Error	C2676	binary '+': 'Person' does not define this operator or a conversion 
                to a type acceptable to the predefined operator
Error	C2088	'+': illegal for class	
\end{verbatim}
\end{scriptsize}

i słusznie - dla klasy {\tt Person} operacja {\tt +} z pewnością nie ma żadnej sensownej interpretacji, a
żadna nie została dostarczona. Rozwiązanie w tym przypadku polega własnie na dostarczeniu takiej
implementacji operatora {\tt +} który nada sens wyrażeniu arytmetycznemu {\tt Person + Person}:

\begin{scriptsize}
\begin{verbatim}
class Person
{
public:
    Person()
    {

    }
    Person operator+(const Person &p) 
    {
        // cokolwiek byleby zdefiniować operator
        return Person();
    }
};
\end{verbatim}
\end{scriptsize}

W C\# typy generyczne nie są makrodefinicjami, są w pełni kompilowalnymi klasami, które muszą mieć sens 
bez względu na to czy i jak będą ukonkretniane. Oznacza to że kompilator musi umieć nadać sens operatorowi {\tt +}
oraz każdemu innemu wyrażeniu w którym występuje zmienna otwartego typu
bez względu na to czy kiedyś, gdzieś, klasa generyczna będzie ukonkretniana typem {\tt int}, {\tt Person} czy
jakimkolwiek innym.

Zauważmy, że główna trudność w C\# polega więc na tym, że skoro w klasie generycznej zmienna typowa może oznaczać
{\em dowolny} typ, to problem nie dotyczy tylko operatora {\tt +}, jak w przykładzie klasy {\tt Vector}. Skala problemu
jest szersza - o type nie wiadomo niczego, nie wiadomo więc czy można do niego zastosować operator {\tt new} albo czy można
na zmiennej tego typu wywoływać jakiekolwiek metody:

\begin{scriptsize}
\begin{verbatim}
public class Foo<T>
{
    public void Bar(T t)
    {
        T _new = new T(); // czy na pewno wolno new T()?

        t.Qux();          // czy T ma metodę Qux()?
    }
}
\end{verbatim}
\end{scriptsize}

Żadna z przykładowych problematycznych linii w powyższym przykładzie nie powinna się skompilować. 
Kompilator nie może zezwolić na użycie operatora {\tt new}, ponieważ w miejscu ukonkretnienia typu {\tt Foo} może 
pojawić się ukonkretnienie typem, do którego {\tt new} nie ma zastosowania, na przykład {\tt int}. Z tego samego
powodu kompilator nie może dopuścić do skompilowania kodu, w którym na zmiennej nieznanego typu wywołana jest jakakolwiek metoda.

Język C\# z tymi problemami radzi sobie proponując tzw. {\bf ograniczenia typowe}. Są to dodatkowe warunki narzucane
statycznie na typy ukonkretniające typ generyczny, które są rodzajem kontraktu między klasą która używa typu
a klasą która będzie go ukonkretniać. Dzięki ograniczeniu typowemu klasa używająca zmiennej typowej dostanie dostęp
do większej liczby operacji które zostaną uznane za dozwolone, z kolei w miejscu ukonkretnienia kompilator
odrzuci typy które nie spełniają ograniczeń.

Dozwolone są następujące ograniczenia typowe:
\begin{itemize}
\item {\tt struct} - ograniczenie spełniają tylko typy proste (struktury) ({\tt int} itp.)
\item {\tt class} - ograniczenie spełniają tylko typy referencyjne
\item {\tt new()} - ograniczenie spełniają tylko typy posiadające konstruktor bezargumentowy (i być może inne konstruktory)
\item {\tt : I} - ograniczenie spełniają tylko typy dziedziczące z {\tt I} gdzie {\tt I} jest jakimś konkretnym interfejsem znanym w trakcie kompilacji
\item {\tt : C} - ograniczenie spełniają tylko typy dziedziczące z {\tt C} gdzie {\tt C} jest jakimś konkretnym typem znanym w trakcie kompilacji
\end{itemize}

Nałożenie ograniczenia wymaga dopisania do definicji klasy generycznej słowa kluczowego {\tt where} i wskazaniu jednego 
(lub wielu, w sytuacji kiedy nie są ze sobą sprzeczne) ograniczenia. Nałożenie ograniczenia daje możliwość użycia
funkcjonalności narzuconej ograniczeniem.

\begin{scriptsize}
\begin{verbatim}
public interface IQux
{
    void Qux();
}

public class Foo<T>
    where T : IQux, new()
{
    public void Bar(T t)
    {
        T _new = new T(); // czy na pewno wolno new T()?
        // tak, wolno, bo T ma ograniczenie new()

        t.Qux();          // czy T ma metodę Qux()?
        // tak, ma, bo T ma ograniczenie : IQux
    }
}
\end{verbatim}
\end{scriptsize}

Próba ukonkretnienia klasy generycznej typem niespełniającym ograniczeń nie uda się na etapie kompilacji:

\begin{scriptsize}
\begin{verbatim}
public class C1 : IQux
{
    public void Qux() { }
}

public class C2 : IQux
{
    public C2(int n) { }
    public void Qux() { }
}

public class C3
{
    
}
...

Foo<C1> c1 = new Foo<C1>();
Foo<C2> c2 = new Foo<C2>();
Foo<C3> c3 = new Foo<C3>();

...
Error	CS0310	'C2' must be a non-abstract type with a public parameterless 
                constructor in order to use it as parameter 'T' in the generic type 
                or method 'Foo<T>'	

Error	CS0311	The type 'C3' cannot be used as type parameter 'T' 
                in the generic type or method 'Foo<T>'. There is no implicit reference 
                conversion from 'C3' to 'IQux'.	
\end{verbatim}
\end{scriptsize}

W powyższym przykładzie, tylko {\tt C1} spełnia oba ograniczenia typowe - ma bezargumentowy konstruktor i
implementuje oczekiwany interfejs. Klasa {\\ C2} nie ma bezargumentowanego konstruktora, a {\tt C3} nie
implementuje wskazanego interfejsu. 

Typy generyczne dobrze poddają się refleksji, dla dowolnego typu można za pomocą refleksji sprawdzić
czy jest to typ generyczny, czy jest otwarty (możliwy do ukonkretnienia) czy ukonkretniony, można
również konwertować typy z otwartych do ukonkretnionych i w drugą stronę.

\begin{scriptsize}
\begin{verbatim}
public class Foo<T>
{
}

class Program
{
    static void Main(string[] args)
    {
        ReflectionDemo1();
        ReflectionDemo2();
        Console.ReadLine();
    }

    /// <summary>
    /// Pobranie otwartego typu generycznego przez refleksję 
    /// i ukonkretnienie go typem int
    /// </summary>
    static void ReflectionDemo1()
    {
        // typ generyczny
        Type foo_t = typeof(Foo<>);
        foreach (Type type in foo_t.GetGenericArguments())
        {
            Console.WriteLine("Argument typowy: " + type.Name);
        }

        // ukonkretnienie
        Type foo_int = foo_t.MakeGenericType(typeof(int));

        object ofi = Activator.CreateInstance(foo_int);

        Console.WriteLine(ofi.GetType().ToString());
    }

    /// <summary>
    /// "Otwarcie" ukonkretnionego typu generycznego
    /// </summary>
    static void ReflectionDemo2()
    {
        Type foo_int = typeof(Foo<int>);

        Type foo_t = foo_int.GetGenericTypeDefinition();

        Console.WriteLine(foo_t.ToString());
    }
}

...
Argument typowy: T
ConsoleApplication18.Foo`1[System.Int32]
ConsoleApplication18.Foo`1[T]
\end{verbatim}
\end{scriptsize}

\subsubsection{Ograniczenia typowe w praktyce}

Z bagażem wiedzy na temat ograniczeń typowych warto wrócić do rzeczywistego przykładu, który był punktem wyjścia
do dyskusji, do przykładu klasy {\tt Vector<T>} i przekonać się na ile mechanizm ograniczeń typowych rzeczywiście
pomaga poradzić sobie w takim przypadku.

Możliwość zdefiniowania dla klasy przeciążonego operatora {\tt +}, która rozwiązuje przypadek dla własnej klasy 
{\tt Person} dla klasy szablonowej w C++, tu w języku C\# w ogóle nie ma zastosowania - nie ma bowiem takiego
ograniczenia typowego, które pozwoliłoby wymagać dla klasy zdefiniowanego operatora {\tt +}. Szczególnym przypadkiem
są operatory porównania ({\tt <}), ponieważ w bibliotece standardowej istnieje odpowiadający im interfejs ({\tt IComparable}), 
który w dodatku jest implementowany przez szereg klas z biblioteki standardowej.

Gdyby więc w klasie {\tt Vector} pojawiła się hipotetyczna metoda w której implementacji użyty był operator porównania

\begin{scriptsize}
\begin{verbatim}
public class Vector<T>
{
    ...
    public bool ShorterOnX( Vector<T> v )
    {
        return this.X < v.X;
    }    
}
\end{verbatim}
\end{scriptsize}

(należy zwrócić uwagę że w tej hipotetycznej metodzie nie porównuje się długości wektorów wyliczanych
zgodnie z regułami algebry, bo do takiego wyliczenia znów potrzebne byłyby operatory matematyczne)

to taki szczególny kod zależny od operatora {\tt <} można zamienić na poprawny

\begin{scriptsize}
\begin{verbatim}
public class Vector<T>
    where T : IComparable
{
    ...
    public bool ShorterOnX( Vector<T> v )
    {
        return this.X.CompareTo( v.X ) < 0;
    }    
}
\end{verbatim}
\end{scriptsize}

wykorzystujący zakładaną semantykę działania operacji {\tt CompareTo}. W takim przypadku klasa generyczna
dawałaby się ukonkretnić zarówno typami wbudowanymi (np. {\tt int} - ponieważ {\tt int} implementuje interfejs
{\tt IComparable}) jak i własnymi (pod warunkiem zaimplementowania w nich interfejsu {\tt IComparable}).

Ale co z operatorem {\tt +} w wyrażeniu {\tt this.X + v.X}? Jak już powiedziano, operator {\tt +} 
nie posiada odpowiadającego mu interfejsu w bibliotece standardowej.

\subsubsection{Abstrakcja operacji jako wymagania danego typu}

Jedno z możliwych rozwiązań szłoby więc w stronę zdefiniowania własnego interfejsu na wzór {\tt IComparable}, który
reprezentowałby abstrakcję operacji dodawania

\begin{scriptsize}
\begin{verbatim}
public interface IAddable<T>
{
    T AddTo( T item );
}
\end{verbatim}
\end{scriptsize}

Refaktoryzacja klasy {\tt Vector} uwzględniająca nowy interfejs i poprawnie implementująca dodawanie wektorów
wyglądałaby wtedy tak

\begin{scriptsize}
\begin{verbatim}
public class Vector<T>
    where T : IAddable<T>
{
    public Vector()
    {
    }

    public Vector( T x, T y )
    {
        this.X = x;
        this.Y = y;
    }

    public T X { get; set; }
    public T Y { get; set; }

    public Vector<T> Add( Vector<T> v )
    {
        return new Vector<T>(this.X.AddTo(v.X), this.Y.AddTo(v.Y));
    }
}
\end{verbatim}
\end{scriptsize}

To rozwiązanie, aczkolwiek poprawnie od strony technicznej, jest rozczarowująco niepraktyczne. Nie ma bowiem
żadnej, najmniejszej możliwości na przekonanie wbudowanego typu {\tt int} żeby zechciał implementować
nowo zdefiniowany interfejs {\tt IAddable} - klasy biblioteki standardowej są bowiem zamknięte na takie modyfikacje.
Rozwiązanie polegające na wyniesieniu operacji {\tt +} do osobnego interfejsu na typie {\tt T} nie da się więc
zastosować w żadnym innym przypadku niż do implementacji wektora własnych typów. A ponieważ typy proste takie jak
{\tt int} są zamknięte również na dziedziczenie po nich, to rozwiązanie można potraktować wyłącznie jako ciekawostkę.

\subsubsection{Abstrakcja operacji jako zobowiązanie typu pomocniczego}

Skoro więc nie można zmusić istniejącego typu do implementowania abstrakcji operacji dodawania, można tę operację
wynieść do całkowicie osobnej abstrakcji. Inaczej mówiąc, można zrobić z interfejsem {\tt IAddable} czym dla interfejsu
{\tt IComparable} jest {\tt IComparer} - wynieść operację poza typ do którego ona się odnosi.

\begin{scriptsize}
\begin{verbatim}
public interface IAdder<T>
{
    T Add(T t1, T t2);
}
\end{verbatim}
\end{scriptsize}

Klasa {\tt Vector} przestaje sama implementować interfejs {\tt IAddable}, zamiast tego wymaga ona {\bf zewnętrznej}
implementacji interfejsu dodającego wartości. Technicznie nie ma znaczenia jak ten zewnętrzny obiekt będzie
dostarczony, może być wymaganym argumentem konstruktora, może być dostarczony jako argument metody. 

\begin{scriptsize}
\begin{verbatim}
public class Vector<T>
{
    public Vector()
    {
    }

    public Vector(T x, T y)
    {
        this.X = x;
        this.Y = y;
    }

    public T X { get; set; }
    public T Y { get; set; }

    public Vector<T> Add(Vector<T> v, IAdder<T> adder)
    {
        return new Vector<T>(adder.Add(this.X, v.X), adder.Add(this.Y, v.Y));
    }
}
\end{verbatim}
\end{scriptsize}

W takiej wersji klasa generycznego wektora poradzi sobie z dodawaniem wektorów dowolnego typu, o ile tylko
zostanie z zewnątrz zasilona implementacją "dodawacza":

\begin{scriptsize}
\begin{verbatim}
public class IntAdder : IAdder<int>
{
    public int Add( int t1, int t2 )
    {
        return t1 + t2;
    }
}
...
Vector<int> v1 = new Vector<int>(1, 0);
Vector<int> v2 = new Vector<int>(2, 1);

Vector<int> v3 = v1.Add(v2, new IntAdder());
\end{verbatim}
\end{scriptsize}

To dużo lepsze rozwiązanie niż poprzednie, pozostawia jednak pewien niedosyt "stylistyczny" - to nie wygląda zbyt
elegancko kiedy wydawałoby się prosta klasa wektora nie realizuje sama z siebie tak prostej rzeczy jak dodawanie, tylko
musi operację dodawania delegować do zewnętrznego obiektu, który z kolei musi być jakoś dostarczony. Mówiąc inaczej,
chciałoby się mimo wszystko móc pisać

\begin{scriptsize}
\begin{verbatim}
Vector<int> v3 = v1.Add(v2);
\end{verbatim}
\end{scriptsize}

raczej niż

\begin{scriptsize}
\begin{verbatim}
Vector<int> v3 = v1.Add(v2, new IntAdder());
\end{verbatim}
\end{scriptsize}

Istnieją co najmniej dwa rozwiązania. 

\subsubsection{Programowanie dynamiczne}

Pierwsze rozwiązanie polega na przerzuceniu ciężaru wykonania operacji {\tt +} na czas {\bf wykonania} programu, a nie na 
czas kompilacji. Szerzej ta możliwość zostanie przedstawiona w kolejnych rozdziałach, tu ograniczymy się do jej zademonstrowania.

\begin{scriptsize}
\begin{verbatim}
public class Vector<T>
{
    public Vector()
    {
    }

    public Vector(T x, T y)
    {
        this.X = x;
        this.Y = y;
    }

    public T X { get; set; }
    public T Y { get; set; }

    public Vector<T> Add(Vector<T> v)
    {
        dynamic _x = this.X;
        _x += v.X;

        dynamic _y = this.Y;
        _y += v.Y;

        return new Vector<T>((T)_x, (T)_y);
    }
}
\end{verbatim}
\end{scriptsize}

Cały ciężar rozwiązania spoczywa teraz na środowisku uruchomieniowym - dla kodu w którym zmienna jest typu 
{\tt dynamic} kompilator wygeneruje kod, który każde wyrażenie zawierające tę zmienną będzie próbował "wiązać"
w trakcie działania programu (przez "wiązać" rozumiemy tu: wyszkukiwać czy aktualny typ zmiennej ma metody
które wyrażenie próbuje wywołać lub czy implementuje inne wymagania, takie jak wymaganie istnienia operatora{\tt +}).

Taki kod ma dwie wady:
\begin{itemize}
\item jest wolniejszy, dynamiczne wiązanie spowalnia kod od kilku do kilkudziesięciu razy
\item może powodować wyjątki dopiero w trakcie działania, przez błędy niewykrywalne w trakcie kompilacji
\end{itemize}

Istotnie, tak zaimplementowany wektor tym razem bezbłędnie zadziała dla liczb całkowitych

\begin{scriptsize}
\begin{verbatim}
Vector<int> v1 = new Vector<int>(1, 0);
Vector<int> v2 = new Vector<int>(2, 1);

Vector<int> v3 = v1.Add(v2);
\end{verbatim}
\end{scriptsize}

ale nie zadziała dla klasy niedostarczającej implementacji operacji dodawania, co więcej, wyjątek pojawia się 
dopiero w trakcie działania programu

\begin{scriptsize}
\begin{verbatim}
public class Person
{
    public string Name;

    public Person() { }
    public Person(string Name)
    {
        this.Name = Name;
    }
}
\end{verbatim}
\end{scriptsize}

Tu próba uruchomienia kodu

\begin{scriptsize}
\begin{verbatim}
Vector<Person> p1 = new Vector<Person>(new Person("Jan"), new Person("Kowalski"));
Vector<Person> p2 = new Vector<Person>(new Person("Zofia"), new Person("Malinowska"));

Vector<Person> p3 = p1.Add(p2);
\end{verbatim}
\end{scriptsize}

spowoduje zgłoszenie wyjątku w trakcie działania programu

\begin{scriptsize}
\begin{verbatim}
'Microsoft.CSharp.RuntimeBinder.RuntimeBinderException' occurred in System.Core.dll

Additional information: Nie można zastosować operatora „+=” do argumentów operacji 
typu „Person” lub „Person”.
\end{verbatim}
\end{scriptsize}

Szczęśliwie, remedium polega na dodaniu implementacji przeciążonego operatora dla typu

\begin{scriptsize}
\begin{verbatim}
public class Person
{
    public string Name;

    public Person() { }
    public Person(string Name)
    {
        this.Name = Name;
    }

    public static Person operator+(Person p1, Person p2)
    {
        return new Person(p1.Name + p2.Name);
    }
}
\end{verbatim}
\end{scriptsize}

po którym to dodaniu, wynik dodawania wektorów jest już poprawny.

\subsubsection{Lokalna fabryka "dodawaczy"}

Drugie rozwiązanie polega na dostarczeniu klasy gromadzącej wiedzę na temat wszystkich możliwych "dodawaczy"
dla wszystkich możliwych typów. Klasa wektora nie będzie już wymagała zasilenia jej zewnętrzną informacją, bo
sama będzie wiedziała gdzie tej informacji szukać. Fabryka zostanie wyposażona w podstawową wiedzę o
dodawaniu kilku wbudowanych typów i udostępni interfejs do rejestrowania kolejnych. Zastosowany tu wzorzec projektowy
w inżynierii oprogramowania nosi nazwę {\bf Factory}.

Jako że rozwiązanie wykorzystuje kilka mechanizmów, które do tej pory nie były prezentowane, zachęca się Czytelnika do
powrotu do tej części podręcznika po lekturze kolejnych rozdziałów, w których nowe elementy zostaną szczegółowo
omówione.

\begin{scriptsize}
\begin{verbatim}
public class AdderFactory
{
    private static Dictionary<Type, object> _adders = 
        new Dictionary<Type, object>();           // 0

    static AdderFactory()
    {
        _adders.Add(typeof(int), new IntAdder()); // 1
    }

    public static void RegisterAdder<T>( IAdder<T> adder )
    {
        _adders.Add(typeof(T), adder);     // 2
    }

    public IAdder<T> CreateAdder<T>()
    {
        foreach ( Type t in _adders.Keys ) // 3
        {
            if (t == typeof(T))            // 4
                return (IAdder<T>)_adders[t];
        }

        throw new ArgumentException(
            string.Format("Brak obsługi logiki dodawania dla typu {0}",
                typeof(T).Name) );
    }
}

public class Vector<T>
{
    public Vector()
    {
    }

    public Vector(T x, T y)
    {
        this.X = x;
        this.Y = y;
    }

    public T X { get; set; }
    public T Y { get; set; }

    public Vector<T> Add(Vector<T> v)
    {
        IAdder<T> adder = new AdderFactory().CreateAdder<T>(); // 5

        return new Vector<T>(adder.Add( this.X, v.X), adder.Add( this.Y, v.Y ));
    }
}
\end{verbatim}
\end{scriptsize}

Klasa fabryki przechowuje statyczną mapę, przyporządkowującą typom klasy realizujące algorytmy dodawania
dla nich ({\tt // 0}). Wykorzystano tu nieomawianą do tej pory biblioteczną kolekcję generyczną {\tt Dictionary},
w której kluczami są typy, a wartościami - klasy implementujace interfejs {\tt IAdder}. Ponieważ ograniczenie typowe
nie może zmieniać się dla każdego elementu słownika, jako typ wartości musi być tu wybrany typ {\tt object}.

Dalej, klasa fabryki rejestruje znane jej implementacje (tu: tylko jedną ale łatwo to rozszerzyć, {\tt // 1}) oraz
przygotowuje możliwość rejestrowania kolejnych ({\tt // 2}). 

Najistotniejsza jest tu metoda {\tt CreateAdder} tworzenia instancji obiektu {\tt IAdder<T>} dla zadanego {\tt T}.
Metoda ta iteruje po zarejestrowanych implementacjach, szuka takiej której klucz odpowiada żądanemu typowi ({\tt // 3}) i 
zwraca zarejestrowany obiekt rzutując go na oczekiwany interfejs ({\tt // 4}).

Sama klasa wektora używa fabryki do wydobycia instancji obiektu dodającego wartości i używa jej ({\tt 5}).

Takie rozwiązanie jest najbardziej eleganckie ze wszystkich do tej pory przedstawionych, choć nadal ma pewne wady.

Zadziała natychmiast dla wektorów liczb:

\begin{scriptsize}
\begin{verbatim}
Vector<int> v1 = new Vector<int>(1, 0);
Vector<int> v2 = new Vector<int>(2, 1);

Vector<int> v3 = v1.Add(v2);
\end{verbatim}
\end{scriptsize}

ale dla wektorów wartości własnego typu zgłosi wyjątek w czasie działania programu

\begin{scriptsize}
\begin{verbatim}
Vector<Person> p1 = new Vector<Person>(new Person("Jan"), new Person("Kowalski"));
Vector<Person> p2 = new Vector<Person>(new Person("Zofia"), new Person("Malinowska"));

Vector<Person> p3 = p1.Add(p2);
...
An unhandled exception of type 'System.ArgumentException' occurred in ConsoleApplication.exe

Additional information: Brak obsługi logiki dodawania dla typu Person
\end{verbatim}
\end{scriptsize}

Tym razem należy fabrykę nauczyć tego na czym polega dodawanie obiektów typu {\tt Person} przez 
jednokrotne zarejestrowanie odpowiedniej logiki w fabryce

\begin{scriptsize}
\begin{verbatim}
public class PersonAdder : IAdder<Person>
{
    public Person Add( Person p1, Person p2 )
    {
        return new Person(p1.Name + p2.Name);
    }
}
...
AdderFactory.RegisterAdder(new PersonAdder());

Vector<Person> p1 = new Vector<Person>(new Person("Jan"), new Person("Kowalski"));
Vector<Person> p2 = new Vector<Person>(new Person("Zofia"), new Person("Malinowska"));

Vector<Person> p3 = p1.Add(p2);
\end{verbatim}
\end{scriptsize}

\subsubsection{Wykonywanie kodu generycznego}

Nawet jeśli typy generyczne są w pełni kompilowane do CIL, z punktu widzenia środowiska uruchomieniowego w którym
kod jest wykonywany, typ generyczny nie może występować w postaci otwartej. Powstaje więc pytanie w jaki sposób
kompilator JIT obsługuje typy generyczne.

Okazuje się, że z punktu widzenia uruchamianego kodu, ważne są wyłącznie ukonkretnienia:
\begin{itemize}
\item dla każdego ukonkretnienia typem prostym środowisko tworzy nowy, odrębny typ uruchomieniowy. Przykładowo, w przypadku
{\tt Vector<T>} oznacza to że dwa różne wystapienia, {\tt Vector<int>} i {\tt Vector<double>} spowodują powstanie
dwóch różnych typów uruchomieniowych, w których {\tt T} jest zastąpione odpowiednim typem prostym
\item wszystkie ukonkretnienia typami referencyjnymi współdzielą jedną i tę samą instancję typu uruchomieniowego.
Przykładowo, dla wystąpień {\tt Vector<Person>} i {\tt Vector<Account>} będą po skompilowaniu CIL do kodu natywnego
przez kompilator JIT obsługiwane jednym i tym samym typem uruchomieniowym
\end{itemize}