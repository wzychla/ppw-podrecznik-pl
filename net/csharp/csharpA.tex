\subsection{Kolekcje generyczne - {\tt System.Collections.Generic}}

Wraz z wyposażeniem języka w mechanizm typów generycznych, omówiony wcześniej podsystem kolekcji niegenerycznych, 
{\tt System.Collections}, został rozszerzony o wsparcie dla kolekcji generycznych. W ramach prezentacji podsystemu zwrócimy
uwagę na kolekcje, które są generycznymi odpowiednikami tych kolekcji niegenerycznych, które już omówiono:

\begin{center}
$$\begin{array}{ll}
\mbox{Kolekcja generyczna} & \mbox{Odpowiadająca kolekcja niegeneryczna}  \\
\hline 
{\tt List<T>} & {\tt ArrayList} \\
{\tt Dictionary<T>} & {\tt Hashtable} \\
{\tt Stack<T>} & {\tt Stack} \\
{\tt Queue<T>} & {\tt Queue} 
\end{array}$$
\end{center}

Biblioteka kolekcji generycznych to bardzo oczywiste zastosowanie mechanizmu typów generycznych. Dzięki konieczności
ukonkretnienia kolekcji typem elementu, pomyłka w wyniku której do kolekcji generycznej trafiłby element niewłaściwego typu
jest wykrywana w trakcie kompilacji, w odróżnieniu od kolekcji niegenerycznej gdzie ewentualne odkrycie takiego problemu
możliwe jest dopiero w trakcie działania programu i to dopiero w miejscu, w którym następuje odczyt danych z kolekcji.

Porównajmy kod który powoduje wyjątek w trakcie działania

\begin{scriptsize}
\begin{verbatim}
class Program
{
    static void Main(string[] args)
    {
        ArrayList alist = new ArrayList();
        alist.Add( 1 );
        alist.Add( "2" ); // <- czy na pewno?

        foreach ( int i in alist ) // <- wartości muszą być int
        {
            Console.WriteLine(i);
        }
    }
}
...
An unhandled exception of type 'System.InvalidCastException' occurred in ConsoleApplication.exe
\end{verbatim}
\end{scriptsize}

z kodem który się nawet nie skompiluje

\begin{scriptsize}
\begin{verbatim}
class Program
{
    static void Main(string[] args)
    {
        List<int> alist = new List<int>();
        alist.Add(1);
        alist.Add("2");

        foreach ( int i in alist )
        {
            Console.WriteLine(i);
        }
    }
}
...
Error	CS1503	Argument 1: cannot convert from 'string' to 'int'	
\end{verbatim}
\end{scriptsize}

Odpowiednikiem kolekcji nie ograniczającej typu elementu podczas kompilacji byłby {\tt List<object>} aczkolwiek
w praktyce bardzo rzadko zachodzi potrzeba takiego luźnego ukonkretniania kolekcji, może to wręcz świadczyć
o jakimś problemie projektowym. W sytuacji, w której typy obiektów w kolekcji generycznej nie są bezpośrednio powiązane
w hierarchii dziedziczenia, zawsze istnieje możliwość ukonkretniania kolekcji generycznej interfejsem, zamiast konkretnym typem:

\begin{scriptsize}
\begin{verbatim}
class Program
{
    static void Main(string[] args)
    {
        List<I> alist = new List<I>();
        alist.Add(new A());
        alist.Add(new B());

        foreach ( I i in alist )
        {
            // ...
        }
    } 
}

public interface I { }

public class A : I { }

public class B : I { }
\end{verbatim}
\end{scriptsize}

Podobnie zachowuje się {\tt Dictionary<T, K>} który jest kolekcją asocjacyjną

\begin{scriptsize}
\begin{verbatim}
static void Main(string[] args)
{
    Dictionary<int, string> dict = new  Dictionary<int, string>();
    dict.Add(1, "foo");
    dict.Add(2, "bar");

    foreach ( KeyValuePair<int, string> i in dict )
    {
        // ...
    }

    foreach ( int i in dict.Keys )
    {
        // ...
    }

    foreach ( string s in dict.Values )
    {
        // ...
    }
} 
\end{verbatim}
\end{scriptsize}

Z uwagi na omówioną wcześniej charakterystykę typów generycznych, 
we współczesnym programowaniu w C\# w zasadzie sporadycznie korzysta się z kolekcji niegenerycznych. Jest to
najbardziej jaskrawe w przypadku prostych kolekcji ukonkretnianych typami prostymi ({\tt List<int>}), które przez
sposób translacji kodu generycznego do kodu natywnego przez kompilator JIT działają zwyczajnie szybciej niż
ich niegeneryczne odpowiedniki ({\tt ArrayList}), w których każda wartość typu prostego musi być pakowana i odpakowywana
aby stać się wartością przekazywaną przez referencję ({\tt object}).
