\subsection{Kolekcje generyczne - {\tt System.Collections.Generic}}

Wraz z wyposażeniem języka w mechanizm typów generycznych, omówiony wcześniej podsystem kolekcji niegenerycznych, 
{\tt System.Collections}, został rozszerzony o wsparcie dla kolekcji generycznych. W ramach prezentacji podsystemu zwrócimy
uwagę na kolekcje, które są generycznymi odpowiednikami tych kolekcji niegenerycznych, które już omówiono:

\begin{center}
$$\begin{array}{ll}
\mbox{Kolekcja generyczna} & \mbox{Odpowiadająca kolekcja niegeneryczna}  \\
\hline 
{\tt List<T>} & {\tt ArrayList} \\
{\tt Dictionary<T>} & {\tt Hashtable} \\
{\tt Stack<T>} & {\tt Stack} \\
{\tt Queue<T>} & {\tt Queue} 
\end{array}$$
\end{center}

Biblioteka kolekcji generycznych to bardzo oczywiste zastosowanie mechanizmu typów generycznych. Dzięki konieczności
ukonkretnienia kolekcji typem elementu, pomyłka w wyniku której do kolekcji generycznej trafiłby element niewłaściwego typu
jest wykrywana w trakcie kompilacji, w odróżnieniu od kolekcji niegenerycznej gdzie ewentualne odkrycie takiego problemu
możliwe jest dopiero w trakcie działania programu i to dopiero w miejscu, w którym następuje odczyt danych z kolekcji.

Porównajmy kod który powoduje wyjątek w trakcie działania

\begin{scriptsize}
\begin{verbatim}
class Program
{
    static void Main(string[] args)
    {
        ArrayList alist = new ArrayList();
        alist.Add( 1 );
        alist.Add( "2" ); // <- czy na pewno?

        foreach ( int i in alist ) // <- wartości muszą być int
        {
            Console.WriteLine(i);
        }
    }
}
...
An unhandled exception of type 'System.InvalidCastException' occurred in ConsoleApplication.exe
\end{verbatim}
\end{scriptsize}

z kodem który się nawet nie skompiluje

\begin{scriptsize}
\begin{verbatim}
class Program
{
    static void Main(string[] args)
    {
        List<int> alist = new List<int>();
        alist.Add(1);
        alist.Add("2");

        foreach ( int i in alist )
        {
            Console.WriteLine(i);
        }
    }
}
...
Error   CS1503  Argument 1: cannot convert from 'string' to 'int'   
\end{verbatim}
\end{scriptsize}

Odpowiednikiem kolekcji nie ograniczającej typu elementu podczas kompilacji byłby {\tt List<object>} aczkolwiek
w praktyce bardzo rzadko zachodzi potrzeba takiego luźnego ukonkretniania kolekcji, może to wręcz świadczyć
o jakimś problemie projektowym. W sytuacji, w której typy obiektów w kolekcji generycznej nie są bezpośrednio powiązane
w hierarchii dziedziczenia, zawsze istnieje możliwość ukonkretniania kolekcji generycznej interfejsem, zamiast konkretnym typem:

\begin{scriptsize}
\begin{verbatim}
class Program
{
    static void Main(string[] args)
    {
        List<I> alist = new List<I>();
        alist.Add(new A());
        alist.Add(new B());

        foreach ( I i in alist )
        {
            // ...
        }
    } 
}

public interface I { }

public class A : I { }

public class B : I { }
\end{verbatim}
\end{scriptsize}

Podobnie zachowuje się {\tt Dictionary<T, K>} który jest kolekcją asocjacyjną

\begin{scriptsize}
\begin{verbatim}
static void Main(string[] args)
{
    Dictionary<int, string> dict = new  Dictionary<int, string>();
    dict.Add(1, "foo");
    dict.Add(2, "bar");

    foreach ( KeyValuePair<int, string> i in dict )
    {
        // ...
    }

    foreach ( int i in dict.Keys )
    {
        // ...
    }

    foreach ( string s in dict.Values )
    {
        // ...
    }
} 
\end{verbatim}
\end{scriptsize}

Z uwagi na omówioną wcześniej charakterystykę typów generycznych, 
we współczesnym programowaniu w C\# w zasadzie sporadycznie korzysta się z kolekcji niegenerycznych. Jest to
najbardziej jaskrawe w przypadku prostych kolekcji ukonkretnianych typami prostymi ({\tt List<int>}), które przez
sposób translacji kodu generycznego do kodu natywnego przez kompilator JIT działają zwyczajnie szybciej niż
ich niegeneryczne odpowiedniki ({\tt ArrayList}), w których każda wartość typu prostego musi być pakowana i odpakowywana
aby stać się wartością przekazywaną przez referencję ({\tt object}).

\subsubsection{Typy funkcyjne}
\label{typy_funkcyjne_2_0}

Wprowadzenie generyczności pozwoliło na wprowadzenie do interfejsu biblioteki standardowej interesujących typów pozwalających na programowanie w stylu funkcyjnym.
Rozszerzenia te mają związek z możliwością nadania nazw typom funkcyjnym (omówiony już mechanizm {\tt delegate}) w wersji generycznej.

Ponieważ najwięcej interesujących rozszerzeń otrzymała biblioteczna klasa {\tt List<T>} i to na jej przykładzie warto te rozszerzenia omówić.

\paragraph{Funkcje anonimowe}

Przed omówieniem rozszrzeń funkcyjnych dla języka, warto również zwrócić uwagę na wprowadzoną do języka możliwość definiowania funkcji w sposób anonimowy, czyli
bez potrzeby nadawania im jawnej nazwy.

Rozważmy przykład sygnatury funkcyjnej i metody przyjmującej funkcję jako argument:

\begin{scriptsize}
\begin{verbatim}
public delegate int TheDelegate(int n);

...

public void HighOrderFunction( TheDelegate f )
{
    // f jest funkcją, można ją więc wywołać
    // ale jak ją przekazać jako argument?
    int result = f( 1 );
}

...

HighOrderFunction( ?? );
\end{verbatim}
\end{scriptsize}

We wcześniejszych wersjach języka, przekazanie funkcji do funkcji jako argumentu wiązało się z koniecznością zdefiniowania takiej funkcji w sposób jawny:

\begin{scriptsize}
\begin{verbatim}
public delegate int TheDelegate(int n);

...

public void HighOrderFunction( TheDelegate f )
{
    // f jest funkcją, można ją więc wywołać
    int result = f( 1 );
}

public int DoubleArgument( int n )
{
    return n+n;
}

...

// wywołanie funkcji z funkcją jako argumentem
HighOrderFunction(new TheDelegate(Double));


\end{verbatim}
\end{scriptsize}

Od wersji 2.0 języka można nie tylko pominąć {\tt new} na typie funkcyjnym (typie delegata), które kompilator sam sobie doda

\begin{scriptsize}
\begin{verbatim}
HighOrderFunction(DoubleArgument);
\end{verbatim}
\end{scriptsize}

ale również - funkcji w ogóle nie trzeba definiować na głównym poziomie klasy. Zamiast tego można przekazać ciało funkcji anonimowej
wprost, w miejscu w którym spodziewany jest argument

\begin{scriptsize}
\begin{verbatim}
HighOrderFunction( delegate( int n ) {
    return n + n;
});
\end{verbatim}
\end{scriptsize}

Warto zwrócić uwagę na tę składnię. Kolejną rewolucję wnosi tu dopiero C\#3.0, który pozwala na definiowanie funkcji anonimowych za pomocą
tzw. {\em lambda wyrażeń}. Ta składnia zostanie omówiona w rozdziale~\ref{cs3}. 

\paragraph{Akcja - {\tt Action<T>}}

Typ funkcyjny {\tt Action<T>} oznaczaja funkcję która wykonuje jakąś akcję na jednym elemencie określonego typu i nie zwraca żadnych wyników:

\begin{scriptsize}
\begin{verbatim}
Action<int> action = delegate (int n)
{
    Console.WriteLine(n);
};

action(1);
\end{verbatim}
\end{scriptsize}

\paragraph{Wyszukiwanie - {\tt Predicate<T>}}

Typ funkcyjny {\tt Predicate<T>} to predykat, czyli funkcja przyjmująca obiekt danego typu jako argument i zwracająca wartość typu prawda/fałsz:

\begin{scriptsize}
\begin{verbatim}
Predicate<int> predicate = delegate (int n)
{
    return n < 5;
};

Console.WriteLine(predicate(1));
Console.WriteLine(predicate(6));
\end{verbatim}
\end{scriptsize}

\paragraph{Porównanie - {\tt Comparison<T>}}

Typ funkcyjny {\tt Comparison<T>} to funkcja przyjmująca dwa argumenty (nazwijmy je {\em lewy} i {\em prawy}) i zwracająca wynik:
\begin{itemize}
\item -1 jeśli lewy parametr jest {\bf mniejszy} niż prawy
\item 0 jeśli lewy i prawy są równe
\item 1 jeśli lewy jest {\bf większy} niż prawy
\end{itemize}

\begin{scriptsize}
\begin{verbatim}
Comparison<int> comparison = delegate (int x, int y)
{
    if (x < y) return -1;
    if (x == y) return 0;

    return 1;
};

Console.WriteLine(comparison(1, 2));
Console.WriteLine(comparison(1, 1));
Console.WriteLine(comparison(2, 1));
\end{verbatim}
\end{scriptsize}

\subsubsection{Wykorzystanie typów funkcyjnych}

Wrócmy do wspomnianego wyżej w~\ref{typy_funkcyjne_2_0} typu {\tt List<T>} i zobaczmy jak nowe typy funkcyjne wprowadzono do interfejsu tej klasy.
