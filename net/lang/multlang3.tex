\subsection{Łączenie kodu z różnych języków}

\subsubsection{Zasady łączenia kodu różnych języków}

Platforma .NET pozwala z niespotykaną wcześniej łatwością łączyć kod napisany w różnych językach.
Dowolny kompilator może produkować biblioteki kodu, które następnie mogą być używane z poziomu innych języków.
W poniższym przykładzie kod klasy napisanej w VB.NET jest wykorzystywany w programie napisanym w C\#.

\begin{scriptsize}
\begin{verbatim}
// CMainForm.vb
Imports System
Imports System.Drawing
Imports System.Windows.Forms

Namespace MainModule
	
  Public Class CMainForm
    Inherits System.Windows.Forms.Form

    Public Sub New
      MyBase.New()
    End Sub

    Protected Overrides Sub OnPaint( e As PaintEventArgs )
      Dim g as Graphics = e.Graphics
      Dim f as Font     = new Font( "Times New Roman", 48 )

      Dim sf as StringFormat = new StringFormat()

      sf.Alignment     = StringAlignment.Center
      sf.LineAlignment = StringAlignment.Center

      g.Clear( SystemColors.Control )
      g.DrawString( "VB.NET", f, Brushes.Black, _
          Me.Width / 2, Me.Height / 2, sf )


     End Sub
		
  End Class
	
End Namespace

// CExample.cs
using System;
using System.Windows.Forms;

using MainModule;

class CExample 
{
  public static void Main()
  {
    Application.Run( new CMainForm() );
  }  
}

C:\example>vbc.exe /target:library CMainForm.vb
Microsoft (R) Visual Basic .NET Compiler version 7.00.9466
for Microsoft (R) .NET Framework version 1.00.3705.288
Copyright (C) Microsoft Corporation 1987-2001. All rights reserved.


C:\example>csc.exe /r:CMainForm.dll CExample.cs
Microsoft (R) Visual C# .NET Compiler version 7.00.9466
for Microsoft (R) .NET Framework version 1.0.3705
Copyright (C) Microsoft Corporation 2001. All rights reserved.

\end{verbatim}
\end{scriptsize}

Aby możliwa była współpraca kodu napisanego w różnych językach, oczywistym wydaje się być wymaganie, aby
języki te spełniały pewne warunki. W przypadku języków projektowanych z myślą o platformie .NET 
sprawa jest prosta. Trudności pojawiają się w przypadku języków dostosowywanych do wymogów platformy .NET,
na przykład języków funkcjonalnych. Przykład kompilatora SML.NET pokazuje, że takie trudności można
z powodzeniem pokonywać.

Czy więc współistnienie wielu języków w obrębie jednej platformy oznacza, że z poziomu kodu C\# można wprost
wołać kod z na przykład SML.NET?

Otóż nie do końca tak jest. Aby języki mogły współpracować ze sobą, konieczne jest aby komunikacja odbywała
się za pomocą dość rygorystycznych reguł określanych przez specyfikację CTS (była już o tym mowa). Oznacza to, że
moduł SMLowy może dowolnie korzystać z możliwości SMLa, ale po to aby pobrać parametry i oddać wyniki
do modułu C\#owego, musi na przykład opakować je w klasy, o które rozszerzono składnię SMLa. Dzięki temu, że
częścią CTS jest definicja typów prostych, wymiana informacji między różnymi językami nie jest trudna: typy
proste przekazuje sie wprost, typy złożone opakowuje się w struktury lub klasy. 

Poniższy prosty przykład pokazuje jak klasę napisaną w C\# można wykorzystać w kodzie SML.NET.

\begin{scriptsize}
\begin{verbatim}
// pola.cs
namespace NExample
{
  public class CExample
  {
    // pola statyczne
    public static readonly string pole_statyczne_readonly = 
      System.String.Concat("SML.", "NET");
    public static int pole_statyczne = 23;

    // pola
    public int pole;

    public CExample( int n ) 
    { 
      pole = n; 
    }
  }
}

(* pola_demo.sml *)
structure pola_demo =
struct

fun main () =
let
  val c = NExample.CExample( 156 )
in
  print ("Pole statyczne readonly " ^ valOf(NExample.CExample.pole_statyczne_readonly) ^ "\n");
  print ("Pole statyczne " ^ Int.toString(!NExample.CExample.pole_statyczne) ^ "\n");
  NExample.CExample.pole_statyczne := 17;
  print ("Pole statyczne " ^ Int.toString(!NExample.CExample.pole_statyczne) ^ "\n");
  print ("Pole " ^ Int.toString(!(c.#pole)) ^ "\n");
  c.#pole := 77;
  print ("Pole " ^ Int.toString(!(c.#pole)) ^ "\n")

end

end

C:\Example>csc /nologo /t:library pola.cs
C:\Example>smlnet -reference:pola pola_demo
C:\Example>pola_demo.exe
Pole statyczne readonly SML.NET
Pole statyczne 23
Pole statyczne 17
Pole 156
Pole 77
\end{verbatim}
\end{scriptsize}

\subsubsection{Pułapki}

Podczas łączenia kodu napisanego w różnych językach programista może natknąć się na przeróżne problemy.
Jednym z najsubtelniejszych z nich jest problem poelgający na zbudowaniu różnych języków w oparciu
o inne modele obiektowe.

Rozważmy następujący przykład kodu w C\#.

\begin{scriptsize}
\begin{verbatim}
using System;
using System.Windows.Forms;

namespace CPulapka
{
  public class A 
  {
    public virtual void Q( int k )
    {
      Console.WriteLine( "A::Q(int)" );
    }  
  }

  public class B : A
  {
    public virtual void Q( double d )
    {
      Console.WriteLine( "B::Q(double)" );
    }  

    public static void Main()
    {
      B b = new B();
      b.Q( 1.0 ); // tu jest OK!
      b.Q( 1 );   // a tu?
    }
  }
}

C:\example>CExample.exe
B::Q(double)
B::Q(double)
\end{verbatim}
\end{scriptsize}

Taki a nie inny wynik działania programu może być w pierwszej chwili dość nieoczekiwany. W klasie {\bf A} 
zdefiniowano bowiem metodę {\bf Q(int)}, która wydaje się lepiej pasować do wywołania {\tt b.Q( 1 )} niż
przeciążona w klasie {\bf B} metoda {\bf Q( double )}. 

Kompilator C\# kieruje się jednak jednoznacznymi regułami dopasowania funkcji, określonymi
w specyfikacji języka. Reguły te mówią, że jeżeli możliwe jest dopasowanie parametrów do funkcji
zdefiniowanej w klasie bieżącej, to funkcja taka zostanie wywołana, mimo że w klasie bazowej może istnieć
funkcja, która w danym kontekście mogłaby być bardziej właściwa (zauważmy, że konwersja 
{\bf int}$\rightarrow${\bf double}, która jest konieczna aby wywołać funkcję {\bf B::Q(double)} jest
{\em gorsza} niż konwersja {\bf int}$\rightarrow${\bf int}, która miałaby miejsce, gdyby 
w {\tt b.Q(1)} wywołana była funkcja {\bf A::Q(int)}).

Najbardziej właściwe pytanie, które należałoby zadać w tym miejscu brzmi: a co się stanie, jeśli
w innym języku programowania reguły te wyglądają inaczej i spróbujemy połączyć kod takich dwóch
języków?

Cóż, przekonajmy się:

\begin{scriptsize}
\begin{verbatim}
// CExample.cs -> CExample.dll
using System;
using System.Windows.Forms;

namespace CPulapka
{
  public class A 
  {
    public virtual void Q( int k )
    {
      Console.WriteLine( "A::Q(int)" );
    }  
  }

  public class B : A
  {
    public virtual void Q( double d )
    {
      Console.WriteLine( "B::Q(double)" );
    }  
  }
}

// CTest.vb kompilowany z referencją do CExample.dll
Imports System
Imports System.Drawing
Imports System.Windows.Forms

Imports CPulapka

Module MainModule

  Sub Main
    Dim b as B
    b = new B()

    b.Q( 1.0 )
    b.Q( 1 )
  End Sub
	
End Module

C:\example>CTest.exe
B::Q(double)
A::Q(int)
\end{verbatim}
\end{scriptsize}

Wynik tekstu potwierdza, że wybierając funkcję do wywołania w danym kontekście, kompilator
danego języka kieruje się swoimi własnymi regułami. W tym przypadku kompilator języka VB.NET
do wywołania {\tt b.Q(1)} dopasował funkcję {\bf A::Q(int)} w przeciwieństwie do kompilatora C\#, który
(jak widzieliśmy) w tym samym przypadku wybrałby funkcję {\bf B::Q(double)}.

Przykład ten jest bardzo pouczający. Świadczy on o tym, że mimo możliwości integracji w obrębie jednej
platformy uruchomieniowej, języki programowania mogą zachowywać dużą dozę niezależności. W końcu gdyby
każdy język trzeba było "na siłę" dopasować do jakichś reguł, zmieniając jednocześnie jego semantykę,
to cała idea .NET byłaby niewiele warta - oznaczałaby bowiem powstanie tak naprawdę jednego sposobu
"ewaluacji" programów, tyle że ubranego w składnie innych języków. W chwili obecnej zaś, programista 
chcący skorzystać z możliwości łączenia kodu różnych języków musi być po prostu świadomy możliwych
problemów z tym związanych - problemów, podkreślmy, natury dość głębokiej i wynikającej z dużych
różnic koncepcyjnych pomiędzy językami i ich modelami obiektowymi. Od strony czysto "technicznej"
platforma .NET daje wyjątkową możliwość bezproblemowego łączenia kodu dowolnych języków programowania.