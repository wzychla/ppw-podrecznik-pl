\section{Ciekawostki .NET}

\subsection{Błąd odśmiecania we wczesnych wersjach Frameworka}

Odśmiecacz, jako kluczowy element środowiska uruchomieniowego, powinien radzić sobie w wielu
różnych sytuacjach. Ciekawostką jst fakt, że w pierwszej wersji .NET Frameworka odśmiecacz czasami nie 
potrafił radzić sobie z usuwaniem niepotrzebnej pamięci.

Przykładowy program próbował w pętli rezerwować sobie coraz większy fragment pamięci, 
zaczynając od bloku 10MB i zwiększając wielkość bloku o 10kB w każdej iteracji. Rezerwowany wewnątrz
pętli blok powinien być oznaczony jako nieużywany, tak się jednak nie działo. W konsekwencji program działał
tak długo, aż skończyła się dostępna w systemie pamięć, po czym kończył się z wyjątkiem {\bf Out of memory}.

Problem został usunięty w poprawce {\bf SP2} do Frameworka 1.0. Nie występuje we Frameworku 1.1.

\begin{scriptsize}
\begin{verbatim}
/* Wiktor Zychla, 2003 */
using System;

class Test
{
   public static void Main(string [] args)
   {
      for (int tries=0; tries < 200; tries++)
      {
         int  iNElements = 10000000 + tries*10000;
         
         try
         {
             byte [] aMyMem = new byte [iNElements-1];
         }
         
         catch (Exception e)
         {
             Console.WriteLine (e);
             Console.WriteLine (iNElements);
             return;
         }         
      }
   }
}
\end{verbatim}
\end{scriptsize}

\subsection{Dostęp do prywatnych metod klasy}

Mimo, że CLS dość mocno broni się przed dostępem do prywatnych składowych obiektów, 
istnieje sposób na dostęp do prytwatnych metod. Okazuje się, że przydaje się tu mechanizm
refleksji. Sposób ten, z oczywistych powodów, nie nadaje się do wydobywania informacji o
prywatnych polach klas. Metody są bowiem składowymi statycznymi i są częścią opisu klasy. 
Pola zaś są składowymi dynamicznymi, unikalnymi dla każdej nowej instancji obiektu.

\begin{scriptsize}
\begin{verbatim}
/* Wiktor Zychla, 2003 */
using System;
using System.Reflection;

namespace Example
{  

  // Klasa z metodą prywatną
  public class ClassA
  {
    private void metodaPrywatna()
    {
      Console.WriteLine( "Metoda prywatna." );
    }
  }

  public class CMain
  {
    static void Main()
    {
      ClassA classA = new ClassA();
      Type     type = classA.GetType();

      foreach (MethodInfo method in type.GetMethods(
               BindingFlags.Instance | BindingFlags.NonPublic))
      {
        if (method.Name == "metodaPrywatna")
        {
          method.Invoke( classA, new object[] {} );
        }
      }
    }
  }
}
\end{verbatim}
\end{scriptsize}

\subsection{Informacje o systemie}

\subsubsection{Informacje o konfiguracji systemu}

Informacje o konfiguracji systemu dostępne są dzięki propercjom i metodom klasy {\bf Environment}.
Istnieje m.in. możliwość uzyskania informacji:

\begin{description}
\item [CommandLine] - parametry linii poleceń
\item [CurrentDirectory] - bieżąca ścieżka 
\item [GetEnvironmentVariables] - zmienne systemowe
\item [GetFolderPath] - lokalizacja specjalnych folderów systemowych 
\item [GetLogicalDrives] - lista napędów logicznych w systemie
\item [MachineName] - nazwa komputera
\item [SystemDirectory] - folder systemowy
\item [Version] - wersja systemu
\end{description}

\subsubsection{Informacje o środowisku graficznym}

Dostęp do informacji o konfiguracji środowiska graficznego możliwy jest dzięki klasie {\bf SystemInformation}.
Dostępne są m.in. następujące informacje :

\begin{description}
\item [ComputerName] - nazwa komputera
\item [MonitorCount] - liczba monitorów
\item [MouseButtons] - liczba przycisków myszy
\item [MouseWheelPresent] - dodatkowa możliwość myszy (obecność kółka)
\end{description}

oraz mnóstwo propercji opisujących rozmiary obiektów graficznych: kursora, ikon, czcionki menu, rozmiaru
ekranu itd.

\subsubsection{Informacje o środowisku uruchomieniowym}

Istnieje również możliwość zasięgnięcia informacji o środowisku uruchomieniowym. Służy do tego
klasa {\bf RuntimeEnvironment} z biblioteki {\bf System.Runtime.InteropServices}. Udostępnia informacje:

\begin{description}
\item [GetRuntimeDirectory] - ścieżka, w której zainstalowano środowisko uruchomieniowe
\item [GetSystemVersion] - wersja środowiska uruchomieniowego
\end{description}

\subsection{Własny kształt kursora myszy}

W szczególnych przypadkach można zmienić kształt kursora myszy. Biblioteki .NET udostępniają klasę
{\bf Cursors}, która udostępnia większość typowych kursorów systemowych. Co jednak zrobić, gdy
typowe kształty nie wystarczają?

Otóż można skorzystać z możliwości przekształcenia dowolnej bitmapy na kursor myszy:

\begin{scriptsize}
\begin{verbatim}
...
// twórz bitmapę, można użyć już istniejącej
Bitmap b = new Bitmap( 30, 15 );
Graphics g = Graphics.FromImage ( b );
g.DrawString ( "NAPIS", this.Font, Brushes.Black, 0, 0 );

// uchwyt do ikony utworzonej z bitmapy
IntPtr ptr = b.GetHicon();

// zamień na ikonę lub kursor
Icon   i = Icon.FromHandle ( ptr );
Cursor c = new Cursor( ptr );

// przypisz kursor oknu
this.Cursor = c;
\end{verbatim}
\end{scriptsize}

\subsection{Własne kształty okien}

Windows potrafi wykorzystać maski bitowe do definiowana kształtów okien. W świecie .NET nieregularne
okna można definiować dzięki obiektom {\bf GraphicsPath}, na przykład:

\begin{scriptsize}
\begin{verbatim}
...
// elipsoidalny kształt okna
GraphicsPath gP = new GraphicsPath();

gP.AddEllipse ( 0, 0, this.Width, this.Height );
this.Region = new Region ( gP );
\end{verbatim}
\end{scriptsize}

\subsection{Podwójne buforowanie grafiki w GDI+}

W GDI+ istnieje możliwość włączenia automatycznego podwójnego buforowania wyświetlanej grafiki.
Obraz zyskuje dzięki temu na płynności, co jest szczególnie przydatne gdy obraz odświeżany jest
dość często.

\begin{scriptsize}
\begin{verbatim}
...
SetStyle(ControlStyles.UserPaint, true);
SetStyle(ControlStyles.AllPaintingInWmPaint, true);
SetStyle(ControlStyles.DoubleBuffer, true);
\end{verbatim}
\end{scriptsize}

\subsection{Sprawdzanie uprawnień użytkownika}

Jeśli podczas pracy aplikacji zachodzi potrzeba sprawdzenia, czy użytkownik ma uprawnienia
administratora można posłużyć się następującym kodem:

\begin{scriptsize}
\begin{verbatim}
using System.Security;
using System.Security.Permissions;
using System.Security.Principal;

...
WindowsPrincipal wid = new WindowsPrincipal( WindowsIdentity.GetCurrent());
bool         isAdmin = wid.IsInRole(WindowsBuiltInRole.Administrator);		
\end{verbatim}
\end{scriptsize}

\subsection{Ikona skojarzona z plikiem}

System operacyjny Windows z niektórymi rozszerzeniami plików kojarzy aplikacje, służące do ich
otwierania. Menedżery plików pokazując listę plików, pokazują też ikony skojarzone z plikami, czyli
ikony aplikacji służących do otwierania tych plików.

Biblioteki .NET nie udostępniają bezpośrednio funkcji do wydobywania ikon skojarzonych z plikami,
należy więc sięgnąć do Win32API:

\begin{scriptsize}
\begin{verbatim}
public class ExtractIcon
{
 [DllImport("Shell32.dll")]
 private static extern int  SHGetFileInfo
 (
  string pszPath,
  uint dwFileAttributes,
  out SHFILEINFO psfi,
  uint cbfileInfo,
  SHGFI uFlags
 );

 [StructLayout(LayoutKind.Sequential)]
 private struct SHFILEINFO
 {
  public SHFILEINFO(bool b)
  {
   hIcon=IntPtr.Zero;iIcon=0;dwAttributes=0;szDisplayName="";szTypeName="";
  }
  public IntPtr hIcon;
  public int iIcon;
  public uint dwAttributes;
  [MarshalAs(UnmanagedType.LPStr, SizeConst=260)]
  public string szDisplayName;
  [MarshalAs(UnmanagedType.LPStr, SizeConst=80)]
  public string szTypeName;
 };

 private ExtractIcon() {}

 private enum SHGFI
 {
  SmallIcon   = 0x00000001,
  LargeIcon   = 0x00000000,
  Icon    = 0x00000100,
  DisplayName   = 0x00000200,
  Typename   = 0x00000400,
  SysIconIndex  = 0x00004000,
  UseFileAttributes = 0x00000010
 }

 public static Icon GetIcon(string strPath, bool bSmall)
 {
  SHFILEINFO info = new SHFILEINFO(true);
  int cbFileInfo = Marshal.SizeOf(info);
  SHGFI flags;
  if (bSmall)
   flags = SHGFI.Icon|SHGFI.SmallIcon|SHGFI.UseFileAttributes;
  else
   flags = SHGFI.Icon|SHGFI.LargeIcon|SHGFI.UseFileAttributes;

  SHGetFileInfo(strPath, 256, out info,(uint)cbFileInfo, flags);
  return Icon.FromHandle(info.hIcon);
 }
}
\end{verbatim}
\end{scriptsize}

\subsection{WMI}

WMI (ang. {\em Windows Management Instrumentation}) jest mechanizmem umożliwiającym diagnozowanie
stanu komputera i systemu operacyjnego w jednorodny sposób: WMI udostępnia obiekt, który
interfejsem przypomina bazę danych. Aby uzyskać interesujące informacje, należy po prostu zadać
odpowiednie zapytanie w języku SQL, odwołując się do odpowiedniego obiektu WMI. 

Możliwości WMI są naprawdę duże. Zadając odpowiednie zapytanie można dowiedzieć się mnóstwo
szczegółów takich jak np. marka producenta płyty głównej, częstotliwość taktowania procesora, numer
wersji BIOSu i wiele, wiele innych.

WMI jest dostępne w Windows począwszy od Windows 2000. W Windows 98 musi być doinstalowane przez
użytkownika.

\begin{scriptsize}
\begin{verbatim}
using System;
using System.Management;
namespace WMI
{
  class CExample
  {
    static void Main(string[] args)
    {
      ManagementObjectSearcher query1 = 
        new ManagementObjectSearcher("SELECT * FROM win32_OperatingSystem") ;
      ManagementObjectCollection queryCollection1 = query1.Get();
      foreach( ManagementObject mo in queryCollection1 ) 
      {
        Console.WriteLine("Name : " + mo["name"].ToString());
        Console.WriteLine("Version : " + mo["version"].ToString());
        Console.WriteLine("Manufacturer : " + mo["Manufacturer"].ToString());
        Console.WriteLine("Computer Name : " +mo["csname"].ToString());
        Console.WriteLine("Windows Directory : " + mo["WindowsDirectory"].ToString());
      }                  

      query1 = new ManagementObjectSearcher("SELECT * FROM win32_Processor") ;
      queryCollection1 = query1.Get();
      foreach( ManagementObject mo in queryCollection1 ) 
      {
        Console.WriteLine(mo["Caption"].ToString());
        Console.WriteLine(mo["CurrentClockSpeed"].ToString());
      }                       

      query1 = new ManagementObjectSearcher("SELECT * FROM win32_BIOS") ;
      queryCollection1 = query1.Get();
      foreach( ManagementObject mo in queryCollection1 ) 
      {
        Console.WriteLine(mo["version"].ToString());
      }                                   
    }
  }
}
\end{verbatim}
\end{scriptsize}
