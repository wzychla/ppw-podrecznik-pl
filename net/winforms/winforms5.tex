\subsection{Menu}

\subsubsection{Tworzenie menu}

Tworzenie menu możliwe jest dzięki dwóm typom danych:
\begin{description}
\item [MainMenu] który służy do tworzenia menu dla okna dialogowego
\item [ContextMenu] który służy do tworzenia menu kontekstowych dla okien
\end{description}

\begin{scriptsize}
\begin{verbatim}
/* Wiktor Zychla, 2003 */
using System;
using System.Drawing;
using System.Windows.Forms;

namespace Example
{
  public class CMainForm : Form
  {   
    TextBox tb; 

    void InitMenus()
    {
      // MainMenu
      MainMenu mainMenu = new MainMenu();

      MenuItem mPlik = new MenuItem( "Plik" );
      MenuItem mPlikZakoncz = new MenuItem( "Zakończ" );
      mPlikZakoncz.Click += new EventHandler( mPlikZakoncz_Click );

      mPlik.MenuItems.Add( mPlikZakoncz );
      mainMenu.MenuItems.Add( mPlik );

      this.Menu = mainMenu; 

      // ContextMenu
      ContextMenu cMenu = new ContextMenu();
      MenuItem mZakoncz = new MenuItem( "Zakończ" );
      cMenu.MenuItems.Add( mZakoncz );
      
      this.tb.ContextMenu = cMenu;
    }

    void mPlikZakoncz_Click( object sender, EventArgs e )
    { 
      this.Close();
    } 

    public CMainForm()
    {
      tb = new TextBox();
      this.Controls.Add( tb );

      InitMenus();
    } 

    public static void Main()
    {
      Application.Run( new CMainForm() );
    }
  }
}
\end{verbatim}
\end{scriptsize}

Obiekt typu {\bf ContextMenu}, reprezentujący menu kontekstowe, ma jeszcze jedną interesującą
właściwość. Otóż ma on metodę {\bf Show}, która po prostu pokazuje menu kontekstowe przy
wskazanym oknie potomnym. Jest to bardzo wygodne wtedy, kiedy menu kontekstowe powinno zostać
ujawnione w jakiejś nietypowej sytuacji. 

Wyobraźmy sobie na przykład scenariusz, w którym każdemu elementowi drzewa {\bf TreeView} powinno odpowiadać
jakieś inne menu kontekstowe, zależne od tego, co zawiera wskazany element. Zamiast przywiązywać jakieś
konkretne menu kontekstowe do obiektu {\bf TreeView}, programista może po prostu przechwycić 
zdarzenie kliknięcia myszą w węzeł drzewa, sprawdzić czy kliknięto prawy klawisz myszy i dopiero
wtedy wykorzystać metodę {\bf Show} do pokazania odpowiedniego menu kontekstowego.

\subsubsection{Własne funkcje rysowania menu}

Wygląd menu można uatrakcyjnić dzięki możliwości określenia własnych funkcji odpowiedzialnych
za rysowanie elementów menu. Programista musi jedynie dodać funkcje obsługi zdarzeń {\bf DrawItem}, 
odpowiedzialnej za rysowanie i {\bf MeasureItem}, odpowiedzialnej za określanie obszaru zajmowanego przez
element menu\footnote{Kilku innym komponentom wizualnym również można zmieniać wygląd w taki sposób.}.

W poniższym przykładzie zmieniamy sposób rysowania tylko tych pozycji menu, które widoczne są
na pasku menu w oknie. W analogiczny sposób można jednak określić sposób rysowania pozycji rozwijalnych,
dodając na przykład możliwość rysowania ikon obok pozycji menu itp.

\begin{scriptsize}
\begin{verbatim}
/* Wiktor Zychla, 2003 */
using System;
using System.Drawing;
using System.Windows.Forms;

namespace WinForms_Dodatki
{
  public class C_XMainMenu : System.Windows.Forms.MainMenu 
  {
    private void topMenu_DrawItem(object sender, 
                                  System.Windows.Forms.DrawItemEventArgs e)
    {
      Rectangle mRect  = 
        new Rectangle(e.Bounds.X, e.Bounds.Y, e.Bounds.Width, e.Bounds.Height);
      Rectangle mRect2 = 
        new Rectangle(e.Bounds.X, e.Bounds.Y, e.Bounds.Width+1, e.Bounds.Height+1);

      if ( (e.State & DrawItemState.Selected) != 0 )  
      {
        e.Graphics.FillRectangle( new SolidBrush( SystemColors.Control ), mRect );
        e.Graphics.DrawRectangle( 
          new Pen( new SolidBrush( SystemColors.ControlDark ), 1 ), mRect );
      }
      else
      if ( (e.State & DrawItemState.HotLight) != 0 ) 
      {
        e.Graphics.FillRectangle( new SolidBrush( SystemColors.ControlLightLight ), mRect );
        e.Graphics.DrawRectangle( 
          new Pen( new SolidBrush( SystemColors.ControlDark ), 1 ), mRect );
      }
      else
      {
        e.Graphics.FillRectangle( new SolidBrush(SystemColors.Control), mRect2 );
      }
						
      MenuItem     mItem   = (MenuItem)sender;
      Font         mFont   = new Font( "MS Sans Serif", 10 );
      StringFormat sFormat = new StringFormat();
			
      sFormat.Alignment     = StringAlignment.Center;
      sFormat.LineAlignment = StringAlignment.Center;

      e.Graphics.DrawString(mItem.Text, mFont, new SolidBrush(Color.Black), mRect, sFormat );

      mFont.Dispose();
    }

    private void topMenu_MeasureItem(object sender, 
                            System.Windows.Forms.MeasureItemEventArgs e)
    {
      MenuItem     mItem = (MenuItem)sender;
      Font         mFont = new Font( "MS Sans Serif", 10 );

      SizeF sizeF  = e.Graphics.MeasureString( mItem.Text, mFont );			
      e.ItemWidth  = (int)sizeF.Width;

      mFont.Dispose();
    }

    public C_XMainMenu( Menu mMenu ) 
    {
      foreach ( MenuItem mItem in mMenu.MenuItems )
      {				
        MenuItem newMenuItem = mItem.CloneMenu();
				
        ApplyMenuProperties ( newMenuItem );
        this.MenuItems.Add ( newMenuItem );
      }			
    }
		
    private void ApplyMenuProperties( MenuItem mItem )
    {
      if ( IsTopMenu( mItem ) )
      {
        mItem.OwnerDraw = true;
        mItem.DrawItem += 
	  new System.Windows.Forms.DrawItemEventHandler(this.topMenu_DrawItem);
        mItem.MeasureItem += 
	  new System.Windows.Forms.MeasureItemEventHandler(this.topMenu_MeasureItem);				
      }

      foreach ( MenuItem subMenu in mItem.MenuItems )
        ApplyMenuProperties( subMenu );
    }

    private static bool IsTopMenu( MenuItem mItem )
    {
      if ( mItem.Parent == null ) return true;
      return mItem.Parent == mItem.GetMainMenu();
    }
  }
}
\end{verbatim}
\end{scriptsize}

Klasa {\bf C\_XMainMenu} określona jest tak, że w kodzie inicjującym menu należy po prostu napisać:

\begin{scriptsize}
\begin{verbatim}
MainMenu mainMenu;
...
C_XMainMenu cxMainMenu = new C_XMainMenu( mainMenu );
this.Menu = cxMainMenu; 
\end{verbatim}
\end{scriptsize}

\subsection{Schowek}

Dostęp do schowka systemowego możliwy jest dzięki obiektowi {\bf Clipboard}. W schowku można
umieścić dowolny obiekt lub sprawdzić czy znajduje się tam obiekt określonego typu.

Poniższy przykład umieszcza w schowku napis, a następnie wydobywa go stamtąd. Dane przekazane
do schowka w ten sposób są dostępne dla wszystkich aplikacji w systemie, podobnie dane
pobierane ze schowka mogą pochodzić z dowolnej aplikacji w systemie.

\begin{scriptsize}
\begin{verbatim}
/* Wiktor Zychla, 2003 */
using System;
using System.Drawing;
using System.Windows.Forms;

namespace Example
{
  public class CMainForm : Form
  {  
    public CMainForm() 
    {
      // umieść dane w schowku
      Clipboard.SetDataObject( "Tekst przesyłany do schowka", true );

      // wydobądź dane ze schowka
      IDataObject ido = Clipboard.GetDataObject();
      if ( ido.GetDataPresent( typeof( string ) ) )
      {
        string s = ido.GetData( typeof( string ) ) as string;
        MessageBox.Show( s );
      }
    }

    public static void Main()
    {    
      Application.Run( new CMainForm() );
    }
  }
}
\end{verbatim}
\end{scriptsize}

