\section{Procesy, wątki, synchronizacja}

\subsection{Tworzenie wątków i procesów}

Zadaniem systemu operacyjnego jest wykonywanie programów, przechowywanych najczęściej
na różnego rodzaju nośnikach. Z punktu widzenia systemu operacyjnego, {\bf program} to więc
nic więcej niż plik, w którym przechowywany jest obraz kodu wynikowego programu.

Program uaktywnia się w wyniku jawnego utworzenia przez system operacyjny {\bf procesu}, który
odpowiada obrazowi programu. W systemie Windows do tworzenia procesu służy funkcja:

\begin{scriptsize}
\begin{verbatim}
BOOL CreateProcess(

    LPCTSTR lpApplicationName,	// nazwa modułu wykonywalnego
    LPTSTR lpCommandLine,	// linia poleceń
    LPSECURITY_ATTRIBUTES lpProcessAttributes,	// atrybuty bezpieczeństwa procesu
    LPSECURITY_ATTRIBUTES lpThreadAttributes,	// atrybuty bezpieczeństwa wątku
    BOOL bInheritHandles,	// dziedziczenie uchwytów
    DWORD dwCreationFlags,	// dodatkowe flagi, np. priorytet
    LPVOID lpEnvironment,	// środowisko
    LPCTSTR lpCurrentDirectory,	
    LPSTARTUPINFO lpStartupInfo,	// własciwości startowe okna
    LPPROCESS_INFORMATION lpProcessInformation 	// zwraca informacje o procesie i wątku
   );
\end{verbatim}
\end{scriptsize}

Proces po załadowaniu do systemu nie wykonuje kodu, dostarcza jedynie przestrzeni adresowej {\em wątkom}.
To wątki są jednostkami, którym system przydziela czas procesora. Każdy proces w systemie ma niejawnie
utworzony jeden wątek wykonujący kod programu. Każdy następny wątek w obrębie jednego procesu musi
być utworzony {\em explicite}. 

Tworzenie wielu wątków w obrębie jednego procesu jest czasami bardzo przydatne. 
Wątki mogą na przykład przejmować na siebie długotrwałe obliczenia nie powodując "zamierania" całego
procesu. Ponieważ wątki współdzielą zmienne globalne procesu, możliwa jest niednoczesna praca wielu wątków na
jakimś zbiorze danych procesu. 

Podsumujmy związek pomiędzy procesami a wątkami:  
\begin{itemize}
  \item Proces nie wykonuje kodu, proces jest obiektem dostarczającym wątkowi przestrzeni adresowej,   
  \item Kod zawarty w przestrzeni adresowej procesu jest wykonywany przez wątek,   
  \item Pierwszy wątek procesu tworzony jest implicite przez system operacyjny, 
        każdy następny musi być utworzony explicite,   
  \item Wszystkie wątki tego samego procesu dzielą wirtualną przestrzeń adresową i 
        mają dostęp do tych samych zmiennych globalnych i zasobów systemowych.  
\end{itemize}
   
Do tworzenia dodatkowych wątków w obrębie jednego procesu służy funkcja:

\begin{scriptsize}
\begin{verbatim}
HANDLE CreateThread(

    LPSECURITY_ATTRIBUTES lpThreadAttributes,	// atrybuty bezpieczeństwa wątku
    DWORD dwStackSize,	// rozmiar stosu (0 - domyślny)
    LPTHREAD_START_ROUTINE lpStartAddress,	// wskaźnik na funkcję wątku
    LPVOID lpParameter,	// wskaźnik na argument
    DWORD dwCreationFlags,	// dodatkowe flagi 
    LPDWORD lpThreadId 	// zwraca identyfikator wątku
   );
\end{verbatim}
\end{scriptsize}

Po utworzeniu nowy wątek jest wykonywany równolegle z pozostałymi wątkami w systemie. 

\begin{scriptsize}
\begin{verbatim}
/* 
 * Tworzenie wątków
 */
#include <windows.h>
#include <stdio.h>
#include <stdlib.h>
  
DWORD WINAPI ThreadProc(LPVOID* theArg);  
   
int main(int argc, char *argv[])  
{  
  DWORD threadID;  
  DWORD thread_arg = 4;  
   
  HANDLE hThread = CreateThread( NULL, 0, 
                   (LPTHREAD_START_ROUTINE)ThreadProc, 
		   &thread_arg, 0, &threadID );  

  WaitForSingleObject( hThread, INFINITE );                                   
   
  return 0;  
}  

DWORD ThreadProc(LPVOID* theArg)  
{  
    DWORD timestoprint = (DWORD)*theArg;  
    for (int i = 0; i<timestoprint; i++)  
      printf("Witam %d \n", i);  
    return TRUE;   
}  
\end{verbatim}
\end{scriptsize}
  
Programista ma do dyspozycji kilka dodatkowych funkcji do manipulacji tworzonymi wątkami
i procesami, m.in.:

\begin{itemize}
\item Ustalanie priorytetu wątku.
\begin{scriptsize}
\begin{verbatim}
BOOL SetThreadPriority(

    HANDLE hThread,	// handle to the thread 
    int nPriority 	// thread priority level 
   );
\end{verbatim}
\end{scriptsize}

\item Ustalanie procesora na którym wykonuje się wątek w systemach wieloprocesorowych.
\begin{scriptsize}
\begin{verbatim}
DWORD SetThreadAffinityMask (

    HANDLE hThread,	// handle to the thread of interest
    DWORD dwThreadAffinityMask	// a thread affinity mask
   );
\end{verbatim}
\end{scriptsize}

\end{itemize}

\subsection{Synchronizacja wątków}

Sytuacja w której wiele wątków jednocześnie operuje na danych globalnych procesu może 
rodzić problemy. Wyobraźmy sobie bowiem dwa wątki mające dostęp do zmiennej globalnej.
Niech pierwszy z wątków wykonuje następujący kod :

\begin{scriptsize}
\begin{verbatim}
i = 0;	
// !
if ( i == 0 )
{
  ...
}
\end{verbatim}
\end{scriptsize}

a drugi:

\begin{scriptsize}
\begin{verbatim}
i = 1;
// *
if ( i == 1 )
{
  ...
}
\end{verbatim}
\end{scriptsize}

Jeśli któryś z wątków zostanie przez system operacyjny wywłaszczony w miejscu oznaczonym w kodzie znakiem "*",
drugi wątek stanie się aktywnym i wykona swoją instrukcję przypisania, po czym ponownie pierwszy wątek
stanie się aktywny, to operacja porównania zakończy się niepowodzeniem, najprawdopodobniej wbrew intencjom 
programisty. To czego potrzeba, aby unikać tego typu problemów, to jakaś forma kontroli nad przełączaniem
wątków przez system.

Win32API udostępnia 5 sposoby synchronizacji wątków. Są to: 
\begin{itemize}
  \item zdarzenia, 
  \item mutexy, 
  \item semafory, 
  \item sekcje krytyczne, 
  \item zegary oczekujące
\end{itemize}

Mechanizm sekcji krytycznej możliwy jest do wykorzystania tylko w obrębie 
jednego procesu (do synchronizacji wątków), jednak jest to metoda 
najszybsza i najwydajniejsza. Pozostałe metody mogą być stosowane również dla wielu procesów. 

\subsubsection{Zdarzenia}

Win32API umożliwia definiowanie własnych zdarzeń za pomocą funkcji 

\begin{scriptsize}
\begin{verbatim}
HANDLE CreateEvent(

    LPSECURITY_ATTRIBUTES lpEventAttributes,	// atrybuty bezpieczeństwa
    BOOL bManualReset,	// flaga ręcznego resetowania
    BOOL bInitialState,	// flaga początkowego stanu
    LPCTSTR lpName 	// nazwa
   );
\end{verbatim}
\end{scriptsize}

Każdy oczekujący wątek widzi zdarzenie jako pewną dwustanową flagę: zdarzenie jest zgłoszone 
albo odwołane. Za pomocą funkcji 

\begin{scriptsize}
\begin{verbatim}
BOOL SetEvent(

    HANDLE hEvent 	// uchwyt zdarzenia
   );
\end{verbatim}
\end{scriptsize}

informujemy system o zaistnieniu zdarzenia. 
Od tej pory zdarzenie jest zgłoszone i wszystkie wątki oczekujące do tej pory na jego 
zgłoszenie mogą wznowić działanie. Zdarzenie zostaje odwołane, kiedy zostanie wywołana 
funkcja 

\begin{scriptsize}
\begin{verbatim}
BOOL ResetEvent(

    HANDLE hEvent 	// uchwyt zdarzenia
   );
\end{verbatim}
\end{scriptsize}

Na zaistnienie wydarzenia w systemie wątki oczekują za pomocą 
funkcji 

\begin{scriptsize}
\begin{verbatim}
DWORD WaitForSingleObject(

    HANDLE hHandle,	        // uchwyt obiektu synchronizacji
    DWORD dwMilliseconds 	// czas oczekiwania (INFINITE czeka aż do 
                                // zajścia zdarzenia)
   );
\end{verbatim}
\end{scriptsize}

Zdarzenie utworzone z ustawioną flagą ręcznego odwoływania 
(CreateEvent(...,TRUE,...,...)) wymaga odwołania explicite (przez ResetEvent()), 
natomiast zdarzenie utworzone z flagą automatycznego odwoływania (CreateEvent(...,FALSE,...,...)) 
zostaje odwołane automatycznie po przepuszczeniu jednego wątku przez funkcję oczekującą. 

Warto również omówić działanie funkcji 

\begin{scriptsize}
\begin{verbatim}
BOOL PulseEvent(

    HANDLE hEvent 	// uchwyt zdarzenia
   );
\end{verbatim}
\end{scriptsize}

Otóż powoduje ona zgłoszenie zdarzenia, po czym natychmiastowe jego odwołanie. 
Działanie oczekujących wątków zależy od tego, czy zdarzenie jest odwoływane 
automatycznie czy ręcznie (patrz paragraf wyżej): jeśli zdarzenie odwoływane jest ręcznie, 
to funkcja PulseEvent() przepuszcza wszystkie wątki oczekujące w danej chwili na zdarzenie, 
po czym odwołuje zdarzenie, jeśli zaś zdarzenie odwoływane jest automatycznie, to funkcja 
PulseEvent() przepuszcza tylko jeden wątek z puli oczekujących w danej chwili 
wątków, po czym odwołuje zdarzenie. 

\begin{scriptsize}
\begin{verbatim}
/*
 * Wykorzystanie zdarzeń do synchronizacji wątków
 */
void main(void)  
{  
      HANDLE hThread[2];  
      DWORD threadID1, threadID2;  
      char szFileName="c:\\myfolder\\myfile.txt";  

      hEvent=CreateEvent(NULL, TRUE, FALSE, "FILE_EXISTS");  

      // tworzymy dwa wątki które czekają na utworzenie pliku  
      hThread[0]=CreateThread(NULL, 0, ThreadProc1, szFileName, 0, &threadID1);  
      hThread[1]=CreateThread(NULL, 0, ThreadProc2, szFileName, 0, &threadID1);  

      HANDLE hFile=CreateFile(szFileName, GENERIC_WRITE, 0, &security, . . .);  

      // kod wypełniający plik danymi np. WriteFile(...)  

      // sygnalizacja wątkom tego, że dane są gotowe  
      // wątki od ich utworzenia tylko na to czekały   
      SetEvent(hEvent);  
      WaitForMultipleObjects(2, hThread, TRUE, _czas_czekania_);  

      CloseHandle(hEvent);  
      CloseHandle(hFile);  
      CloseHandle(hThread[0]);  
      CloseHandle(hThread[1]);  
}  
   
DWORD ThreadProc1(LPVOID* arg)  
{  
      char szFileName = (char*)arg;  
      // tutaj wątek otwiera zdarzenie określone w module głównym  
      HANDLE hEvent = OpenEvent(SYNCHRONIZE, FALSE, "FILE_EXISTS");  
      // czeka na jego pojawienie się  
      WaitForSingleObject(hEvent, INFINITE);  
      // i czyta dane zapisane do pliku  
      HANDLE hAnswerFile = ::CreateFile(szFileName, GENERIC_READ, 0, NULL, 
                             OPEN_EXISTING, FILE_ATTRIBUTE_NORMAL, NULL);  

      // przetwarzaj dane 
      // ...
      return TRUE;  
}  
  
DWORD ThreadProc2(LPVOID* arg)  
{  
      char szFileName = (char*)arg;  
      // tutaj wątek otwiera zdarzenie określone w module głównym  
      HANDLE hEvent = OpenEvent(SYNCHRONIZE, FALSE, "FILE_EXISTS");  
      // czeka na jego pojawienie się  
      WaitForSingleObject(hEvent, INFINITE);  
      // i czyta dane zapisane do pliku  
      HANDLE hAnswerFile = ::CreateFile(szFileName, GENERIC_READ, 0, NULL, 
                             OPEN_EXISTING, FILE_ATTRIBUTE_NORMAL, NULL);  

      // przetwarzaj dane  
      // ...
      return TRUE;  
}  
\end{verbatim}
\end{scriptsize}

\subsubsection{Mutexy}
\label{subsection_mutexy}

Nazwa mutex pochodzi od angielskiego terminu mutually exclusive (wzajemnie wykluczający się). 
Mutex jest obiektem służącym do synchronizacji. Jego stan jest ustawiony jako "sygnalizowany", 
kiedy żaden wątek nie sprawuje nad nim kontroli oraz "niesygnalizowany" kiedy jakiś wątek 
sprawuje nad nim kontrolę. Synchronizację za pomocą mutexów realizuje się tak, 
że każdy wątek czeka na objęcie mutexa w posiadanie, zaś po zakończeniu operacji 
wymagającej wyłączności, wątek uwalnia mutexa. 

W celu stworzenia mutexa, wątek woła funkcję 

\begin{scriptsize}
\begin{verbatim}
HANDLE CreateMutex(

    LPSECURITY_ATTRIBUTES lpMutexAttributes,	// atrybuty bezpieczeństwa
    BOOL bInitialOwner,	// flaga własności przy tworzeniu
    LPCTSTR lpName 	// nazwa
   );
\end{verbatim}
\end{scriptsize}

W chwili tworzenia wątek może zażądać natychmiastowego prawa własności do mutexa. 
Inne wątki (nawet innych procesów) uzyskują uchwyt mutexa za pomocą funkcji 

\begin{scriptsize}
\begin{verbatim}
HANDLE OpenMutex(

    DWORD dwDesiredAccess,	// flaga dostępu
    BOOL bInheritHandle,	// uchwyt dziedziczony na procesy tworzone
                                // przez CreateProcess
    LPCTSTR lpName 	// nazwa
   );
\end{verbatim}
\end{scriptsize}

Następnie czekają na objęcie mutexa w posiadanie (za pomocą WaitForSingleObject()). 
Do uwalniania mutexów służy funkcja 

\begin{scriptsize}
\begin{verbatim}
BOOL ReleaseMutex(

    HANDLE hMutex 	// handle of mutex object  
   );
\end{verbatim}
\end{scriptsize}

Jeśli wątek kończy się bez uwalniania mutexów, które posiadał, takie mutexy uważa się za porzucone. 
Każdy czekający wątek może objąć takie mutexy w posiadanie, 
zaś funkcja czekająca na przydział mutexa (WaitForSingleObject(), jak widać bardzo uniwersalna funkcja) 
zwraca wartość WAIT\_ABANDONED. W takiej sytuacji warto zastanowić się, czy gdzieś nie wystąpił 
jakiś błąd (skoro wątek, który był w posiadaniu mutexa nie oddał go explicite przez 
ReleaseMutex(), to najprawdopodobniej został zakończony w jakiś nieprzewidziany sposób). 
Mutexy są w działaniu bardzo podobne do semaforów. 

\begin{scriptsize}
\begin{verbatim}
/*
 * Wykorzystanie mutexów do synchronizacji wątków
 */
void main(void)  
{  
      HANDLE hThread[2];  
      DWORD threadID1, threadID2;  

      char szFileName="c:\\myfolder\\myfile.txt";  

      hMutex=CreateMutex(NULL, TRUE, "FILE_EXISTS");  

      // tworzymy dwa wątki które czekają na utworzenie pliku  

      hThread[0]=CreateThread(NULL, 0, ThreadProc1, &hMutex, 0, &threadID1);  
      hThread[1]=CreateThread(NULL, 0, ThreadProc2, &hMutex, 0, &threadID1);  

      HANDLE hFile=CreateFile(szFileName, GENERIC_WRITE, 0, &security, . . .);  

      // kod wypełniający plik danymi np. WriteFile(...)  

  
      // sygnalizacja wątkom tego, że dane są gotowe  
      // wątki od ich utworzenia tylko na to czekały   

      ReleaseMutex(hMutex);  
      WaitForMultipleObjects(2, hThread, TRUE, _czas_czekania_);  

      CloseHandle(hMutex);  
      CloseHandle(hFile);  
      CloseHandle(hThread[0]);  
      CloseHandle(hThread[1]);  
}  

DWORD ThreadProc1(LPVOID* arg)  
{  
      HANDLE hMutex = (HANDLE)(*arg);  
      WaitForSingleObject(hMutex, INFINITE);  

      // i czyta dane zapisane do pliku  
      HANDLE hAnswerFile = ::CreateFile(szFileName, GENERIC_READ, 0, NULL, ...);

      // przetwarzaj dane  
      ReleaseMutex(hMutex);  
      return TRUE;  
}  

DWORD ThreadProc2(LPVOID* arg)  
{  
      HANDLE hMutex = (HANDLE)(*arg);  
      WaitForSingleObject(hMutex, INFINITE);  

      // i czyta dane zapisane do pliku  
      HANDLE hAnswerFile = ::CreateFile(szFileName, GENERIC_READ, 0, NULL, ...);

      // przetwarzaj dane  
      ReleaseMutex(hMutex);  
      return TRUE;  
}  
\end{verbatim}
\end{scriptsize}

\subsubsection{Semafory}

Semafory mogą być wykorzystywane tam, gdzie zasób dzielony jest na ograniczoną ilość użytkowników. 
Semafor działa jak furtka kontrolująca ilość wątków wykonujących jakiś fragment kodu. 
Za pomocą semaforów aplikacja może kontrolować na przykład maksymalną ilość otwartych plików, 
czy utworzonych okien. Semafory są w działaniu bardzo podobne do mutexów. 

Nowy semafor tworzony jest w funkcji 

\begin{scriptsize}
\begin{verbatim}
HANDLE CreateSemaphore(

    LPSECURITY_ATTRIBUTES lpSemaphoreAttributes,	
    LONG lInitialCount,	// początkowa wartość licznika
    LONG lMaximumCount,	// maksymalna wartość licznika
    LPCTSTR lpName 	// nazwa
   );
\end{verbatim}
\end{scriptsize}

Wątek tworzący semafor specyfikuje wartość wstępną i maksymalną licznika. 
Inne wątki uzyskują dostęp do semafora za pomocą funkcji 

\begin{scriptsize}
\begin{verbatim}
HANDLE OpenSemaphore(

    DWORD dwDesiredAccess,	// dostęp
    BOOL bInheritHandle,	// dziedziczenie
    LPCTSTR lpName 	// nazwa
   );
\end{verbatim}
\end{scriptsize}

i czekają na wejście za pomocą funkcji ... (to już powinno być jasne jakiej). 

Po zakończeniu pracy w sekcji krytycznej wątek uwalnia semafor za pomocą funkcji 

\begin{scriptsize}
\begin{verbatim}
BOOL ReleaseSemaphore(

    HANDLE hSemaphore,	// uchwyt
    LONG lReleaseCount,	// wartość dodawana do licznika
    LPLONG lpPreviousCount 	// otrzymuje poprzednią wartość licznika
   );	
 \end{verbatim}
\end{scriptsize}

Wątki nie wchodzą w posiadanie semaforów! W przypadku mutexów, 
jeśli wątek zażąda po raz kolejny dostępu do tego mutexu, którego jest już właścicielem, 
dostęp taki zostaje mu przyznany natychmiast. 

Jeśli wątek nagle rozpocznie czekanie na ten sam semafor, to semafor zachowuje się tak, 
jakby wejścia zażądał każdy inny wątek. Inaczej wygląda także sprawa uwalniania semaforów i mutexów: 
mutex może być uwolniony tylko przez wątek, który jest jego właścicielem, 
licznik semafora może być zwiększony przez dowolny wątek, który z tego semafora korzysta. 

\begin{scriptsize}
\begin{verbatim}
/*
 * Wykorzystanie semaforów do synchronizacji wątków
 */
void main(void)  
{  
      HANDLE hThread[2];  
      DWORD threadID1, threadID2;  
      char szFileName="c:\\myfolder\\myfile.txt";  

      hSemaphore=CreateSemaphore(NULL, 0, 1, "FILE_EXISTS");  

      // tworzymy dwa wątki które czekają na utworzenie pliku  
      hThread[0]=CreateThread(NULL, 0, ThreadProc1, &hSemaphore, 0, &threadID1);  
      hThread[1]=CreateThread(NULL, 0, ThreadProc2, &hSemaphore, 0, &threadID1);  

      HANDLE hFile=CreateFile(szFileName, GENERIC_WRITE, 0, &security, . . .);  

      // kod wypełniający plik danymi np. WriteFile(...)  

      // sygnalizacja wątkom tego, że dane są gotowe  
      // wątki od ich utworzenia tylko na to czekały   

      ReleaseSemaphore(hSemaphore, 1, NULL);  
      WaitForMultipleObjects(2, hThread, TRUE, _czas_czekania_);  

      CloseHandle(hSemaphore);  
      CloseHandle(hFile);  
      CloseHandle(hThread[0]);  
      CloseHandle(hThread[1]);  
}  
 
DWORD ThreadProc1(LPVOID* arg)  
{  
      HANDLE hSem = OpenSemaphore( SEMAPHORE_ALL_ACCESS, "FILE_EXISTS");  
      WaitForSingleObject(hSem, INFINITE);  

      // i czyta dane zapisane do pliku  

      HANDLE hAnswerFile = ::CreateFile(szFileName, GENERIC_READ, 0, NULL, ...);

      // przetwarzaj dane  

      ReleaseSemaphore(hSem, 1, NULL);  
      return TRUE;  
}  

DWORD ThreadProc2(LPVOID* arg)  
{  
      HANDLE hSem = OpenSemaphore( SEMAPHORE_ALL_ACCESS, "FILE_EXISTS");  
      WaitForSingleObject(hSem, INFINITE);  

      // i czyta dane zapisane do pliku  

      HANDLE hAnswerFile = ::CreateFile(szFileName, GENERIC_READ, 0, NULL, ...);

      // przetwarzaj dane  

      ReleaseSemaphore(hSem, 1, NULL);  
      return TRUE;  
}  
\end{verbatim}
\end{scriptsize}

\subsubsection{Sekcja krytyczna} 

Interfejs programowania Win32API udostępnia typ danych CRITICAL\_SECTION, 
który wraz z odpowiednim zestawem funkcji może być wykorzystany do implementacji sekcji krytycznej. 

\begin{scriptsize}
\begin{verbatim}
/*
 * Wykorzystanie sekcji krytycznej do synchronizacji wątków
 */
#include <windows.h>  
#include <stdio.h>  
#include <stdlib.h>  
#include <assert.h>  

#define MAXTRY 3  

CRITICAL_SECTION cs;          // dzielona na wszystkie wątki  
  
// główny wątek programu  
void ThreadMain(char *name)  
{  
      int i;  

      for (i=0; i<MAXTRY; i++)  
      {  
            EnterCriticalSection(&cs);  

            /* proszę spróbować też zamiast 
	       powyższej linii napisać 
			
   			  while ( TryEnterCriticalSection(&cs)==FALSE )  
              {  
                  printf(“%s, czekam na wejście\n”, name);   
                   Sleep(5); 
              }
              
			  uwaga! - tylko na WinNT
            */  

            printf(“%s, jestem w sekcji krytycznej!\n”, name);  
            Sleep(5);  
            LeaveCritcalSection(&cs);  

            printf(“%s, wyszedłem z sekcji krytycznej!\n”, name);  
      }  

}  

// tworzy wątek potomny  

HANDLE CreateChild(char* name)  
{  
      HANDLE hThread; DWORD dwId;  
      hThread = CreateThread(NULL, 0, 
                (LPTHREAD_START_ROUTINE)ThreadMain, 
		(LPVOID)name, 0, &dwId);  

      assert(hThread!=NULL); return hThread;            
}  

int main(void)  
{  
      HANDLE hT[4]; 

      InitializeCriticalSection(&cs);  

      hT[0]=CreateChild(“Jurek”);  
      hT[1]=CreateChild(“Ogórek”);  
      hT[2]=CreateChild(“Kiełbasa”);  
      hT[3]=CreateChild(“Sznurek”);  

      WaitForMultipleObjects(4, hT, TRUE, INFINITE);  

      CloseHandle(hT[0]);CloseHandle(hT[1]);  
      CloseHandle(hT[2]);CloseHandle(hT[3]);  

      DeleteCriticalSection(&cs);  

      return 0;  
}  
\end{verbatim}
\end{scriptsize}

