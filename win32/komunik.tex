\section{Komunikacja między procesami}
\label{apiKomunikacja}

Zajmiemy się dokładniej komunikacją za pomocą tzw. gniazd. 
Sama idea gniazd została opracowana na Uniwersytecie Kalifornijskim w Berkley w celu 
zapewnienia możliwości komunikacyjnych w systemie Unix. Opracowany tam interfejs programowania 
nosi nazwę "Berkley socket interface" i jest stopniowo z mniejszymi lub większymi 
zmianami przejmowany przez kolejne systemy operacyjne. 

Największa zaleta jaką daje korzystanie z gniazd to w miarę prosty i przejrzysty interfejs 
niezależny od warstwy komunikacyjnej (oczywiście stosunkowo najmniej wygodnie korzysta 
się z interfejsu gniazd w czystym C, w Javie czy C\# jest jeszcze prościej). 

\subsection{Charakterystyka protokołów sieciowych}

Skoro gniazda są tylko interfejsem programowania służącym do oprogramowania transmisji 
sieciowych, to zajmijmy się przez chwilę samymi protokołami\footnote{Mowa na razie
o protokołach {\em fizycznych}, czyli sposobach transmisji danych między różnymi
maszynami w sieci. Jeśli przesyłane dane są w jakiś sposób interpretowane, to mamy
do czynienia z protokołem {\em logicznym}.} i ich charakterystykami. 

Bez wchodzenia w szczegóły, możemy podzielić protokoły na 
\begin{itemize}
  \item oparte o wiadomości (ang. message-oriented). Takim mianem określamy protokoły, które 
        przesyłają dane w paczkach o skończonej pojemności. Nadawca wielokrotnie 
	wysyła małe paczki informacji, odbiorca zaś musi wielokrotnie te paczki odczytywać. 
	Wyobraźmy sobie, że nadawca wysyła 10 paczek po 64 znaki. Mimo że w warstwie 
	transmisji protokół może zdecydować o połączeniu tych wiadomości w jedną paczkę, 
	po dotarciu do odbiorcy wiadomości mogą być rozdzielone i przekazane w takich samych 
	paczkach, w jakich były nadawane (choć niekoniecznie). 
  \item strumieniowe (ang. stream-based). Tutaj proces nadawania trwa nieustannie, zaś odbiorca 
        w danej chwili dostaje tylko informacji, ile w danej chwili do niego dotarło. 
\end{itemize}

Inna ważna linia dzieli protokoły na 
\begin{itemize}
  \item wykorzystujące bezpośrednie połączenie (ang. connection-oriented). 
        Te protokoły przed wysłaniem czy odebraniem jakichkolwiek informacji 
	nawiązują bezpośrednie połączenie między nadawcą i odbiorcą. Chodzi o 
	zagwarantowanie tego, że oba zainteresowane podmioty istnieją i są 
	połączone ścieżką gotową do transmisji danych. 
  \item bezpołączeniowe (ang. connectionless). Nadawca nie ma tutaj nawet gwarancji, 
        że odbiorca istnieje. Trochę przypomina to usługi poczty - nadawca adresuje 
	wiadomość i wysyła ją, nie wie jednak czy odbiorca na tę wiadomość czeka ani 
	czy wiadomość do odbiorcy dotrze. 
\end{itemize}

Każdy protokół może charakteryzować się posiadaniem lub brakiem następujących cech:
\begin{itemize}
  \item wiarygodność (ang. reliability). Wiarygodny protokół musi zapewniać dotarcie każdego 
        bajtu informacji od nadawcy do odbiorcy, 
  \item zachowywanie porządku (ang. ordering). Protokół który zachowuje porządek, dba o to 
        by informacje docierały do odbiorcy w takiej samej kolejności w jakiej były nadawane, 
  \item łagodne zakończenie (ang. graceful close). Kiedy któraś ze stron zamierza zamknąć 
        połączenie, protokół może dać obu stronom szansę na doczytanie informacji, które są jeszcze w drodze. 
  \item otwarta transmisja (ang. broadcast data). Protokół może transmitować dane tak, aby 
        docierały do wszystkich maszyn w sieci. Wada takich protokołów polega na tym, że 
	wszystkie maszyny w sieci muszą tracić czas na analizę przesyłanych informacji, mimo 
	że niekoniecznie muszą być nimi zainteresowane. 
  \item niezależność od warstwy komunikacyjnej (ang. routability). Transmisja w takich protokołach 
        nie zależy od tego ile i jakie maszyny stoją pomiędzy nadawcą i odbiorcą. 
	Możliwe jest nawet, by kolejne informacje docierały do odbiorcy inną drogą. 
\end{itemize}

\subsection{Podstawy biblioteki Winsock}

Gniazda zaadaptowano do systemu Windows pod nazwą Winsock. 
Za pomocą tej biblioteki można tworzyć aplikacje korzystające z wielu protokołów sieciowych. 
Dzięki prostemu interfejsowi, z perspektywy programisty korzystanie z różnych 
protokołów wygląda niemal identycznie. 

Przy nawiązywaniu połączeń jedna strona jest "serwerem", który oczekuje połączenia, 
druga strona jest "klientem", który nawiązuje połączenie z "serwerem". Protokół fizyczny określa
sposób transmisji danych między klientem a serwerem (sposób w jaki dane
są przesyłane). Protokół fizyczny to jednak nie wszystko - gwarancja, że dane zostały przesłane nie
oznacza, że przekaz został zrozumiany. Wyobraźmy sobie na przykład, że klient i serwer 
komunikują się za pomocą protokołu fizycznego
TCP/IP i że klient wysłał do serwera strumień danych {\em "QWERTY"}. Co serwer ma zrobić z takim przekazem?
Otóż za interpretację przesyłanych danych odpowiadaja coś co moglibyśmy nazwać protokołem logicznym, 
czyli pewien umówiony zestaw komunikatów rozumianych przez obie strony transmisji. Kilka powszechnie
znanych przykładów protokołów logicznych to HTTP, FTP, TELNET.

Jeden fizyczny komputer może pełnić rolę serwera sieciowego dla wielu różnych protokołów logicznych.
W przypadku protokołu TCP/IP istnieje możliwość zdefiniowania aż 65535 rozłącznych serwerów wirtualnych,
oczekujących na połączenia z klientami. Numer takiego wirtualnego serwera zwykle nazywa się 
{\em portem}\footnote{Słowo {\em port} określa tu więc tylko numer za pomocą którego klienci mogą rozróżniać 
wirtualne serwery oczekujące na połączenia na tej samej maszynie. Nie ma to nic wspólnego z żadnymi
fizycznymi portami komputera.}.

Aby uporządkować możliwy bałagan jaki daje dowolność wyboru portu dla serwera nasłuchującego połączeń, usługi
typowe mają ściśle przyporządkowane numery portów. Na przykład łącząc się z jakąś maszyną na
port o numerze 80 możemy być niemal pewni, że skomunikujemy się z serwerem HTTP, zaś na porcie
21 oczekuje TELNET, a na porcie 25 FTP\footnote{Jest to jednak tylko umowa. Nie ma przeszkód w uruchomieniu
serwera HTTP na porcie powiedzmy 3333. Przeglądarki Internetowe domyślnie łączą się właśnie do portu
80 maszyny docelowej, jednak istnieje możliwość wymuszenia innego numeru portu, na przykład
{\em http://www.qwe.com:3333}.}. Zobaczmy więc najpierw jak przeglądnąć porty lokalnej maszyny
w poszukiwaniu serwerów oczekujących na połączenia.

\begin{scriptsize}
\begin{verbatim}
#include <stdio.h>
#include <winsock2.h>

int ssocket,new_socket;
WSADATA wsd;
struct  sockaddr_in addr;
int     sourceport;

int main()
{
 printf( "TCP/IP port status:\n" );

 if (WSAStartup(MAKEWORD(2,2), &wsd) != 0)
 {
   printf("Błąd ładowania Winsock 2.2!\n");
   return 1;
 }

 for ( sourceport=0; sourceport<65535; sourceport++ )
 {
  ssocket=socket(AF_INET,SOCK_STREAM,IPPROTO_TCP); 
  addr.sin_family=AF_INET; 
  addr.sin_addr.s_addr=htonl(INADDR_ANY); 
  addr.sin_port=htons( (unsigned short)sourceport); 
  
  if (bind(ssocket,(struct sockaddr*)&addr,sizeof(addr)))
  {
   printf("Otwarty port %d\n", sourceport);
  } 
  shutdown(ssocket, SD_BOTH); 
  closesocket(ssocket);
 }

 return 0;
}
\end{verbatim}
\end{scriptsize}

Jeśli próba połączenia gniazda z adresem o ustalonym porcie kończy się niepowodzeniem, to 
przyjmujemy, że port jest zajęty\footnote{Tak naprawdę mogą być inne przyczyny błędu
funkcji bind(), jednak możemy je pominąć.}, jednak nie ma możliwości określenia jakiego
protokołu logicznego spodziewa się serwer nasłuchujący na określonym porcie. Jak zobaczymy
bowiem w poniższym przykładzie, serwer może w ogóle nie posługiwać się żadnym protokołem logicznym.

Kod programu serwera: 

\begin{scriptsize}
\begin{verbatim}
// prosty moduł serwera
// komunikacji za pomocą Winsock
// użycie: server.exe

#include <winsock2.h>
#include <stdio.h>
#include <stdlib.h>

#define DEFAULT_PORT 5000
#define DEFAULT_BUFFER 4096

// tylko Visual C++
#pragma comment(lib, "ws2_32.lib")

// funkcja: wątek do komunikacji z klientem
DWORD WINAPI ClientThread(LPVOID lpParam)
{
    SOCKET sock = (SOCKET)lpParam;
    char szBuf[DEFAULT_BUFFER];
    int ret,
        nLeft,
        idx;

// serwer będzie oczekiwał na informacje od klienta
while(1)
{
    // najpierw odbierz dane 
    ret = recv(sock, szBuf, DEFAULT_BUFFER, 0);
    if (ret == 0)
        break; 
    else if (ret == SOCKET_ERROR)
    {
        printf("błąd funkcji recv(): %d\n", WSAGetLastError());
        break; 
    } 

    szBuf[ret] = '\0';
    printf("RECV: '%s'\n", szBuf);

    // następnie odeślij te dane, poporcjuj jeśli trzeba
    // (niestety send() może nie wysłać wszystkiego)
    nLeft = ret;
    idx = 0;
    while(nLeft > 0)
    {
        ret = send(sock, &szBuf[idx], nLeft, 0);
        if (ret == 0)
            break;
        else if (ret == SOCKET_ERROR)
        {
            printf("błąd funkcji send(): %d\n", WSAGetLastError());
            break; 
        } 
        nLeft -= ret;
        idx += ret;
    }
} 
return 0; 
}

int main(int argc, char *argv[])
{
    WSADATA wsd;
    SOCKET sListen,
    sClient;
    int iAddrSize;
    HANDLE hThread;
    DWORD dwThreadID;
    struct sockaddr_in local, client; 
    struct hostent *host = NULL;

// inicjuj Winsock 2.2 
if (WSAStartup(MAKEWORD(2,2), &wsd) != 0)
{
    printf("Błąd ładowania Winsock 2.2!\n");
    return 1;
}

// twórz gniazdo do nasłuchu połączeń klientów
sListen = socket(AF_INET, SOCK_STREAM, IPPROTO_IP);
if (sListen == SOCKET_ERROR) 
{
    printf("Błąd funkcji socket(): %d\n", WSAGetLastError());
    return 1;
}

// wybierz interfejs (warstwę komunikacyjną)
local.sin_addr.s_addr = htonl(INADDR_ANY);
local.sin_family = AF_INET;
local.sin_port = htons(DEFAULT_PORT); 
if (bind(sListen, (struct sockaddr *)&local, sizeof(local)) == SOCKET_ERROR)
{
    printf("Błąd funkcji bind(): %d\n", WSAGetLastError());
    return 1;
}

// nasłuch
host = gethostbyname("localhost");
if (host == NULL)
{
    printf("Nie udało się wydobyć nazwy serwera\n");
    return 1;
}

listen(sListen, 8);
printf("Serwer nasłuchuje.\n");
printf("Adres: %s, port: %d\n", host->h_name, DEFAULT_PORT);

// akceptuj nadchodzące połączenia
while (1)
{
    iAddrSize = sizeof(client);
    sClient = accept(sListen, (struct sockaddr *)&client, &iAddrSize);
    if (sClient == INVALID_SOCKET)
    {
        printf("Błąd funkcji accept(): %d\n", WSAGetLastError());
        return 1;
    }
    printf("Zaakceptowano połączenie: serwer %s, port %d\n", 
        inet_ntoa(client.sin_addr), ntohs(client.sin_port));
    hThread = CreateThread(NULL, 0, ClientThread,
        (LPVOID)sClient, 0, &dwThreadID);
    if (hThread == NULL)
    {
        printf("Błąd funkcji CreateThread(): %d\n", WSAGetLastError());
        return 1;
    } 
    CloseHandle(hThread); 
} 
closesocket(sListen);

WSACleanup();
return 0;
}
\end{verbatim}
\end{scriptsize}

Kod programu klienta: 

\begin{scriptsize}
\begin{verbatim}
// prosty moduł klienta
// komunikacji za pomocą Winsock
// użycie: klient.exe -s:IP

#include <winsock2.h>
#include <stdio.h>
#include <stdlib.h>

#define DEFAULT_COUNT 5
#define DEFAULT_PORT 5000
#define DEFAULT_BUFFER 4096
#define DEFAULT_MESSAGE "Wiadomość testowa"

// tylko Visual C++
#pragma comment(lib, "ws2_32.lib")

char szServer[128], szMessage[1024];

// funkcja sposob_uzycia
void sposob_uzycia()
{
    printf("Klient.exe -s:IP\n");
    ExitProcess(1);
}

void WalidacjaLiniiPolecen(int argc, char **argv)
{
    int i;

    if (argc < 2)
    {
        sposob_uzycia();
    }

    for (i=1; i<argc; i++)
    {
        if (argv[i][0] == '-')
        {
        switch (tolower(argv[i][1]))
        {
        case 's':
            if (strlen(argv[i]) > 3)
            strcpy(szServer, &argv[i][3]);
            break;
        default:
            sposob_uzycia();
            break;
        }
        }
    }
}

int main(int argc, char *argv[])
{
    WSADATA wsd;
    SOCKET sClient;
    char szBuffer[DEFAULT_BUFFER];
    int ret, i;
    struct sockaddr_in server;
    struct hostent *host = NULL;

// linia poleceń

WalidacjaLiniiPolecen(argc, argv);

// inicjuj Winsock 2.2 
if (WSAStartup(MAKEWORD(2,2), &wsd) != 0)
{
    printf("Błąd ładowania Winsock 2.2!\n");
    return 1;
}

strcpy(szMessage, DEFAULT_MESSAGE);

// twórz gniazdo do nasłuchu połączeń klientów
sClient = socket(AF_INET, SOCK_STREAM, IPPROTO_IP);
if (sClient == INVALID_SOCKET) 
{
    printf("Błąd funkcji socket(): %d\n", WSAGetLastError());
    return 1;
}

// wybierz interfejs 
server.sin_addr.s_addr = inet_addr(szServer);
server.sin_family = AF_INET;
server.sin_port = htons(DEFAULT_PORT); 

// jeśli adres nie był w postaci xxx.yyy.zzz.ttt
// to spróbuj go wydobyć z postaci słownej

if (server.sin_addr.s_addr == INADDR_NONE)
{
    host = gethostbyname(szServer);
    if (host == NULL)
    {
        printf("Nie udało się wydobyć nazwy serwera: %s\n", szServer);
        return 1;
    }
    CopyMemory(&server.sin_addr, host->h_addr_list[0], host->h_length);
}

if (connect(sClient, (struct sockaddr *)&server, sizeof(server)) == SOCKET_ERROR)
{
    printf("Błąd funkcji connect(): %d\n", WSAGetLastError());
    return 1;
}

// wysyłaj i odbieraj dane

for (i=0; i<DEFAULT_COUNT; i++)
{
    ret = send(sClient, szMessage, strlen(szMessage), 0);
    if (ret == 0)
        break;
    else if (ret == SOCKET_ERROR)
    {
        printf("Błąd funkcji send(): %d\n", WSAGetLastError());
        return 1;
    }
    printf("Wysłano %d bajtów\n", ret);

    ret = recv(sClient, szBuffer, DEFAULT_BUFFER, 0);
    if (ret == 0)
        break;
    else if (ret == SOCKET_ERROR)
    {
        printf("Błąd funkcji recv(): %d\n", WSAGetLastError());
        return 1;
    }
    szBuffer[ret] = '\0';
    printf("RECV [%d bajtów]: '%s'\n", ret, szBuffer);
}
closesocket(sClient); 

WSACleanup();
return 0;
}
\end{verbatim}
\end{scriptsize}

Serwer w pętli oczekuje na klientów i po nawiązaniu połączenia tworzy nowy wątek, 
który zajmuje się odbieraniem komunikatów i odsyłaniem ich z powrotem (to tylko przykład!). 
Klient wymaga oczywiście parametru, którym jest nazwa serwera. 
Klient nawiązuje połączenie i wysyła do serwera komunikat, po czym czeka na jego echo. 

Struktura obu programów jest dość podobna. Oba tworzą odpowiednie gniazda, przy czym: 
\begin{itemize}
  \item serwer najpierw przypisuje gniazdu adres (za pomocą funkcji bind()), 
        a następnie wchodzi w tryb nasłuchu (listen()), gdzie zatrzymuje się czekając 
	na połączenia klientów (accept()). 
  \item klient nawiązuje połączenie z serwerem za pomocą funkcji connect(). 
\end{itemize}

Oba programy wymieniają się krótkimi informacjami za pomocą funkcji send() i recv(). 
Serwer nasłuchuje na zadanym porcie, zaś po nawiązaniu połączenia z klientem tworzy nowe 
gniazdo i nowy wątek, po czym komunikuje się przez nowo utworzone gniazdo w nowym wątku. 
Dzięki temu serwer może obsługiwać wielu klientów jednocześnie, bez względu na to jak długo 
trwałoby obsługiwanie pojedyńczego klienta. Podsumowanie schematów działania serwera i klienta
znajdują się w tabelach \ref{tab:TCPServer} i \ref{tab:TCPKlient}.

\begin{table}[ht]
	\begin{center}

	 \mbox{socket()}                                   \\ $\downarrow$ \\
	\framebox[7cm]{Tworzenie gniazda}                  \\ $\downarrow$ \\
	 \mbox{bind()}                                     \\ $\downarrow$ \\
	\framebox[7cm]{Przypisywanie adresu}               \\ $\downarrow$ \\
	 \mbox{listen()}                                   \\ $\downarrow$ \\
	\framebox[7cm]{Nasłuch połączeń od klientów}       \\ $\downarrow$ \\
	 \mbox{accept()}                                   \\ $\downarrow$ \\
	\framebox[7cm]{Akceptacja połączenia klienta}      \\ $\downarrow$ \\
	 \mbox{send(), recv()} \\ 

	\end{center}
\caption{Schemat serwera TCP korzystającego z gniazd} 
\label{tab:TCPServer}
\end{table}

\begin{table}[ht]
	\begin{center}

	 \mbox{socket()}                                   \\ $\downarrow$ \\
	\framebox[7cm]{Tworzenie gniazda}                  \\ $\downarrow$ \\
	 \mbox{connect()}                                  \\ $\downarrow$ \\
	\framebox[7cm]{Połączenie z serwerem}              \\ $\downarrow$ \\
	 \mbox{send(), recv()} \\ 

	\end{center}
\caption{Schemat klienta TCP korzystającego z gniazd} 
\label{tab:TCPKlient}
\end{table}

Jak w związku z tym napisać własną przegląrarkę Internetową? Otóż kod programu klienta
musiałby łączyć się do wybranego serwera do portu 80, a następnie wysyłać polecenia
protokołu HTTP, na przykład {\bf GET index.html}, po którym serwer odpowiedziałby
przesyłając strumień danych, będący kodem strony {\bf index.html}. Kod tej strony należałoby
następnie sparsować i udostępnić użytkownikowi, wyławiając przy okazji hyperlinki, dzięki
którym użytkownik mógłby wydawać programowi kolejne polecenia.

A jak napisać własny serwer WWW? Otóz kod programu serwera musiałby oczekiwać na połączenia
klientów i reagować na dobrze sformułowane polecenia protokołu HTTP. Na przykład, gdyby polecenie
od klienta brzmiało {\bf GET index.html}, serwer musiałby z dysku odczytać zawartość pliku
{\bf index.html} i wysłać go do połączonego klienta. 

W praktyce napisanie dobrej przeglądarki Internetowej czy dobrego serwera WWW mogłoby być dość żmudne, jednak
bardzo ograniczone możliwości można uzyskać niewielkim nakładem pracy. 

\subsubsection{Gniazda asynchroniczne}

Wywołania funkcji operujących na gniazdach są najczęściej {\em blokujące}, to znaczy że
wykonanie kodu zatrzymuje się na wywołaniu funkcji na tak długo, aż działanie takiej funkcji
zakończy się. Może to powodować mnóstwo kłopotów przy tworzeniu programów okienkowych - okno
przestanie reagować na komunikaty, ponieważ kod zatrzyma się na wykonaniu funkcji obsługującej gniazda.

Aby poradzić sobie z tym problemem można zażądać od WinSock {\em asynchronicznej} obsługi gniazd, to
znaczy informowania o zdarzeniach związanych z gniazdami za pomocą komunikatów. 
Asynchroniczny tryb pracy WinSock ustala się za pomocą funkcji 

\begin{scriptsize}
\begin{verbatim}
int WSAAsyncSelect (

    SOCKET s,	
    HWND hWnd,	
    unsigned int wMsg,	
    long lEvent	
   );
\end{verbatim}
\end{scriptsize}
