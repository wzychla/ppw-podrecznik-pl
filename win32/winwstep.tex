\section{Fundamentalne idee Win32API}

Interfejs programowania Win32API można podzielić na spójne podzbiory funkcji przeznaczonych
do podobnych celów. Dokumentacja systemu mówi o 6 kategoriach:

\begin{description}
\item[Usługi podstawowe] Ta grupa funkcji pozwala aplikacjom na korzystanie z takich możliwości
systemu operacyjnego jak zarządzanie pamięcią, obsługa systemu plików i urządzeń zewnętrznych,
zarządzanie procesami i wątkami.
\item[Biblioteka Common Controls]Ta część Win32API pozwala obsługiwać zachowanie typowych
okien potomnych, takich jak proste pola edycji i comboboxy czy skomplikowane ListView i TreeView.
\item[GDI]GDI ({\em Graphics Device Interface}) dostarcza funkcji i struktur danych, które
mogą być wykorzystane do tworzenia efektów graficznych na urządzeniach wyjściowych takich jak
monitory czy drukarki. GDI pozwala rysować kształty takie jak linie, krzywe oraz figury zamknięte,
pozwala także na rysowanie tekstu.
\item[Usługi sieciowe]Za pomocą tej grupy funkcji można obsługiwać warstwę komunikacji sieciowej, na
przykład tworzyć współdzielone zasoby sieciowe czy diagnozować stan konfiguracji sieciowej.
\item[Interfejs użytkownika]Ta grupa funkcji dostarcza środków do tworzenia i zarządzania interfejsem
użytkownika: tworzenia okien i interakcji z użytkownikiem. Zachowanie i wygląd tworzonych okien jest 
uzależnione od właściwości tzw.{\em klas okien}. 
\item[Powłoka systemu]To funkcje pozwalające aplikacjom integrować się z powłoką systemu, na przykład
uruchomić dany dokument ze skojarzoną z nim aplikacją, dowiadywać się o ikony skojarzone z plikami i
folderami czy odczytywać położenie ważnych folderów systemowych.
\end{description}

Programowanie systemu Windows wymaga przyswojenia sobie trzech istotnych elementów.

Po pierwsze - wszystkie elementy interfejsu użytkownika, pola tekstowe, przyciski,
comboboxy, radiobuttony\footnote{'Angielskawe' brzmienie tych terminów może być
trochę niezręczne, jednak ich polskie odpowiedniki bywają przerażające. Pozostaniemy
więc przy terminach powszechnych wśród programistów.}, 
wszystkie one z punktu widzenia systemu są oknami. Jak zobaczymy,
Windows traktuje wszystkie te elementy w sposób jednorodny, przy czym niektóre okna mogą być tzw.
{\em oknami potomnymi} innych okien. Windows traktuje okna potomne w sposób szczególny, zawsze
umieszczając je w obszarze okna macierzystego oraz automatycznie przesuwając je, gdy użytkownik
przesuwa okno macierzyste\footnote{To dość ważne. Gdyby programista musiał dbać o przesuwanie się
okien potomnych za przesuwającym się oknem macierzystym, byłoby to niesłychanie niewygodne.}.

Po drugie - z perspektywy programisty wszystkie okna zachowują się prawie dokładnie tak samo
jak z perspektywy użytkownika. Użytkownik, za pomocą myszy, klawiatury lub innego wskaźnika, wykonuje
różne operacje na widocznych na pulpicie oknach. Każde zdarzenie w systemie, bez względu na źródło jego
pochodzenia, powoduje powstanie tzw. {komunikatu}, czyli pewnej informacji mającej swój cel i niosącej
jakąś określoną informację. Programista w kodzie swojego programu tak naprawdę 
zajmuje się obsługiwaniem komunikatów, które powstają w systemie przez interakcję 
użytkownika\footnote{I nie tylko - komunikaty mogą mieć swoje źródło w samym systemie. 
Komunikaty wysyłają do siebie na przykład okna i okna potomne, źródłem komunikatów mogą być zegary itd.}.

Po trzecie - do identyfikacji obiektów w systemie, takich jak okna, obiekty GDI, pliki, biblioteki, 
wątki itd., Windows korzysta z tzw. {\em uchwytów} (czyli 32-bitowych identyfikatorów). Mnóstwo
funkcji Win32API przyjmuje jako jeden z parametrów uchwyt (czyli identyfikator) obiektu systemowego, przez
co wykonanie takiej funkcji odnosi się do wskazanego przez ten uchwyt obiektu. W języku C różne uchwyty
zostały różnie nazwane (HWND, HDC, HPEN, HBRUSH, HICON, HANDLE itd.) choć tak naprawdę są one najczęściej
wskaźnikami na miejsce w pamięci gdzie znajduje się pełny opis danego obiektu. Z perspektywy programisty, są
one, jak już powiedziano, unikatowymi identyfikatorami obiektów systemowych. 

Dokładne poznanie i zrozumienie trzech wymienionych wyżej elementów stanowi istotę poznania i
zrozumienia Win32API. Idee które leżą u podstaw wyżej wymienionych elementów są jednakowe
we wszystkich wersjach systemu Windows i z dużą dozą prawdopodobieństwa można powiedzieć, że nie 
ulegną zasadnicznym zmianom w kolejnych wersjach systemu. Programista może oczywiście znać mniej
lub więcej funkcji Win32API, umieć posługiwać się mniejszą lub większą ilością komunikatów, znać
mniej lub więcej typów uchwytów, jednak bez zrozumienia zasad, wedle jakich wszystkie te elementy
składają się na funkcjonowanie systemu operacyjnego Windows, programista pisząc program 
będzie często bezradny.

