\subsection{Subclassowanie okien}

Często przechwytywanie zdarzeń nie wystarcza, a programista chciałby sięgnąć głębiej, aż na poziom
komunikatów. 

\subsubsection{Obsługa komunikatów własnych okien}

Obsługa komunikatów nie nastręcza żadnych kłopotów jeśli to programista tworzy klasę
opisującą okno. Wystarczy po prostu przeciążyć metodę {\bf WndProc}.

\begin{scriptsize}
\begin{verbatim}
/* Wiktor Zychla, 2003 */
using System;
using System.Drawing;
using System.Windows.Forms;

namespace Example
{
  public class CMainForm : Form
  {   
    const int WM_LBUTTONDBLCLK = 0x0203;

    protected override void WndProc( ref Message m )
    {
       switch ( m.Msg )
       {
         case WM_LBUTTONDBLCLK : MessageBox.Show( "Dwuklik!" ); break;
       }
       base.WndProc( ref m );
    }

    public CMainForm()
    {
      this.Text = "Okno"; 
      this.Size = new Size( 200, 200 );
    }

    public static void Main()
    {
       Application.Run( new CMainForm() );
    }
  }
}
\end{verbatim}
\end{scriptsize}

Struktura {\bf Message} przechowuje w sobie wszystkie parametry komunikatu ({\bf LParam}, {\bf WParam}, itd.),
ma również metodę {\bf GetLParam}, która służy do rzutowania parametru przekazanego w {\bf LParam} na wskazany
typ.

Warto zwrócić uwagę na konieczność wywołania funkcji {\bf WndProc} z klasy bazowej. Bez tego wywołania okno
nie utworzy się, bowiem zabraknie mu większości potrzebnej funkcjonalności.

\subsubsection{Obsługa komunikatów istniejących okien}

Przedstawiona w poprzednim podrozdziale technika nie nadaje się do obsługi komunikatów w już istniejących
komponentach, na przykład {\bf TextBox} czy {\bf Button}. Możnaby co prawda przeciążyć już istniejący komponent
i dodać obsługę komunikatów do klasy potomnej, ale interesująca byłaby możliwość taka jak opisana na stronie
\pageref{subclassingAPIFunkcje}, czyli dodawanie własnej funkcji obsługi do już istniejącego
okna.

Okazuje się, że i taki scenariusz jest możliwy, wymaga jedynie utworzenia klasy dziedziczącej z klasy
{\bf NativeWindow} i skojarzenia uchwytów okien. 

\begin{scriptsize}
\begin{verbatim}
/* Wiktor Zychla, 2003 */
using System;
using System.Drawing;
using System.Windows.Forms;

namespace Example
{
  public class CSubclass : NativeWindow
  {
    const int WM_LBUTTONDBLCLK = 0x0203;

    protected override void WndProc( ref Message m )
    {
       switch ( m.Msg )
       {
         case WM_LBUTTONDBLCLK : MessageBox.Show( "Dwuklik!" ); break;
       }
       base.WndProc( ref m );
    }

    public CSubclass() {} 
  }

  public class CMainForm : Form
  {   
    CSubclass subclass = new CSubclass();
    TextBox t;

    public CMainForm()
    {
      t = new TextBox();
      this.Controls.Add( t );

      // subclassing okna potomnego
      subclass.AssignHandle( t.Handle );

      this.Text = "Okno"; 
      this.Size = new Size( 200, 200 );      
    }

    public static void Main()
    {
       Application.Run( new CMainForm() );
    }
  }
}
\end{verbatim}
\end{scriptsize}

